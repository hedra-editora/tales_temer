%\textbf{Temer e o fascismo comum }

%\textbf{Sumário}%

%\textbf{Nota introdutória}%

%\textbf{1. Mediocridade, política e violência}%

%\textbf{2. \emph{Ordem} e violência no Brasil}%

%\textbf{3. Tradição da mentira tradição do ódio}%

%\textbf{4. Crise e alucinose, anticomunismo do nada}%

%\textbf{5. A extrema direita de hoje e o Brasil: modos de usar}%

%\textbf{6. Democracia de extermínio?}%

%\textbf{7. O Carnaval da Tortura}%

%\textbf{8. O Estado não está sendo favorável à vida no Brasil}%

%\textbf{9. Neofascismo e o cinema urgente brasileiro }%

%\textbf{10. O tempo é mal e o país partido: estilhaços do Brasil }%

%\textbf{11. Um político preso, um preso político}%

%\textbf{12. Fascismo comum, sonho e história}%

%\textbf{Sobre os textos }

\chapter{Nota introdutória}

Este livro, que nunca deveria ter sido escrito se houvesse alguma razão
melhor na história, é resultado de um processo que se impôs a mim e
muitos de nós. Meu projeto original, que deu início à série que se
completa aqui, era o de escrever apenas sobre o ex"-presidente Lula e a
relação da personalidade do grande político com o poder como ele o
produziu, bem como levar a cabo a avaliação dos níveis de contradição do
que significou \emph{um governo de esquerda de pleno mercado} no mundo
político e social contemporâneo.

Mas o processo acelerado de degradação da política e do quadro da
democracia no Brasil, com a chegada por aqui do impacto maciço da crise
que efetivamente derrubou os mercados mundiais a partir de 2008, e que
levou finalmente, de modo atrasado pela alavancagem lulo"-petista
descrita no primeiro livro da série, à derrubada do governo reeleito de
Dilma Rousseff e ao governo de interesses à direita de Michel Temer no
Brasil, tornou o projeto inicial uma parte menor de um estudo mais amplo
sobre as condições de poder e crise contemporâneas, nas quais
capitalismo e novas modalidades de fascismo estreitam vínculos
novamente, em detrimento da própria dimensão da democracia, entre nós
fraca, irônica e insólita. Contra meu desejo, e de todos com quem
convivo felizmente, a história violenta e bárbara da própria ideia da
política no presente acabou por tomar todo o plano do projeto, se
impondo como o problema real, não apenas deste livro, mas também dos
outros dois que o antecederam, \emph{Lulismo, carisma pop e cultura
anticrítica} e \emph{Dilma Rousseff e o ódio político. }

Este trabalho, ainda mais fortemente que os anteriores, foi escrito
diretamente sobre o tempo aberto de um processo histórico em efetivo
acontecimento, de modo que não há garantia sobre o valor mais constante
daquilo que se buscou destacar aqui como marca significativa da
história, no relato do processo de uma cultura política viva, que tem
efeitos de choque sobre o autor. Por isso, diferentemente dos outros
dois ensaios, este trabalho inclui alguns textos publicados previamente
em revistas e periódicos durante o período da tomada do poder pela nova
direita no Brasil, pós impeachment de Dilma Rousseff. Buscou"-se assim
acompanhar as inquietações e demandas muito presentes no próprio tempo
vivido, da política expressa com forte intensidade na vida. Como grande
parte da política relevante do período passou a ser feita na rua, ou na
rua eletrônica da internet, com movimentos extremados de paixão à
direita, era necessário marcar esta dimensão, muito importante, vida
real da política que a história institucional e econômica das coisas do
poder costuma frequentemente \emph{esquecer}. Os escritos a que me
refiro foram publicados em revistas do campo progressista, e não na
grande imprensa..., revistas que configuraram um campo de opinião que
acompanhou o espaço do avanço da miséria política e o aumento da
violência na vida coletiva cotidiana brasileira.

Apresento também aqui um diário de notação e de espanto político sobre o
processo de ocupação do espaço da cultura pela nova ordem excitada de
paixão conservadora, com sua tendência à ação agressiva, à censura e à
produção maciça de mentiras na internet. No tique taque do dia a dia, o
íntimo e o histórico têm correspondência no processo de subjetivação, e
de subjetivação de um sujeito político, o que me interessa como escritor
e como psicanalista. E apresento também, de início, para manter vivo um
certo plano de investigação anteriormente estabelecido, um retrato,
inédito, do tipo de personagem e subjetividade política que Michel Temer
representa no Brasil. Ou, dito de outro modo, que Brasil representa um personagem
anódino, grave cínico, e, se é alguma liderança, liderança de
que tipo de poder efetivo destas paragens?

Mas o foco central do trabalho é a avalição da nova modalidade de ação
política de ódio e de mentira, nova modalidade de fascismo brasileiro,
que chamei de \emph{comum} porque construído familiarmente na vida
comum, no mundo da vida e nas relações de sociabilidade, e do qual
estudei a sua convocação psíquica de massas. É o mundo de base mesmo do
governo de todo poder ao Capital e da muito medíocre afirmação
conservadora que tomou o Brasil a partir de maio de 2016, que aparece
nestes escritos. O espaço político de tendências autoritárias,
neofascistas, ao modo kitsch local, que busquei caracterizar como um
setor vital da nova direita brasileira. Mas não apenas, já que também
esse é um estranho movimento mundial, no momento histórico da crise
maior do poder e sua reprodução, que, tristemente, quase não pode mais
ser pensado.

\medskip

\begin{flushright}
\emph{São Paulo, abril de 2018}
\end{flushright}

\chapter{Mediocridade, política e violência}

Temer faz parte da estirpe dos homens medíocres do poder brasileiros.
Nada nele é especial, fascinante ou criativo. Nada nele nunca
surpreende, brilha ou dá esperança. Seu mundo é o dos gabinetes e dos
acordos de bastidores. Não há nada a sonhar e nada a esperar a seu
respeito. Seu universo de corpo e espírito, se podemos falar assim a seu
respeito, é o mundo da infraestrutura da política, onde as decisões
indizíveis são tomadas e os acordos das facções da política são feitos,
entre os interesses que podem e os que não podem vir à luz do dia.
Ainda, neste mundo são as mais tradicionais oligarquias políticas
brasileiras, tradicionalmente fisiológicas, patrimonialistas e
antissociais que ele representa, e das quais se tornou um líder. Um
líder vazio.

De fato, nunca imaginei perder algumas horas escrevendo sobre personagem
tão anódino e destituído de graça, tão verdadeiramente desinteressante
do ponto de vista humano, representante daquilo que, apesar do imenso
poder institucional que representa e opera, parece ser apenas o pior
existente na política brasileira há muito e desde sempre. Um pior
bastante comum, diga"-se a seu favor. Eu nunca escreveria sobre um tal
sujeito, se o Brasil não insistisse em produzir o impensável. O
impensável conservador.

Temer é homem do controle das máquinas burocráticas de partido, dos
almoços e dos tratos dos donos do poder entre si, os \emph{ricos entre
si}, como dizia Machado de Assis, mas de fato gente de uma riqueza nova
e realmente improdutiva, vida do poder que emergiu e se constituiu
inteiramente, exclusivamente, por dentro da esfera interior da política.
Da política compreendida por seu lado tendente ao orgânico, à
maquinaria, à mecânica, de algum modo algo vedado à democracia, ao
\emph{espaço da habitação coletiva e social} da política, que lhe é
oposto. Ele viveu representando mesmo a parte submersa do iceberg, a
face obscura da lua, os porões revestidos de veludo do poder, a arte
política incógnita cuja grande face evita com satisfação a expressão
pública. Homem de classe média conservadora paulistana, cujo horizonte é
apenas a ascensão reprodutiva de todos os preconceitos fundamentais e da
própria estrutura social pré"-existente do poder, do poder como ele é,
que se eleva socialmente pela política e que tem nela a sua vida mais
ordinária, o seu estilo e o seu ganha pão, \emph{que é muito mais ganho
do que pão} -- \emph{o Brasil, o seu} \emph{modo de comer, de dormir e
de ganhar dinheiro}, como dizia Mario de Andrade -- é o giro dos
negócios, das decisões de Estado, do apoio e da sustentação política e
da gestão de interesses privados no governo, que precisam de
institucionalidade porosa para dar destino à essa gente e classe, o
primitivo patrimonialismo sempre modernizado brasileiro, que este tipo
de sujeito político, avesso a qualquer sonho ou desejo de transformação,
incapaz de questionar o poder em qualquer nível, promove a cada segundo,
a cada gesto, a cada respiração, a cada prato que come, sono que dorme e
sonho que tem, dinheiro que se deposita em suas contas, enfim a cada
movimento que faz no mundo e na sua vida, privada pública, pública
privada.

Homens, sem espírito, do poder como ele é. A mediocridade garantida da
reprodução do Brasil como ele é, o mundo mítico teórico dos \emph{donos
do poder}, o homem sério que faz negócios no balcão que ocupa e vive de
fato, fundamentalmente, para ascender \emph{na firma}, entre o partido e
os amigos preferenciais nos negócios: procurador, secretário de governo,
deputado federal, vice presidente, presidente, praticamente sem nunca
ter vindo à praça pública expressar uma mísera ideia de comprometimento.
Sempre indicado pelo alto, sempre alavancado para cima e sempre
sustentando e articulando interesses dos que estão, como ele próprio,
dentro do jogo, que só se expande como mesmo. Sempre representando
praticamente nada na vida pública popular, na vida social, na imagem de
um sonho de política e vida política nacional para fora das suas Câmaras
e gabinetes.

Ele foi e é um lobista de partido, de negócios e da própria democracia,
no sentido de que o lobista \emph{vive} a política na interioridade
privada dos interesses, nos lobbies dos hotéis, nas antessalas dos
palácios, nas conversas a portas fechadas, nos clubes e nos
restaurantes, entre iguais e poderes diretos, expandindo o poder como o
bom negócio que ele é, para os de sempre, e como sempre. \emph{As
usual...} É a vida privada, quase interior, da economia que cruza o
espaço da política neste tipo de homem. Jamais este tipo de ator do
poder tem relação com algo da rua, do espaço \emph{público social}, e
ele é tão mais poderoso quanto mais ascende sem nunca ter precisado se
relacionar com alguma imagem política de algum povo que o comprometa, e
algum desejo popular que o contamine contra o jogo do poder.

Esse é o caso de Temer. Ele é o anti"-populista brasileiro por
excelência, o dono da estrutura, das regras do jogo, mas não da bola, o
gerente avalista da negociação e o arbitro dos equilíbrios e repasses do
próprio poder.

Ele representa a organicidade da política em si, responsável somente
pela conservação e expansão do sempre o mesmo. Seu poder vem, e ainda é,
da máquina orgânica e profundamente enraizada nos municípios do Brasil,
o partido único de oposição que emergiu da ditadura, e herdou o poder de
controle da política desde os municípios do Brasil, construído mesmo
quando era a única opção permitida de oposição política nos 22 anos de
ditadura civil militar no Brasil. Os vinte e dois anos de reserva de
mercado política para o \versal{MDB}, depois \versal{PMDB},
agora \versal{MDB} novamente. Porque no
Brasil quando um partido vai completamente à falência se troca o seu nome, mas não
os seus homens. \versal{MDB}, o partido que gira ao redor do próprio eixo, que
apenas se expande, amplia os negócios e controla fortemente a expansão
da democracia no Brasil. A força de Temer, um \emph{insider} orgânico,
que herdou o \versal{PMDB} paulista após a morte de Quércia e a degradação
política de Fleury, vem do enraizamento material da oligarquia política
peemedebista em todo o Brasil, de municípios ao Congresso, herdada da
própria administração torpe da política feita na ditadura militar.

Porém, seria um exagero não digno do personagem delegar a Temer a
primazia de tal mundo do poder sem imaginação, meramente reprodutivo de
qualquer coisa que já exista. Principalmente do poder que já existe. São
muitos os Michel Temer na política brasileira e eles existem há muito
tempo. O conservadorismo orgânico, e antissocial, é uma norma de fundo,
um baixo profundo que organiza toda a dança, da política e da sociedade
brasileira, praticamente um \emph{significante mestre} de nossa vida
política e social. O Brasil se mantém um pais muito difícil não por
acaso. É imensa a tradição de uma fidalguia medíocre do nada brasileira
-- a \emph{aristocracia do nada} de Paulo Emílio Sales Gomes -- que se
perde na noite negra de nossas \emph{raízes,} fundada essencialmente na
reprodução desejada da estrutura de poder e privilégios absolutos em
relação ao destino de imensa, e inominável \emph{como real} para este
tipo de senhor, exploração do povo brasileiro.

A origem histórica desse conservadorismo está em usar e negar mesmo a
simbolização, o registro problemático e o desejo necessário, de mudança,
diante do corpo inteiramente expropriado do escravo negro, origem da
ideia negociada futura de povo brasileiro. Nossa mentalidade fidalga de
orientação direta para o poder colonial, escravista extrativista,
expressa no conservadorismo saquarema do Império no século \versal{XIX}, depois
no conservadorismo positivista, que fundou a abstração da \emph{ordem}
como progresso, depois em nosso conservadorismo autoritário de
fazendeiros café com leite da República Velha paulista, depois em nossa
direita liberal e dependente da modernização industrial dos anos 1950, e
no choque de intervenção ideológica da direita grosseira da guerra fria
mundial dos anos 1960 e, por fim, mas não por último, o homem médio que
nada quer senão privilégios de classe no Brasil de pouca transformação
social, gerido subjetivamente em profundidade superficial pela indústria
cultural pós moderna, são todos eles, em vários momentos sociais da
história brasileira, circuitos humanos e socais reprodutivos,
conservadores, que convivem sempre bem com violência, a exploração
extremada e a escravidão nacionais. A falta de imaginação política,
humana e social é de fato endêmica há muito em São Paulo, por exemplo, e
é responsável por políticas e distorções do espaço democrático tão
serias quanto pelos nomes muito rebaixados que sustenta no mundo: Jânio,
Maluf, Alckmin, Kassab, Temer... Trata"-se de uma profunda e
condescendente tradição brasileira da relação entre mediocridade e
poder.

Mas Temer é ainda mais radical, nesta ordem do império do mesmo, que
seus pares de menos sucesso. Ele chegou a presidência de fato tendo toda
uma carreira estruturada inteiramente por dentro do sistema de controle
político dos partidos, e ``do partido'' que controlou para os próprios
interesses o processo da democratização pós-1984. Ele é, mais do que
todos os outros, um homem cujo rosto mal conhece a luz da dinâmica do
espaço público social, uma espécie de burocrata político que é uma
garantia, que se alimenta do poder como ele é, alimentando"-o também, em
um único gesto. Um burocrata político, sem burocracia, apenas política,
conservador e \emph{garantista}, o homem certo para sustentar o negócio,
controlar as listas de interesses, receber as demandas, encaminhar os
poderes, fazer a coisa em si da política andar. Seu governo foi apenas
uma encomenda.

Trata"-se de uma potência real da política, cujo sentido é articular e
dar vida aos desejos de poder, um articulador decoroso do vínculo
capital política, controlador cordial de grandes bancadas de interesse.
Um \emph{exu}, poderíamos dizer, força que faz atravessar todas as
forças do desejo de poder em seu próprio corpo, da política, para
comunicar e entregar o que se acertou. Mas, sem a amoralidade erótica
afirmativa popular dos exus, de algum modo mais morto do que vivo, mais
mecanicamente reprodutor do que espiritualmente implicado. O
\emph{mordomo} do poder, que servilmente entrega o combinado, sempre
tirando a própria parte. Vindo do fundo da estrutura do poder orgânico e
organizado do capitalismo e da política brasileira, ele mal deixa
entrever sua vida imaginativa, sua proposta de país, seu desejo de
civilização, quando aparece no espaço público compartilhado. Sua voz
melíflua e suas mãos que giram sobre si mesmas, representando longos
cálculos e negociações de velhos espertos, de fato não falam nada. Uma
flor nascida no pântano da riqueza brasileira, nos porões dos negócios e
interesses, que mal sabe ver e suportar a luz do dia do espaço
turbulento dos interesses populares e sociais. Um \emph{vampiro}, vai
dizer sobre ele a imaginação popular, tentando dar figurabilidade para a
complexidade ctônica dessa flor dos porões da política e do poder, da
estufa sem o sol da esperança social, que, não se sabe bem como isso se
tornou possível, chegou ao poder nacional ao seu próprio modo.

O governo Temer foi exatamente o que se podia esperar dele, duro contra
os interesses do trabalho no Brasil, cioso do poder concentrado nas
intervenções que fez no espaço político social brasileiro -- nos ataques
aos institutos de representação de minorias e direitos humanos, na
mudança simplificadora das diretrizes da educação, feitas da noite para
o dia sem consulta nem satisfação a nenhuma instância implicada, como se
democracia fosse isso, na apoteose degradante da militarização da crise
social que a crise política radical que o levou ao poder produziu no
Brasil em 2017 e 2018 -- bem autoritário na política quando expressa nas
ruas contra a sua legitimidade discutível, e muito comprometido com a
blindagem da espetacular máquina de corrupção da política, utilizada
estrategicamente para levá"-lo ao poder, mas que não poderia ser de
nenhum modo cobrada, ou imputada, o seu esquema e partido universal da
política brasileira. Mas, é preciso admitir, todas estas ações fortes de
uso e de gosto do poder sempre foram feitas com a voz tênue e contida de
quem comunica verdades óbvias e evidentes, truísmos assustadoramente
medíocres, já decido há muito nos arcanos conservadores do poder
brasileiro e fugindo deste modo do conflito que de fato instaurava.

Seu governo, apesar da figura, conseguiu dar um ponto de gestão
neoliberal da economia que espantou o mundo econômico existente -- e que
fala fortemente da ruína política em que a esquerda nacional
institucional se metera, a sua impotência total -- invenção que recebeu
explícitas ressalvas de economistas de todo o mundo, de ganhadores de
Prêmios Nobel, como Krugman e Stiglitz até o setor da \versal{ONU} de avaliação das
políticas econômicas de países e de regiões: ele congelou a estrutura do
orçamento dos gastos sociais do país, por vinte anos... O Brasil não pode
mais encaminhar dinheiro público para a Educação, por exemplo, sem dizer
precisamente qual outra rubrica do orçamento será reduzida. Esta grande
contribuição ao controle dos mercados financeiros sobre o país que agora se inscrevia em
lei, inexistente em qualquer outro país do mundo, foi uma das
suas primeiras ações, só antecedida em velocidade e
desembaraço pela mudança, quase imediata após a sua tomada do poder, da
legislação de exploração dos campos de petróleo do pré"-sal brasileiro,
atendendo interesses das grandes empresas mundiais do setor...

Já para o capital e os interesses da exploração cotidiana local, sua
principal obra foi a destruição das leis trabalhistas brasileiras, de
fato o grande interesse econômico do movimento social empresarial que o
levou ao poder. No final de 2017 -- Temer assumiu em maio de 2016 -- as
leis reguladoras dos direitos do trabalho vigentes desde 1943 se
transformaram drasticamente. Complementando a vitória da lógica da
terceirização total, conquistada em uma ação de velocidade extrema da
direita organizada pelas eleições no início de 2015 -- e pelo peso
pesado, capo de máquina política fisiológica, Eduardo Cunha, parceiro de
Temer, até ser preso por revelação de suas contas milionárias na Suiça,
como resultado de sua política temerária de abertura do processo de
impeachment contra Dilma Rousseff -- a nova legislação do trabalho no
Brasil determinava que os \emph{acordos} entre as empresas e os
trabalhadores podiam a partir daí \emph{se sobrepor e ter plena
vigência} sobre a Consolidação das Leis Trabalhistas...

Instituiu"-se a nova modalidade do trabalho intermitente, contratado e
pago por fragmentos da jornada de trabalho, apenas por horas
trabalhadas... Esvaziou"-se a força dos sindicatos e passou"-se a fazer
exigências financeiras fortes para os trabalhadores acessarem a justiça
do trabalho... Tornou"-se possível a extensão de um contrato temporário
de trabalho por 9 meses... -- e neste ponto podemos observar o
funcionamento da novilíngua do poder brasileiro, típica da época, que
diz e desdiz o que diz, mas diz e, principalmente, faz o que quer..,
assim a legislação sobre o trabalho temporário estabelece que ``ele não
pode exceder três meses'', mas, \emph{há previsão para prorrogação de
até 180 dias}, e, ainda \emph{por mais 90 dias, se comprovada a
manutenção das condições que levaram à contratação temporária...}
resultado: uma condição de trabalho que não pode exceder três meses pode
durar 270 dias... Do mesmo modo que a \versal{CLT} continua vigente mesmo quando
o empresariado pode impor qualquer acordo que a negue... É o mundo
político escorregadio das ações de força, mas sob o signo de uma
ilegitimidade mais ampla e geral, dos poderes que chegaram ao governo
após a derrubada da presidente Dilma Rousseff.

O governo Temer também foi um poderoso espaço instituído de facilitação
de interesses diretos de grandes setores, empresas e forças econômicas
junto ao Estado brasileiro. Os negócios, típicos dos bastidores dos
poderes, da venda e compra de interesses, próprios da política
patrimonialista conservadora brasileira que se tornou una com as grandes
corporações econômicas no sistema geral da política corrupta e alienada,
chegavam, com aquele homem no poder, à ação muito visível da própria
Presidência da República. O presidente cedia, sem grandes explicações,
de tempos em tempos, grandes nacos da coisa pública a algum grupo de
poder econômico, interno ou externo ao país. A principal destas
concessões foi, como já disse, a imediata alteração das regras de
participação da Petrobras na exploração dos campos de petróleo do
pré"-sal e sua abertura para a posse de grupos estrangeiros, realizada
muito rapidamente, em tramitação quase de urgência, embora não se saiba
urgência de quem, logo após a tomada do poder, ainda em 2016, pelo novo
grupo.

Era o jogo mesquinho de ataque direto ao Estado e ao público, uma
espécie avançada de rodada de capitalismo de nova acumulação primitiva,
com venda direta e privada de controles e espaços públicos no Brasil,
que a chegada ao poder de um grupo de força privada por fora da política
do voto, com a derrubada discutível da presidente eleita, permitia em
grande escala aos vencedores dessa política. Neste mundo explícito da
tomada econômica direta do pais, interesses de caráter coletivo,
social ou comum definitivamente não faziam parte de nenhuma pauta
governamental.

A desfaçatez irresponsável com interesses sociais e da vida da pobreza
no Brasil, sacrificados diretamente ao grande lucro, chegou ao auge em
dois casos muito escandalosos e grandiosos, cuja repercussão negativa
foi simplesmente mundial -- como já havia sido o congelamento do
orçamento nacional por 20 anos -- fazendo o governo de controle direto
de interesses econômicos do Brasil ter que recuar -- mais uma vez
\emph{pela ação} \emph{da lei inglesa nos mares internacionais do
Brasil...} --, ter que simplesmente voltar atrás, o que ele fez com a
pura máscara de cara de pau, como lhe era típico, desfazendo de um dia
para o outro as ações tidas por decididas, como se nada nunca tivesse
acontecido. Era um governo que checava as forças políticas de ganho e
interesse diretamente frente aos atos decididos de apropriação, todos
orientados para o lucro de alguém ou algum setor -- como o seu ministro
da Secretaria de Governo, e articulador político, Geddel Vieira Lima,
tentando liberar um prédio de alto luxo, no qual era proprietário, em
área tombada e preservada de patrimônio nacional pelo \versal{IPHAN} em Salvador,
na Bahia... pouco mais tarde, o mesmo homem de confiança do presidente
foi preso por manter 51 milhões de reais em notas de dinheiro, ``sem
origem'', em um apartamento vazio...

Essas ações predatórias de grandes interesses econômicos, que encontravam
muita facilidade no governo de direita do Brasil pós-impeachment,
encontraram um limite em um processo de opinião pública mundial e de
interesses humanos mais amplos do que a perspectiva míope da máxima
vantagem em mínimo espaço de tempo que governava o país permitia
enxergar. Uma dessas decisões desastradas do sistema da cobiça sem
controle social do governo pós"-impeachment, que durou uma semana, e foi
derrubada por um forte movimento mundial movido por artistas pops, de
recriminação intensa e denuncia generalizada do governo brasileiro, foi
a liberação da Reserva Nacional Ambiental de Cobre e seus Associados, na
Amazônia, área que contemplava outras nove reservas naturais e indígenas
-- onde habita o povo Wajpi, por exemplo -- para a plena exploração
comercial de minérios, ao que tudo indica, capitaneada e desejada por
uma empresa gigante do setor, canadense...

A segunda ação, ainda mais lesiva a interesses coletivos, sociais e dos
pobres, foi a tentativa do Ministério do Trabalho, do ministro político
Ronaldo Nogueira, de descaracterizar a ideia e de dificultar fortemente
a fiscalização de \emph{trabalho análogo à escravidão} nas fazendas do
agronegócio do Brasil. Com uma portaria simples do Ministério do
Trabalho se destruía vinte anos de política de combate ao trabalho
escravo no país, de consonância com as determinações internacionais a
respeito. Desta vez foi o mundo político, da \versal{ONU} à União Europeia, que
estranhou a degradação social e política promovida no Brasil pelo
governo que emergiu vitorioso das cinzas do impeachment de Dilma
Rousseff. E o episódio é também muito expressivo e claro sobre o
prestígio dos Direitos Humanos universais para o governo de Michel
Temer.

Não foi de nenhum modo por acaso, portanto, que, após alcançar um novo e
gritante recorde de desemprego, depois dos anos de pleno emprego dos
governos petistas -- recorde fixado agora, com o aprofundamento da crise
econômica associada ao aprofundamento da crise política, em cerca 12,5 milhões
de pessoas, em 12\%, mais os 4 milhões que desistiram de procurar
trabalho segundo o \versal{IBGE}, taxa antissocial que, segundo a avalição de
muitos, permanecerá neste patamar ao menos até 2019 quando o governo,
dito ``golpista'' por muitos, deve finalmente deixar o poder -- que,
tudo sendo do modo que foi, uma farra de interesses antissociais no
poder, no carnaval de 2018 a expressão popular do desfile da escola de
samba do Rio de Janeiro Paraíso do Tuiuti tenha tido um imenso e justo
impacto sobre a consciência de todo país, cuja opinião geral era até
então artificialmente contida pelo controle da avaliação do governo
pelos próprios interesses, meios e homens que o produziram. Um desfile
de carnaval histórico, sobre o degradado Brasil pós impeachment, cujo
ápice foi o carro alegórico final, em que todas as formas de exploração
e violência econômica sobre o povo brasileiro foram representadas, e que
se completava com a alegoria da figura no topo, caracterizando
explicitamente o presidente do Brasil, que portava a faixa presidencial,
o terno e o cabelo que se confundia com chifres..., alegoria nomeada
pela imaginação carnavalesca como ``Vampiro do povo brasileiro''. Estava
celebrada, no próprio período, a avaliação popular do sentido do golpe
que Capital, política e espírito conservador produziram no país em 2015
e 2016. Um governo muito difícil para a vida popular, e em grande parte
fracassado em recuperar o país da própria violência que o levou ao
poder.

Por tudo isso, é evidente que a chegada de um tipo político como Michel
Temer ao governo, em um país com as contradições e desafios do Brasil,
não pode se explicar de nenhum modo pela força caraterística produtora
de poder \emph{do próprio homem} e personagem histórico. De fato, Temer
no máximo significava uma garantia de continuidade e de luta ranhida
pela impunidade de seu amplo grupo político de apoio e homens próximos.
Assim mesmo muitos foram presos. E, mais importante, uma garantia de
grande liberdade e conivência para as ações de poderes econômicos que
buscavam redesenhar o espaço social do Brasil através de seu governo. Um
\emph{garantista} da reprodução do poder, acostumado há muito tempo,
como o seu partido, com o grande negócio do Brasil. Mas, como homem que
nada diz sobre nada que importe sobre a vida, a história e o destino do
país, ele não chegaria ao poder por seus méritos e força.

Ele é apenas o resultado, desencantado e medíocre, do pacto social e dos
amplos grupos de interesse que se articularam ao redor de seu governo de
transição capitalista, após a liquidação política e judicial da esquerda
no Brasil. Ele é a face sem força civilizatória de um movimento \emph{de
todo poder ao poder}, que de fato não sabe enunciar nenhuma civilização
que não seja continuar os termos da acumulação, fixada a priori como a
norma de tudo que há a fazer na vida e no Brasil. Temer era o fantoche
conservador e decoroso, mediador, do novo estágio mais avançado e
envenenado do neoliberalismo no Brasil. Para levá"-lo, por acaso, ao
poder, o movimento social de múltiplos grupos à direita conseguiu uma
união inédita na democracia brasileira, que não foi feita em seu nome,
mas estritamente para a derrubada e descaracterização da história
recente dos bem sucedidos governos petistas.

Mercado financeiro, indústria nacional que rompia os tratos com o
governo desenvolvimentista petista, grande mídia pautada por grandes
interesses econômicos e seus próprios donos, judicialização antipetista
com práticas jurídicas exclusivas e um imenso movimento, multifacetado,
de ação de uma nova simbólica de direita, tendente à violência, nas ruas
do Brasil, alinhados em um projeto de retomada de alguma cultura
conservadora, como real cortina para a retomada dos interesses diretos
dos capitais brasileiros do governo, sua nova possessão histórica, foram
os verdadeiros agentes da chegada ao poder de um homem tão anódino,
desinteressante e inexpressivo, a respeito do país e da vida, quanto era
o próprio movimento social que o investiu.

O resto final de sua política de facilitação de tudo que é poder --
afora a continuidade espetacular das denúncias de corrupção sobre seu
partido e ele próprio, que o levou a comprar o congresso por duas vezes
no período para impedir o avanço legal da devassa sobre seus negócios, e
de seu secretário político preso, com grandes corporações como o grupo
\versal{JBS} do agronegócio brasileiro, ou seu tradicional polo de financiamento
em empresas atuantes no porto de Santos... -- foi a real cultura
agressiva, antissocial e antidemocrática, o uso instrumentalizado do
ódio na política, do qual o grupo que o levou ao poder se utilizou em
excesso durante todo o processo da guinada do governo brasileiro à
direita nos anos de 2016, 2017 e 2018.

Temer é o produto da ação de grupos de poder, coletivos de interesse,
ele é apenas um membro de sua própria \emph{horda}, e dentre aqueles
grupos está o dos novos autoritários brasileiros, que, produzindo real
política paranoica anti"-esquerda nas ruas, mantiveram constante a
degradação do espaço público brasileiro durante todo período de seu
governo.

Grupos agressivos, produtores satisfeitos de mentira pública, capazes de
projetar violência política sobre tudo o que não seja a plena liberdade
para o capital brasileiro tomar suas plenas decisões sobre os destinos
do país no poder, agora sem crítica nem controle. Este caldo de cultura
grupal, movido a uma epidemia de mentiras públicas, \emph{fake news} e
ação de constrangimento constante nas ruas, foi mesmo a real face
pública do governo de Michel Temer, sempre incapaz de, por ele mesmo,
balbuciar qualquer coisa rica sobre a vida brasileira.

Esta cultura política regredida, estúpida e violenta, das novas
direitas, produtoras de imensas máquinas de mentira na internet e ativas
na busca de cerceamento de direitos na vida brasileira, estabeleceu o
fundo da consciência degradada para as duas maiores catástrofes
políticas do período do neo"-golpe da democracia brasileira: a
inacreditável ascensão política do fascista tupiniquim, muito ignorante
e violento, Jair Bolsonaro, com seus apoiadores, verdadeiros fanáticos
da burrice e da estupidez, e a trágica e irresponsável intervenção
federal militar no Estado do Rio de Janeiro.

Temer, o vampiro, também quer dizer esta tendência de rebaixamento da
política e do poder à violência, que tomou o pais durante seu breve,
instável e malfadado reinado.

\chapter{\emph{Ordem} e violência no Brasil\footnote{Publicado
  originalmente em \emph{Bala perdida, a violência policial no Brasil e
  os desafios para sua superação}, São Paulo: Boitempo, Carta Maior,
  2015.}}

Existe no Brasil uma ideia muito própria de \emph{ordem.} Esta
noçã\emph{o, vaga mas ativa, indefinida mas muito afirmativa}, é usada
em momentos estratégicos por homens de Estado, está presente no
horizonte do discurso conservador nacional e na sustentação das ações
policiais mais duras, em geral de impacto social muito violento. Na
estranha \emph{ordem e progresso} brasileira, o peso autoritário e
fantasmagórico da noção de \emph{ordem} vem sempre primeiro. Ela
antecede o progresso -- seja lá o que se conceba por tal, democracia ou
integração social -- e parece pairar solitária, mas sem conceito, antes
e acima de algum processo histórico e real concebível.

E como parte deste sortilégio, nunca se deve perguntar de onde, ou de
quem, emana a ordem do impensável chamado à \emph{ordem} brasileiro. Não
se deve perguntar que ordem, ordem de que, para que ou para quem. O polo
abstrato de onde emana a ordem está sempre distante de qualquer vida
social concreta -- a metrópole, o rei, o capital multinacional?
Dependendo do ponto de vista que chequemos esta noção, que pensemos sua
constelação imaginária e simbólica, tão radical, ela pode até mesmo se
colocar claramente em oposição à ideia moderna de lei -- entendida como
alguma norma racional pactada política e socialmente por uma sociedade e
uma nação, e também, em nível mais amplo, entre as nações --. É possível
e provável, e os exemplos são inúmeros, que no Brasil tenha se
constituído um verdadeiro campo político, e psíquico, de uma ação pela
\emph{ordem} que não corresponda aos direitos universais, relativos à
história do processo normativo e político ocidental, os mesmos que, para
estar inserido, o país também professa; e, até mesmo, em um grau ainda
mais fantástico, é possível que tal chamado à ordem não corresponda ao
próprio campo e estrutura das próprias leis, mais ou menos racionais,
mais ou menos sociais, vigentes no próprio país. O imperativo de nossa
ordem, dado de força direto sobre toda matéria social e histórica sempre
frágil, não tem nenhuma correspondência com o famosos imperativo
categórico de Kant, onde é a razão autossustentável e universalmente
reconhecível, entre nós apenas mais uma das miríades de ideias que não
correspondem à coisa, que de fato se expressa.

O principal agente social \emph{desta ordem acima da lei}, que recebe
dela o seu mandato não escrito, e goza do privilégio de ser sujeito
deste desejo social fantasmático, é a polícia, ou melhor dizendo,
\emph{as várias policias}, reais ou imaginárias, existentes no país. E,
ainda, durante os períodos de exceção ditatoriais brasileiros o aspecto
policialesco para dentro foi também realizado pelas próprias forças
armadas -- e aqui seria mais preciso dizer os períodos de \emph{mais
exceção} brasileiros\emph{...} da origem militar positivista da
República, passando pelo pacto senhoril antidemocrático da primeira
República paulista mineira, pela ditadura de 1937/1945 e pela grande
ditadura de 1964/1984, se não incluirmos aí a aberta política de
ilegalidades consentidas a respeito da escravidão, do Império -- .
Nestes momentos o povo e as dinâmicas sociais nacionais, sem leitura em
um quadro simbólico de legitimidade moderna do conflito de classes, são
\emph{o verdadeiro inimigo}, contra quem deve se levantar a \emph{ordem}
mais radical, legitimadora de toda exceção, a começar pelo direito à
tortura e ao assassinato, e o exército e as forças armadas apenas se
confundem com a longa tradição da polícia discricionária brasileira, das
milícias dos ``capitães de assalto'' da Colônia -- origem dos capitães
do mato negreiros ``de tão tenebrosa memória'', nas palavras de Caio
Prado Jr. --, até os \emph{soldados amarelos} e os delegados
torturadores varguistas, à escuderia LeCocq dos agentes matadores da
polícia ditatorial dos anos 1970, chegando à atual polícia moderna,
equipada e assassina, que se espalha pelo país, tolerada com toda
complacência e cumplicidade.

Não há dúvida que uma ação política tão especial, e seu desenho profundo
dos sujeitos da experiência pública, como é a noção de ordem por tantas
vezes acionada no campo conservador brasileiro, deve ter uma longa e
profunda história, e também uma própria genealogia. O quadro original de
mentalidades e o estatuto de capitalismo mercantil, colônia de plena
exploração sob o foco de uma monarquia absoluta católica e
transoceânica, com estrutura econômica e social escravista e de
latifúndio, que perdurou no Brasil por longos 300 anos, e mais o século
da variação Imperial nacional dos mesmos temas -- nos mesmos 400 anos da
emergência da revolução do capitalismo liberal industrial na Europa, e
no século \versal{XIX}, nos Estados Unidos -- é importante para a situação desta
natureza de ordem autoritária, anti"-humanista, e sua estranha relação
com a própria lei.

Sérgio Buarque de Holanda recorda, por exemplo, que embora houvesse
alguma regulação de direitos a respeito da vida e da morte de escravos,
e mesmo de agregados, no rarefeito espaço jurídico e público colonial,
de fato, e de direito, as decisões a respeito destes objetos se davam
amplamente no âmbito mais estrito da porta da fazenda para dentro, e o
legislador e executor destas penas, que da mutilação levavam muitas
vezes à morte, era o senhor -- pelo braço servil de seus capatazes,
muitas vezes negros -- sujeito real de um poder absoluto sobre as suas
posses, poder de reais contornos sadianos. Nesta dimensão muito ampla
das coisas brasileiras, concentração absoluta de poder pessoal dos
senhores -- em correspondência à metrópole e ao seu rei católico --
anti"-humanismo escravocrata, capricho particular e sadismo pessoal e
subjetivo, faziam as vezes da lei pública que não devia atravessar a
esfera primeira do domínio pessoal sobre a própria propriedade. Este
poder arbitrário extremo, dominação não inscrita em um código, de vida e
de morte, sobre o corpo negro e trabalhador, mantinha antigos traços do
tipo de domínio sobre o homem fixado à terra europeu, de estrutura
feudal, mas já estava alavancado em um horizonte de um liberalismo
radical, de valor prioritário da propriedade, e do proprietário, em que,
em escala titânica, se explorava e se produzia para o mercado mundial do
tempo.

Estes homens eram ainda, em mentalidade, senhores do tipo antigo europeu
-- para quem toda mobilidade social de massas era apenas rigidamente
inconcebível -- e eram também modernos, na medida da busca da
produtividade da exploração sobre o corpo em parte descartável de seu
escravo, produtividade orientada para o mercado mundial no qual estavam
inseridos. Verdadeiros \emph{donos} da \emph{mercadoria bem de produção}
do \emph{corpo do escravo}, na esfera da posse patriarcal da grande
terra, tais senhores eram os enunciadores da \emph{ordem} que emanava de
seu próprio corpo sobre o outro e sobre tudo mais em suas possessões,
desconhecendo os pactos frágeis da lei externa colonial, lei que também
desconhecia, por princípio de ordem, mesmo que buscasse alguma
regulação, a natureza das violências decididas nas fazendas distantes do
Brasil.

Uma obra moderna sobre esta mesma \emph{ordem} subjetiva, como \emph{São
Bernardo} de Graciliano Ramos, de 1934, nos dá ainda na primeira metade
do século \versal{XX} um estudo de tal mentalidade, que perdurava, e de tal uso
do espaço da propriedade para a real ordem do dono, avesso à lei
distante, mais própria de uma burguesia emergente que também não se
diferenciava suficientemente, nem psiquicamente, nem produtivamente, de
tal legítimo \emph{dono do poder} fundado na escravidão e no racismo
brasileiros.

E, no ano de 2014, de fato, foram assassinados no Brasil pelo menos 34
assentados, trabalhadores sem terra, sindicalistas rurais, lideranças e
membros de comunidades rurais, no Mato Grosso, no Pará, em Tocantins, no
Maranhão..., em crimes que não costumam ser investigados nem resolvidos
por polícia alguma, nem justiça. Em 2013, foram 36 mortes violentas no
campo, em 2012, 36, em 2011, 29, em 2010, 34... de modo que 1720 pessoas
foram mortas em 25 anos, com a vexatória média, bem brasileira, de 1
condenado pela justiça para cada 17 mortos cidadãos brasileiros do
campo.

É certo que, noutra direção, importa muito para o sentido da
\emph{ordem} brasileira o fato histórico da não observação por parte da
elite Imperial do século \versal{XIX} da lei que proibiu o tráfico de escravos,
que definiu as posições públicas dos senhores brasileiros \emph{a favor
da transgressão sistemática de suas próprias leis}, em postura e prática
perversas que foram centrais ao espaço jurídico do Império brasileiro.
Após o estabelecimento da lei de 7 de novembro de 1831, 750 mil escravos
entraram no pais, \emph{mercadoria ilegal tolerada por um Estado de
exceção generalizada}, para mover a produção e a riqueza da nação, até a
final abolição formal da escravidão brasileira em 1888. Deste modo,
nossa elite imperial cindiu a sua \emph{ordem} econômica e simbólica da
própria lei estabelecida por ela mesma, transformando o espaço público
nacional em uma espécie original de \emph{ordem pirata}, criando no
processo um estatuto único de \emph{irrealidade da lei}.

Não por acaso, na época, a partir de 1845, quem fazia observar a
\emph{lei} local eram as canhoneiras inglesas, afundando navios
negreiros brasileiros na costa do Brasil, assumindo o papel de polícia
internacional dos próprios interesses, diante de um país escravista
marginal, o Brasil. A polícia mundial da marinha inglesa punha ordem na
desordem escravista local, que inventava sua nova \emph{ordem} de
irrealidade da lei no país moderno/arcaico americano. E foi exatamente a
mesma estrutura de \emph{lei exterior} ao país, que se desrealiza
internamente, que obrigou o governo brasileiro a produzir a sua frágil
reparação e julgamento dos agentes de Estado torturadores e assassinos
da ditadura de 1964, no apagar das luzes do governo Lula: o Brasil foi
condenado no Tribunal Interamericano de Direitos Humanos, e obrigado, de
novo desde fora, a finalmente cumprir as leis que professava.

O estado de exceção brasileiro do século \versal{XIX} era uma ordem que isolava
mesmo \emph{a totalidade da lei geral}, e foi esta exata ordem que
completou a \emph{forma} do nosso famoso sujeito volúvel, a elite
brasileira, extremamente violenta e fundamentalmente cínica, ainda que
moderna, bem descrita formalmente por Machado de Assis, a partir de
1880, e estudada por Roberto Schwarz, a partir de 1964. Para alguns
historiadores, como Luiz Felipe de Alencastro e Sidney Chalhoub, esta
\emph{grande recusa} em aceitar a própria lei, no caso da proibição não
reconhecida do tráfico negreiro, é de fato a organização histórica que
deu origem ao cinismo e à dissolução da relação subjetiva com a lei, o
``pecado original'', sempre antissocial, tão próprio da elite dirigente
brasileira e sua ordem, interna e externa.

Também Antonio Candido observou que era de fato apenas a polícia, o
major Vidigal -- um Capitão Nascimento do tempo? -- que balizava e
tencionava o movimento entre o espaço da ordem e o da desordem na
representação social do Brasil dada na forma de \emph{Memórias de um
Sargento de Milícia} (1852), que dava uma imagem dialética da vida
brasileira entre a década de 1850 e as suas origens, desde a chegada da
corte em 1808, no Rio de Janeiro. Bem ou mal, no caso do livro, os
emissários da ordenação social, da integração pública no espaço social
regulado pelo Estado, não eram, de nenhum modo, funcionários públicos ou
burocracia, instituições, dinâmicas sociais, políticas de governo, mas,
simplesmente, de modo radical, apenas o prosaico agente policial da
cidade, que perseguia capoeiras, feiticeiros e festas, negros. Apenas o
major parecia ser o único representante do espaço do Estado sobre o
mundo da vida nas origens nacionais.

E seria assim que se manteria o lugar da representação policial na
sociedade tensionada, mas sempre atrasada no ganho social, até
\emph{Deus e o Diabo na Terra do Sol}, em 1964, em que o mercenário e
assassino Antônio das Mortes realizava exatamente a mesma função social
do major Vidigal, em um quadro de exigências sociais novas desconhecidas
do país. E também, em uma imagem ainda mais clara, porque documental,
com o exército transformado em \emph{milícias de} \emph{capitães de
assalto}, verdadeiros capitães do mato, para perseguir e prender
camponeses que demandavam direitos, no exemplar e perfeito \emph{Cabra
Marcado para Morrer}, de 1984, filme que, nas origens da
redemocratização\emph{, falava das próprias origens da democracia
enraizadas na ditadura de 1964}.

De fato, do ponto de vista desta longa experiência política, estética e
formal, \emph{polícia no Brasil parece sempre ter sido}
``\emph{departamento de ordem política e social'',} antes mesmo da
existência de qualquer estrutura de ação social e de desenvolvimento de
um Estado burocrático moderno, interessado ou não na integração e no
resgate do déficit social brasileiro. É aí mesmo que se situa o pacto
policial da \emph{ordem} fixada \emph{sem desenvolvimento social}, marco
primeiro da civilização local, própria da elite autoritária brasileira,
que informa a dimensão não regulada por nenhuma lei de nossa polícia.

Caio Prado Jr. nos lembra ainda que, durante a Colônia, o mínimo e
primeiro agente público que intervinha nas necessidades citadinas e
cotidianas do mundo da vida era um verdadeiro derivado improvisado dos
corpos militares das \emph{ordenanças} portuguesas -- a terceira força
militar colonial, após as tropas de linha e as milícias. Antes de ser um
agente público, ou um funcionário, o colono português, que sustentava o
seu próprio interesse em qualquer ato público que realizasse, era
originalmente e principalmente um \emph{capitão"-mor}, ou um
\emph{sargento"-mor} de um \emph{corpo de ordenança}. Ou seja, o
proto"-funcionário público brasileiro foi, antes de tudo, um militar, ou
um policial... Um policial da ordem colonial escravista.

Esta fantasmagoria imensa de nosso passado policialesco autoritário e
particularista não é massa morta de referências esquecidas. Ela é
matéria viva, campo dialético negativo algo presente, não ultrapassado,
mesmo que modernizado. Sem o entendimento desta história é difícil
explicar como o avanço de nossa democracia fantasmagórica, e parasita,
que ainda mantém o seu polo antissocial da \emph{ordem} apoiado sobre as
polícias, chegou aos 56.337 assassinados no Brasil em 2014. Ou como a
policia de São Paulo, dos muito elegantes, decorosos e sérios
peessedebistas entronizados no poder, matou 801 pessoas naquele mesmo
ano, ou como de janeiro de 2015 até o momento em que escrevo, em março,
a mesma polícia paulista já matou 180 cidadãos brasileiros.

Enfim, sem tal matéria histórica ainda viva fica difícil compreendermos
como o Brasil chegou, nas contas da \versal{ONU}, a produzir 11\% dos
assassinatos do mundo, em sua mais plena normalidade social,
institucional e psíquica.

\chapter{Tradição da mentira tradição do ódio}

%Tales Ab'Sáber

\begin{flushright}
\begin{minipage}{0.84\textwidth}
\footnotesize
\emph{``Comunista bom é comunista morto. Dilma, Maduro, Hugo, Fidel, Lula,
lixo do mundo.''}

\smallskip
\emph{``Army, Navy and Air Force. Please save us once again of communism.''}

\smallskip
\emph{``Dilma Rousseff devia ter sido enforcada na Oban.''}

\smallskip
\emph{``O \versal{PT} segue o comunismo. O \versal{PT} adora o diabo. O \versal{PT} rouba, mente e quer
fechar todas as igrejas porque não acredita em Deus.''}

\medskip
\hspace*{\fill}--- Cartazes de manifestação contra o governo
\end{minipage}
\end{flushright}

Um momento extraordinário de revelação do Brasil, de um modo muito
íntimo do país produzir a si próprio, nos foi dado pela conjunção de um
documento concreto com um importante filme nacional. Como não poderia
deixar de ser, sobre momentos significativos de aprendizado de nós
mesmos, é uma das obras fortes que a cultura crítica e exigente do
Brasil remete de tempos em tempos à outra cultura brasileira -- a
prática, imediata, anti"-intelectual, quando não basicamente autoritária
-- que deve nos servir de baliza, de referência para que não nos
percamos no sentido do que é o Brasil.

Em uma cena mínima do imenso \emph{Cabra marcado para morrer --} como se
sabe, um filme feito em dois tempos, entre 1964 e 1984, por Eduardo
Coutinho e seus companheiros de viagem -- temos recuperado pelo rico
espaço de razões do filme \emph{uma pequena notícia de jornal} publicada
pelo Diário de Pernambuco à respeito dos acontecimentos e da intervenção
do exército nas filmagens realizadas pelos estudantes cariocas em
conjunto com os camponeses locais na região do Engenho da Galiléia, no
início de 1964. O projeto original do filme era o de uma rara mistura
entre ficção e documentário, concebido por uma vanguarda estética da
época realmente engajada, em que os atores que viveram as violências das
relações de trabalho no latifúndio nordestino, de uma história que levou
ao assassinato de uma liderança camponesa, \emph{encenariam suas
próprias vidas} como personagens em um filme que contaria a sua própria
história. Tratava"-se de mais uma das experiências de modernidade
cinematográfica e compromisso social frente o problema do trabalho, da
miséria e do atraso no nordeste brasileiro, que possivelmente viria a
ser outra obra prima de nosso cinema moderno nacional, o cinema novo do
Brasil dos anos de 1960.

Enfim, uma ficção baseada em fatos da violência do latifúndio sobre os
camponeses, filmada nos locais originais onde as ações se passaram, com
os próprios atores sociais representando a si mesmos como personagens.
Tudo orientado como ação política pela luta camponesa por direitos no
campo, principalmente pela conquista de direitos trabalhistas. O projeto
expressava com muita força o pacto, característico do tempo, dos jovens
universitários de esquerda organizados para suas ações estéticas e
políticas com a classe trabalhadora e a vida popular brasileira, no
caso, do nordeste agrário, que então podia contar com a solidariedade e
com a razão técnica daquela fração nova da elite, que pensava o
desenvolvimento dos direitos da vida popular como verdadeiro horizonte
de desenvolvimento geral do país.

O famoso filme dos anos 1980 recuperava a historia e a vida dispersa de
seus atores pelo golpe e pela ditadura civil/militar daquele ano de
1964, e contava também a história do próprio empastelamento da filmagem
original pelos militares pernambucanos, com as correlatas perseguições,
prisões e fugas da equipe e com a posterior recuperação, quase
milagrosa, de parte do material filmado que havia sido apreendida. A
intervenção na filmagem se deu poucos dias depois do infame dia 31 de
março de 1964, que deu início a vinte e um anos de ditadura no Brasil.
Acompanhamos toda a relação de Eduardo Coutinho e dos jovens da \versal{UNE} que
se aproximaram do movimento político camponês por direitos trabalhistas
e por reforma agrária no campo brasileiro, o trabalho de concepção do
filme, a reconstituição de sua filmagem, com o uso excepcional das
rigorosas imagens originais para reencenar o trabalho e a época, vinte
anos depois. Neste momento, ocorre o golpe, e a historia da filmagem o
registra \emph{na própria carne}. Todo o trabalho social daquele pacto
entre sul e norte, classe média e vida popular, estudantes e
trabalhadores é simplesmente destruído pelo exército brasileiro, em um
gesto de força que é bem mais do que uma metáfora do destino do país,
das imensas clivagens e das forças que de fato importariam a partir daí.
Os estudantes precisam fugir às pressas, as atividades políticas
camponesas são proscritas. As lideranças populares são perseguidas,
presas e torturadas e precisarão \emph{se exilar no próprio país}, em
uma atualização radical e desde baixo da famosa sentença de Sérgio
Buarque de Holanda, que dizia respeito a nossa própria experiência
cultural.

Então, com toda esta matéria prévia para que tenhamos perfeitamente em
mente o contexto, o sentido e o valor das ações históricas de todos
aqueles sujeitos, os jovens estudantes artistas e intelectuais de
esquerda, a classe trabalhadora recentemente organizada no mundo rural
brasileiro, o seu pacto produtivo de trabalho por um novo país, e também
o exército brasileiro agindo como cão de guarda, capitão do mato, das
elites questionadas da época, o filme nos dá notícia de como aquela
história singular, aquele pequeno evento, que se dava em um dos centros
das ações que envolveram 1964, foi noticiada na cidade do Recife. E a
notícia, tanto quanto o golpe e a ditadura brutal que se seguiria, era
um verdadeiro escândalo de distorção, e poderíamos até mesmo dizer de
\emph{assassinato}, de qualquer nível de verdade possível sobre aquela
história.

Segundo a pequena notícia, o exército brasileiro havia desbaratado na
região da Galiléia \emph{um foco de treinamento de esquerdistas
internacionais --} de fato, o exército \emph{acreditava} que se tratava
de uma milícia cubana --. O núcleo guerrilheiro era fortemente armado,
tinha sofisticado aparato técnico de propaganda para realização da
lavagem cerebral dos camponeses locais e preparava uma série de
assassinatos na região, com base no filme de propaganda que era exibido,
\emph{Marcados para morrer}. Por fim, o treinamento era contínuo,
acontecia dia e noite.

Afora o ridículo, o patético e o cômico da situação distorcida, da
realidade explícita e extrema da mentira pública, ela era mais
precisamente trágica e agônica, já que violenta ao extremo. Ao mesmo
tempo que, vista à luz do filme, tal \emph{pulsão pela mentira} era
intensamente reveladora. Pessoas foram presas, foram torturadas e foram
mortas com base naquelas falsificações radicais, próprias da época, atos
simbólicos torpes e comuns, bem articulados à ditadura real, que jamais
foram reparados, ou julgados. Era a real cultura da mentira de nosso
fascismo nacional comum, que foi extremamente importante no jogo
político que fundamentou o golpe de força à direita, e sua ditadura
fundada nestas mesmas bases simbólicas.

Das grandes violências históricas descritas, ou escritas, sobre momentos
de terror do avanço autoritário brasileiro, a falsa noticia documentada
no filme de Coutinho que justificava a plena repressão sobre o movimento
social no campo brasileiro tem verdadeira correspondência com o momento
de \emph{transe}, de horror e de grande mentira, descrito com tensão
limite no início de \emph{Memórias do cárcere}, quando o dedicado e
democrático funcionário da educação pública de Alagoas, e autor do
livro, Graciliano Ramos, é preso no início da ditadura Vargas, por ser
comunista, quando, de fato, ele era preso por não ter concedido
vantagens pessoais indevidas à sobrinha do famigerado tenente que o
mandava para cadeia e exílio no próprio país.

Estas são as violências extremas brasileiras, simbólicas e reais,
tradicionalmente franqueadas ao espírito autoritário conservador do pior
brasileiro. E nossa arte e pensamento, com trabalho limite e espírito de
negatividade realizou esforços públicos significativos para deixar clara
a natureza deste fundo não ultrapassado da vida brasileira.

A mínima notícia de jornal em estado de plena mentira de \emph{Cabra
marcado para morrer} é boa formação simbólica, material e objetiva,
mesmo que comum, de algo de nossa vida pública e política. Esta formação
permite que julguemos modos de ser da nossa tradição do ódio e da
violência \emph{desde} \emph{cima}, que, na vida brasileira, durante
muito tempo não pode ser responsabilizada publicamente, julgada ou
condenada. Do mesmo modo, fundado narcisicamente no mesmo princípio de
mentira liberada, pudemos ver hoje o honrado deputado Jair Bolsonaro
dedicar o seu voto pelo impedimento da Presidente Dilma Rousseff ao
honrado coronel Brilhante Ustra -- um homem que a teria torturado --
aquele que foi o único torturador do exército de torturadores do Estado
de terror de 1964 declarado como tal pela justiça da democracia
brasileira. Mas, segundo Bolsanaro, Ustra não foi um torturador -- como
as filmagens de \emph{Cabra} \emph{marcado} \emph{para} \emph{morrer}
eram um foco de guerrilha cubana -- e sim um homem bom. E o próprio
Bolsonaro não é um fascista, mas um democrata. E, como temos visto todos
os dias, muitos no Brasil pensam deste modo, em um movimento regressivo
quase demente que só aumenta e que põe muita coisa em risco.

Este movimento ainda funciona no pleno registro da mentira interessada,
que confunde linguagem com força, que foi o da notícia da guerrilha
cubana na Galiléia, de 1964. A má informação, a má fé absoluta do
episódio, faziam parte necessária do sistema do horror real, no qual o
plano simbólico é reduzido ao plano da ação criminosa de um Estado real.
Um Estado de terror e antissocial que se estabeleceu no Brasil em 1964,
com base naquele tipo de ação desonesta e de mentalidade, consentida e
promovida, que autenticava a repressão sem limites, falsa em seu próprio
fundamento.

Das muitas grandezas que \emph{Cabra marcado para morrer} põe e revela
sobre o Brasil, esta pequena passagem de \emph{montagem intelectual}, da
notícia plenamente mentirosa, da pura falsificação, da afirmação de uma
guerra inexistente, de uma invasão cubana inexistente, com armas e
treinamentos de guerrilhas inexistentes, do exército brasileiro, do
jornal pernambucano e de seus donos -- estes sim, violentamente
existentes -- não é das menos importantes. Ela revela a contrapartida do
Estado de terror, que também é sua fundamentação ideológica, presente no
\emph{terror da falsificação} da vida simbólica partilhada, advinda do
desejo do poder, no limite do delírio. Foi exatamente este tipo de
movimento que o ditador Diáz, o vitorioso da crise política do Brasil
alegórico de \emph{Terra em transe}, nomeou em seu famoso discurso final
sobre o país: ``\emph{Aprenderão, pela força...}''

Se a notícia era mentira em toda a linha e em todos os termos, ela era
construção de imaginário e vida pública real, e era verdadeira como
ação, na direção da redução da vida política ao novo estado de guerra.
Redução de toda a tensão e do trabalho político \emph{do outro}, e dos
múltiplos outros, que é o que caracteriza a democracia, ao estatuto de
inimigo extremo e objetificado, pronto para ser exterminado, por que,
segundo a própria ideologia, ele visa o \emph{nosso} extermínio. E
extermínio aqui é a palavra adequada: a vida politica se torna guerra
real nesta mentalidade, apoiada na expansão simbólica sem controle de
uma guerra fria mundial, o continente simbólico de onde se sonhava tal
sacrifício primitivo. O inimigo está na posição do mal absoluto, o risco
originário à própria civilização do poder, e por isso ele deve ser
simplesmente extirpado, destruído, exterminado. Quando falamos em
fascismo diante de ações simbólicas como estas não estamos sendo
condescendentes com os conceitos: Wilhelm Reich lembra, em seu livro
sobre a psicologia do nazismo, que o modo de Hitler situar em \emph{Mein
kampf} politicamente o lugar dos judeus no processo civilizatório era
exatamente este, o de um absoluto negativo que punha em risco toda a
vida e o desenvolvimento da civilização positiva, ocidental, ariana, que
tinha seu ápice nele próprio. Todo pensamento fascista funciona assim.

A pequena notícia de jornal, do Diário de Pernambuco, é plena mentira
histórica. Mas é também, plena ação política, positiva e de ódio. O
plano da ação e da violência alcançou e modulou o plano do imaginário,
da história e da linguagem. E é exatamente isto, em um contexto
tecnológico complexo, de grandes e fragmentários fluxos de informações
circulando na velocidade do pensamento pelo mundo, que habita a noção
contemporânea, que ganhou o mundo, de pós"-verdade. Este tipo de objeto
da cultura é sintoma de sua dimensão autoritária e perversa a um tempo,
mas é ato de violência preciso de quem o opera, bem construído.

O que estou tentando sugerir é que todo ódio é também uma mentira. No
Brasil de hoje temos muitos cidadãos que fizeram política pesada baseada
em suas \emph{visões do inferno}. Eles viram no Brasil a Venezuela, no
\versal{PT} um aparato estalinista, em Lula um proto"-ditador chefe de quadrilha,
nas políticas sociais e culturais do governo de esquerda o prenúncio da
revolução comunista de 2015 no Brasil, nos tratados comerciais com a
China, nos médicos cubanos do programa \emph{mais médicos}, bem como no
recebimento de meia dúzia de miseráveis haitianos pelo Brasil, uma
evidente invasão do país pela China, por Cuba e pelo Haiti, de
guerrilheiros que ocupariam o país para iniciar a revolução comunista de
hoje.

Muitos gritaram nas ruas, em pleno 2015 do fim do governo de Barack
Obama, que pessoas como eu deviam \emph{ir para Cuba}¸ que o \versal{PT} cindiu o
país e inventou a corrupção organizada no Brasil. Em um nível ainda mais
escandaloso estas vozes também tomaram periódicos e jornalistas
radicalmente parciais, unilaterais e, sendo assim, positivamente
antiéticos. Estes são produtores de \emph{meios atos} degradados e mais
do que patéticos daquele mesmo tipo de ação política da mentira,
justificativa de violência real, da notícia do jornal pernambucano de
1964 à respeito da interrupção das filmagens de \emph{Cabra marcado para
morrer}, uma obra prima do cinema político de todos os tempos e lugares.
Todas estas ações da mentira tentaram colocar o \versal{PT} no lugar de inimigo
absoluto da civilização, o que o \versal{PT} simplesmente não é. Os problemas do
\versal{PT}, por mais sério que sejam, são exatamente os mesmos problemas de
nosso sistema político geral.

Este é um mecanismo de violência que o Brasil cultivou, sobre o qual ele
fundou a sua modernidade contemporânea e que criticou de modo
insuficiente, superficial. A democracia conviveu com a suspensão da
crítica a esta ordem de violências, fechando os olhos cuidadosamente à
coisa. Permitiu o elogio e a proteção dos mecanismos de recusa e de
distorção da realidade política de seu próprio tempo, por aquele tipo de
homem e de produção simbólica que visa a implementação da violência como
política, e no discurso. Agora, ela colhe os frutos desta reserva
brasileira \emph{da mentira} como ação política. A pulsão à mentira de
hoje justifica a interrupção, com argumentos jurídicos frágeis, mas com
uma ideologização feroz e espetacular à direita, de um governo eleito.

É a ordem da tradição autoritária brasileira se apresentando novamente
no nosso novo mundo, da cidadania e do \emph{fascismo} de consumo.

\chapter{Crise e alucinose, anticomunismo do nada}

É um consenso do campo político democrático e progressista brasileiro --
aquele que poderíamos definir, em um grau zero de posicionamento, como
\emph{comprometido} \emph{ética e juridicamente} \emph{com os direitos
humanos universais --} que nosso longo último período de exceção
antidemocrático, o da grande ditadura civil"-militar de 1964-84, não foi
suficientemente nem corretamente \emph{elaborado.} Em uma certa medida
simbólica importante, aquele regime ditatorial não foi transformado em
linguagem, lei e psiquismo, no processo público de construção da
democracia brasileira recente. A história concreta de nosso trabalho
social, psíquico e ético, de nossa saída da ditadura, o período
contemporâneo conhecido como \emph{redemocratização}, não implicou uma
profunda e verdadeira democratização \emph{das mentalidades} e do
\emph{fundo autoritário brasileiro}, uma entidade nacional que tem
profundidade histórica de longa duração.

A norma do barramento da memória que pudesse fazer efeito público legal,
e a suspensão de uma justiça de transição legítima o suficiente, pela
Lei da Anistia de 1979, para muitos uma lei extorquida pelo campo
autoritário com o interesse estrito de impedir a punição dos homens de
Estado envolvidos em assassinatos, desaparecimentos e torturas de
brasileiros, foi plenamente vitoriosa no desenho e no controle do
processo jurídico da transição democrática a respeito das grandes
violências ditatoriais. E esta situação concreta excêntrica, do
impedimento do direito da democracia de julgar a ditadura, pelo direito
mais forte da ditadura de tutelar a democracia, mais uma vez, como é
comum no caso brasileiro, está em nítido desacordo com os tratados
internacionais a respeito da justiça das transições democráticas, que o
Brasil também assina.

Deste modo, diante da norma universal ocidental, algo ainda se passava
no país de modo análogo ao tempo da origem em que o Brasil, pressionado
pela Inglaterra, havia proibido o tráfico de escravos desde 1830, porém
continuava a praticá"-lo, bem como a conceber a própria sociedade como
escravista, tornando"-se uma espécie de nação pirata diante do direito
internacional da época, diante dos termos da modernidade do século \versal{XIX},
já regulada por contratos entre as classes -- e assim, legitimando o
bombardeio pelas canhoneiras inglesas de navios negreiros brasileiros,
na própria costa do Brasil... --. Do mesmo modo, no tardar da hora
histórica do fim do segundo mandato do Presidente Lula, já em 2010/2011,
passados \emph{seis mandatos presidenciais} sob o signo da
\emph{redemocratização}, o Brasil foi finalmente condenado, em um
processo movido pelas famílias de assassinados e desaparecidos pelo
Estado ditatorial, na Corte Interamericana de Direitos Humanos da \versal{OEA}, a
reestabelecer \emph{algum processo de justiça de transição}, até então
barrado pela lei de Anistia impingida, não realizado de modo a
corresponder ao direito internacional.

Passados 26 anos do fim da ditadura civil militar de 1964, por influxo
externo, e mais uma vez correndo o risco de se tornar um país
\emph{pária} do direito internacional que também professava, o Brasil
foi finalmente obrigado a instalar a sua relativamente tímida Comissão
Nacional da Verdade sobre a sua ditadura. Uma obrigação internacional de
direito interno, que, ainda uma vez, promoveu a mobilização radical de
vozes locais plenamente a favor do não julgamento das violências
imprescritíveis realizadas na última ditadura -- \emph{íntima --}
brasileira.

Se o problema se esgotasse neste ponto tudo seria muito ruim, mas, ainda
assim, também uma solução tardia. Mas, de fato, nada para aí. Como se
sabe, a sustentação de uma política corrente e cotidiana \emph{contra os
direitos humanos} no Brasil é uma vexatória, para não dizer bárbara,
constante pública entre nós. Ela se origina, em um nível histórico
imediato, exatamente na necessidade de legitimar e proteger a ação
criminosa de agentes públicos -- militares, policiais, burocratas --
sustentados por dinheiro empresarial civil, da ditadura de 1964-1984. E,
em horizonte mais distante, mas talvez mais importante, o ativo
\emph{anti"-humanismo} brasileiro se enraíza na longuíssima tradição de
concentração extrema de poder, antipopular, dos trezentos anos coloniais
de escravidão e do século original do Império escravocrata do Brasil.

Escravidão significa uma massa social de trabalhadores forçados, sem
direito pessoal algum, e uma casta senhorial com direito ao sadismo
franqueado sobre o corpo do escravo, sua \emph{propriedade}. Um mundo
escravista é, de fato, o real negativo de qualquer possibilidade de
existência de \emph{direitos humanos}. E, hoje, como é possível
constatar todos os dias, políticos de direita, radialistas, pastores, e
programas de tevê diários, revistas, e uma parte significativa da
opinião pública em uma verdadeira ``tempestade'' na internet, entre
pobres, classe média e ricos, enuncia cotidianamente, felizes em seu
direito à posição de violência, que não deve haver \emph{direitos
humanos para bandidos no Brasil}, quando, como se sabe, a lei
internacional é aquela que precisamente garante \emph{aos bandidos} o
direito universal. O arcaísmo satisfeito, contemporâneo, da posição
irracional é nítido, positivo na sua condescendência com \emph{o novo
terror}.

Por isso, e pela falência interessada do Estado nesta matéria --
compreensível, sendo assim as coisas do direito social ao sadismo da
nação -- no relatório de janeiro de 2015 da Human Rigths Watch, ficamos
sabendo, mais uma vez, que o Brasil \emph{convive com abusos crônicos
como tortura, execuções extrajudiciais, impunidade de crimes cometidos
durante a ditadura e má condições de seus presídios}. Além do novo
destaque, da espetacular e crescente ação de extermínio da polícia, por
exemplo, em São Paulo e no Rio de Janeiro: ``O número de pessoas mortas
em decorrência de intervenções policiais nesses Estados aumentou
drasticamente em 2014 (40\% no \versal{RJ} e mais de 90\% em \versal{SP})''\footnote{\versal{BBC}
  Brasil, ``Direitos Humanos: relatório de \versal{ONG} crítica mortes pela
  polícia e prisões medievais'', 20/01/2015.}.

Esta \emph{recusa} pública consumada em aceitar a regra democrática
universal, que atravessa campos interessados da política, da mídia de
massas e da polícia no Brasil, é uma \emph{forma}, um princípio político
reiterado, de franquear o espaço público à tradicional posição
autoritária brasileira e de insistir nela. Ela tem correspondência com
uma outra posição, \emph{também limite e de recusa}, irracional,
mobilizada como corpo, como grupo e como voz nas manifestações
sucessivas de 2015 pelo impeachment da Presidente Dilma Rousseff.

Do mesmo modo que a recusa em aceitar a regra ocidental mundial, já
tradicional, da Declaração Universal dos Direitos Humanos da \versal{ONU} de
1948, cria uma distorção, para a satisfação de alguns, do \emph{lugar do
Brasil no mundo}, produzindo realidade pública de pensamento político e
social \emph{exterior} ao que é universal -- distorção particular que,
lembrando uma noção do psicanalista Wilfred Bion, podemos chamar de
\emph{alucinose -- um dos discursos} fortes que exigem o impedimento da
Presidente funciona de modo análogo a esta poderosa \emph{formação}
psico"-política brasileira.

Este discurso particular, muito acentuado nas manifestações das ruas,
\emph{de que o governo petista é comunista, estalinista, de que os
petistas, além de ladrões consumados -- ``petralhas'' diz o mau
jornalista animador deste público -- e um câncer no Brasil,
transformaram o Brasil na Venezuela,} é a consumação algo delirante, a
realização simbólica plena atual, da tradicional posição autoritária à
direita brasileira. Diante das distâncias impressionantes destas
enunciações de alguma realidade das coisas políticas brasileiras se
torna muito claro o que Bion quis dizer com o \emph{nome} psicanalítico
de \emph{alucinose}: uma distorção efetiva da \emph{capacidade de
pensar} fundada na necessidade de \emph{saturar} a realidade com desejos
que não suportam frustração, bem como no impacto corrosivo dos
mecanismos psíquicos ligados ao ódio sobre o próprio pensamento.

É quase degradante, indigno do que importa, termos que produzir uma
medida para a grave distorção, muito interessada, que este discurso,
aproximado da violência, significa no espaço público brasileiro. Mas
chegamos a este nível das coisas, no grande processo atual de regressão
democrática, do qual, sem dúvida, a política ampla do \versal{PT} frente o Brasil
também faz parte. Assim, tornou"-se necessário lembrarmos simplesmente o
óbvio: diferentemente da Venezuela, no Brasil a Presidência da república
está limitada a uma reeleição, a Justiça, o Ministério Público a Polícia
Federal são autônomos e ativos -- importantes membros do partido do
governo, e do governo, foram condenados -- a imprensa é independente e o
grande Capital nacional sempre esteve bem representado no governo. Além
disto, o governo petista foi aquele que dinamizou o \emph{capitalismo de
mercado interno brasileiro} durante os anos de 2004 a 2010, em um nível
de atividade e integração social, via aumento de empregos formais, até
então inéditos no país. E, por fim, o governo petista paga as contas
políticas da sua aliança, corrupta, com um setor arque"-tradicional do
\emph{Capital nacional}, as imensas empreiteiras que controlavam o
Estado para seus interesses, ao menos desde a criação de Brasília, e
não, de nenhuma ação de socialização levada à cabo. Neste processo,
brasileiro, do \emph{populismo de mercado} lulo"-petista, não há traço de
comparação político institucional possível com a Venezuela chavista.

Mas, relembrar estes dados históricos concretos nada significa para
aqueles que, ao ouvi"-los, devem gritar que o lugar deste autor é em Cuba
-- quando \versal{EUA} retomam relações com Cuba.... --; se não começarem a
expressar o pensamento como \emph{passagem ao ato}, de bater nas panelas
para calar o adversário na linguagem -- em uma metáfora muito concreta,
já no limite da ação física, do desejo evidente de \emph{bater, usar a
força} e \emph{calar} --.

De fato, o processo da regressão política, da alucinose, de grande parte
daqueles que ocupam as ruas brasileiras hoje contra o governo é
verdadeiramente \emph{espetacular}. Ele soube ligar a tradição de longa
duração do \emph{anti"-humanismo antidemocrático} autoritário brasileiro,
com a nova organização midiática de eventos, massivos, e com a
ambiguidade satisfeita da grande mídia nacional frente às mazelas reais
do quarto governo petista. Os pontos políticos reais -- a crise
econômica, que tem vínculo com o desaquecimento da economia mundial, e o
grave sistema de corrupção que, condescendente com o modo tradicional de
fazer política no Brasil, corroeu o governo petista e seu capital
simbólico político -- já seriam suficientes para uma crítica severa ao
governo. No entanto, esta oposição necessita de mais: necessita de
argumentos falsos, do desejo de projetar sobre o governo um objeto
interno mal, o comunismo inexistente. Esta duplicação imaginária do mal
do governo, \emph{os comunistas inexistentes}, tem a função psíquica de
liberar velhas fantasias autoritárias brasileiras, pois, sendo o governo
\emph{comunista} ele deve ser tratada como um \emph{comunista,} no
código de doutrina de guerra da velha Guerra Fria -- já liquidada no
mundo -- ou seja anulado e destruído. Por isso, por uma verdadeira
política do direito ao ódio, e sua ação como política, \emph{o
comunista} é o objeto fetiche negativo do anticomunista, e do
anti"-petista atual, \emph{assim como o escravo era o objeto fetiche do
senhor de escravo}, e o judeu, o do nazista. Ele é a vítima fixada, cuja
culpa está dada por ser ela própria, para o pleno direito da ação
política do ódio.

Tudo pareceria apenas farsa, se a força política de tais mecanismos
subjetivos, ancorados na política do direito ao ódio, e do ódio como
política, não mantivesse intocada, como uma reserva histórica possível,
a longuíssima tradição autoritária brasileira, aquela mesma que, na
saída de nossa ultima ditadura não foi publicamente criticada nem
elaborada. A farsa, de uma guerra fria particular anti"-petista, dos
anticomunistas do nada atuais, \emph{produtora de alucinose no lugar de
pensamento,} organizada em \emph{recusa da história} e em
\emph{fetichização} do seu \emph{objeto mal}, que permite, no limite, a
ação violenta, restritiva dos sentidos, quer tornar"-se tragédia,
dissolver os parâmetros da política democrática, produzindo um
\emph{novo estado de transe brasileiro}. De fato, alimentando políticos
de direita na sua prática de forjar provas para o impedimento da
Presidente, ilegítimos por que denunciados no mesmo esquema de corrupção
que abala o governo, a nova farsa subjetiva do homem autoritário
brasileiro, \emph{anticomunista do nada}, foi força de produção do
\emph{transe} contemporâneo, a dissolução dos limites e da ordenação
entre os diversos poderes em jogo.

Como se tornou um consenso agora, \emph{a crise política}, e seu grau de
alucinose irracional, aprofundou seriamente a crise econômica. Mas, como
funciona o psiquismo parcial do \emph{anticomunista do nada}, a culpa de
tudo será sempre do inimigo, aquele que deve ser vencido em uma guerra
de liquidação total, a \emph{comunista} Presidente Dilma Rousseff, que
\emph{transformou o Brasil na Venezuela}. Ele próprio, em um último
mecanismo deste psiquismo, um tipo de \emph{self fascista}, está sempre
desobrigado em reconhecer as próprias responsabilidades, e os resultados
das próprias ações -- precisamente como a Lei de Anistia brasileira
tornou prerrogativa -- no processo violento, destrutivo e irracional
que, de fato, protagoniza.

\chapter{A extrema direita de hoje e o Brasil:\\ modos de usar}

O processo político histórico recentíssimo do Brasil surpreendeu a
muitos por recolocar no plano da vida pública, e da produção de energia
política e até mesmo de proposições políticas positivas, uma dimensão e
uma entidade político social estranha, mas íntima, e dada até ontem por
ultrapassada e extinta. Trata"-se da velha configuração social de uma
particular direita radicalizada, uma força política imaginada,
comprometida com arcaísmos aos quais se deve determinar o caráter, até
então de pouca extensão e representação, mas que passou a ter função
política efetiva no processo da forte crise de governo contemporânea,
tornando"-se mais intensa e algo estratégica no tempo do que o
tradicional conservadorismo cordial, ou banal, brasileiro. Uma força
ativa na sua concepção própria, e distorcida, de mundo, tendente ao
extremo de um sistema de ideias que, sem o prejuízo de um excesso,
buscam e são a repercussão de um real sistema delirante das coisas
políticas que evita intensamente dobras ou veleidades dialéticas.

Durante um bom tempo do andamento da crise radical do ultimo governo de
ampla coalisão petista peemedebista, crise real do próprio sistema
político brasileiro como hoje se sabe inteiramente, o nosso mundo
estável dos conceitos políticos e do entendimento da vida social se
recusou a incluir, a compreender e a lidar com a presença e a função
estruturante significativa da nova velha extrema direita brasileira em
todo o processo produtor daquela mesma crise. Não só o fenômeno, a sua
linguagem, os seus objetos e a sua subjetividade não eram olhados de
frente, como problema real de política e de conceitos sobre o Brasil, e
sobre o Brasil atual, como ele era desconsiderado nas próprias funções
que passou a ter \emph{na real organização} da crise, na ação prática
que passou ao primeiro plano das forças políticas existentes do país.
Como Adorno notou bem um dia sobre este ponto, tudo se passava como se o
sistema oficial da inteligência \emph{não pudesse nomear e agir
politicamente sobre a realidade do fascismo} presente no seu próprio
horizonte. Um ponto cego politico, uma formação negativa e inconsciente,
mas também um lance estratégico de política real, como veremos, que
sempre cala sobre o pior e ainda uma vez, e que nos cabe perguntar sobre
a sua própria origem, a natureza da sua própria recusa.

Agora, passados dois anos da malfadada eleição do segundo mandato
petista de Dilma Rousseff, e do cerco de todas as proporções e de muitas
forças que se abateu sobre o seu governo, que nunca chegou a existir de
fato dada a forte paralisia estratégica politica que levou a sua queda,
todos se espantam tardiamente com o real resto politico negativo e
insistente, a fratura exposta da vida politica nacional, tendente a
posições aproximadas de alguma modalidade de fascismo, da presença
politica cada vez mais evidente de um Jair Bolsonaro no país -- como se
sabe, um ex"-militar de baixa patente, formado em regime autoritário,
apologista da ditadura de 1964-1984, da tortura e do assassinato de
pessoas de esquerda, homem que já declarou com ênfase, em espaço de
poder democrático, o seu machismo misógino, o seu racismo e sua
homofobia --. Uma presença movimento que, ao mesmo tempo que se
configura como grotesca e autoevidente piada de mal gosto sobre o nível
político e simbólico que o país alcançou, é ação politica positiva clara
de desprezo efetivo por direitos acordados historicamente, e convite
aberto, no limite da prática, à violência direta como ação política. Tal
presença mórbida, \emph{necropolítica} como dizem os novos teóricos
críticos, também passou a representar \emph{estranho} \emph{e}
\emph{familiar} projeto de futuro, e de degradação nacional triunfante,
no horizonte rebaixado agonístico do próprio jogo contemporâneo do
capitalismo mundial, que produz efeitos semelhantes, mas diferentes, em
várias outras nações.

De fato é sempre preciso lembrar e afirmar que, no caso brasileiro, a
politica oficial da redemocratização em relação aos agentes do terror de
Estado da ditadura de 1964-1984, às ações da extrema direita presentes
no poder ditatorial -- homens que mataram, torturaram, sequestraram e
desapareceram com brasileiros e, ainda em 1981, colocavam bombas em
espaços civis -- foi praticamente nula. A lei geral aceita a respeito do
terror de estado brasileiro foi a real lei política outorgada pela
própria ditadura civil"-militar, a lei da Anistia de 1979, de fato uma
lei de autoproteção e auto"-anistia dos homens bárbaros da ditadura
sustentados pelo poder brasileiro. A partir deste princípio protetor
forte da extrema direita da ditadura da Guerra Fria brasileira, na ativa
conciliação da Nova Republica e do processo geral da redemocratização
brasileira, o espaço público e politico nacional suspendeu o trabalho de
elaboração do seu passado violento e traumático recente, qualquer desejo
de operar sobre a sua forma e energia, passando diretamente à fantasia
protetora e superficial de uma realidade pacificada por mero desejo,
fantasia distorcida defensiva de uma hegemonia cordial nacional, agora
modernizada e democrática. Assim o Brasil não só produziu uma exceção em
todo o sistema simbólico legal universal das politicas de justiça e de
transição, barrando o \emph{julgamento real}, e portanto também o
simbólico em alguma medida, sobre a sua ditadura íntima, mas também
gestou e manteve viva a ideia de uma real tutela do campo autoritário,
sobretudo do Exército e das forças armadas brasileiras, mas também dos
senhores e empresários que os sustentaram, sobre o sistema geral da
democracia. A ditadura brasileira determinou a lei central a seu
respeito que posicionou o próprio Estado de Direito democrático que
deveria ultrapassá"-la, e não o contrário, como de fato preconizam os
acordos e as normas internacionais de transição. A ordem democrática
estaria sobre uma determinação mais ampla, uma concessão de força
cedida, mas não superada, de tais forças.

Noutras palavras, a extrema direita ditatorial brasileira foi protegida,
destacada do processo de julgamento real e simbólico e premiada no
processo de democratização, com a própria lei de anistia tutelar com que
invadiu e deformou o espaço democrático. Apenas após seis governos
democráticos seguidos, e por ser derrotado e condenado em um tribunal
internacional de direitos humanos em ação movida por familiares de
desaparecidos, na \versal{OEA}, que não reconheceu a legalidade de nosso rápido
recalcamento politicamente interessado da ditadura, o Estado brasileiro
inventou uma tardia Comissão Nacional da Verdade, para simular uma
justiça de transição que nunca existiu entre nós. Ela passou a
funcionar, sem prestígio nem força, com imensos constrangimentos
causados pelas forças armadas que mal cumpriam suas determinações,
durante o primeiro governo de Dilma Rousseff, processo que teve, como
veremos, efeito importante na mobilização original contemporânea da nova
extrema direita brasileira.

Deste modo, a democracia brasileira carregou o enclave simbólico de uma
força arcaica que, não podendo ser \emph{julgada}, permaneceu viva como
um \emph{real} transcendente a todo movimento histórico. Ao \emph{fim do
processo}, a \emph{forma} final do Brasil \emph{sair de sua ditadura
íntima}, sem sair, confirmava a estrutura da presença intocada e
retornante do nosso mesmo atraso e violência antissocial, muito bem
percebidos e formalizados por Glauber Rocha em \emph{Terra em transe},
de 1967, um filme que tentava exatamente diagnosticar como a vida
política brasileira da época girou mesmo em falso, e como a força
arcaica de uma repetição infinita de um autoritarismo antissocial
extremo brasileiro nos fez então \emph{entrar na ditadura}...

Do mesmo modo, como todos sabemos, o processo geral da redemocratização
não desalienou as polícias locais de seus fortes vínculos militares, de
forma a buscar e indicar um estado social de democracia plena como um
sistema de segurança fundamentalmente cidadão; ao mesmo tempo que, ao
longo do tempo, com a sustentação de múltiplos e sucessivos governos de
direita e com o imaginário anti"-social e racista tradicional nacional
reinvestido nesta política efetiva de Estado, o espaço da democracia
liberou e autorizou a real ação paliativa da crise social brasileira,
sempre \emph{produzida} \emph{adiada}, de um direito informal mas
constante ao extermínio dos cidadões \emph{matáveis} pelas policias
brasileiras, mantendo contínuos, agora sobre a massa de pobres
nacionais, os gestos de violência, tortura e assassinatos
característicos do período ditatorial.

Tal convivência com o estado de exceção permanente, com um certo grau de
\emph{extermínio} \emph{paliativo}, para pobres e negros no Brasil, sem
falar na realidade da violência constante contra povos indígenas, é de
fato uma politica continuada de nosso democracia invadida pelo campo
autoritário nacional desde a sua raiz, e se ela não é dita pelo poder
público, é afirmada praticamente, mantendo"-se em um nível infantilizante
ideológico de um banal segredo de polichinelo de Estado: uma politica
sabida e reconhecida por todos, inclusive internacionalmente, mas
simplesmente ilegal, contra os códigos acordados desde a constituição de
1984 e os vínculos do país com os protocolos internacionais de direitos
humanos.

Assim se configura, ainda mais uma vez, o fundo prático cindido de nossa
mentalidade autoritária, atuando em um espaço contra a própria lei
simbólica, e afirmando o falso vínculo estratégico com a lei -- para a
imagem superficial de civilidade dos homens bons da nação --
simultaneamente e tudo ao mesmo tempo agora. É uma versão pesada,
ctônica, de terror e de necropolítica afirmativa, -- 60.000 assassinatos
nos últimos anos no país, dos quais as policias participam com cerca de
10.000, contra a ordem da lei... -- para a velha figura da
\emph{volubilidade} de caráter nacional, de fato, em termos
psicanalíticos clássicos uma estrutura \emph{socialmente}
\emph{perversa}, forma definidora das elites brasileiras desde a leitura
de Machado de Assis a seu respeito, ainda no século \versal{XIX}.

Nosso século \versal{XIX} em que, aprofundando o ponto, o Brasil estabeleceu a
lei do fim do tráfico internacional de escravos visando a abolição
completa ainda em 1830, a reafirmou como novidade em 1850 e concedeu o
termino da escravidão formal, por fim, em 1888, de modo a fazer o espaço
social nacional atravessar o século de \emph{forma real}
\emph{escravocrata}, mas também de \emph{modo legal} afirmado
publicamente contra o tráfico. O homem político nacional advindo daí se
tornava assim um nem isso nem aquilo histórico e social, ou ainda, bem
ao contrário também, um isso e aquilo simultaneamente. Precisamente o
que Machado de Assis descreveu na forma genial de seu \emph{Brás Cubas,}
de 1880, e que Roberto Schwarz em um outro momento de recrudescimento e
exposição do autoritarismo nacional, nos anos de 1960 e 1970, reconheceu
a estrutura profunda de formação subjetiva social: liberais
escravocratas, homens modernos submetidos às leis de trocas
internacionais do mercado de \emph{commodities} de seu tempo, abertos e
checados pela cultura moderna, cientifica e industrial dos países
centrais, mas também amarrados com satisfação em seus espetaculares
privilégios concentracionários, antissociais e sádicos locais, pela
afirmação continuada da forma da vida e da formação escravocrata.

Retornando ao ponto contemporâneo: se a política de extermínio mínimo
prática de Estado e de governos sobre a massa de pobres e jovens negros,
os cidadãos matáveis do país, não é afirmada pelos agentes públicos que
a promovem, mantendo"-se a fachada fetichista de normalidade democrática
e de direitos civis efetivos, instância ideológica para o uso narcísico
e satisfação civilizatória dos próprios senhores no processo -- que além
de assassinos são também civilizados, democráticos e elegantes -- por
outro lado, tal prática se tornou uma clara política cultural, produtora
de indústria ampla no plano da comunicação social e da política, gerando
um texto matriz corrente de radical ressentimento e de ataque aberto aos
direitos humanos. São os muitos programas de rádio e televisão
conservadores que se espalham por todo o país, de extrema direita, por
que não dizer, que acontecem de manhã até a noite, e que se utilizam da
crise social e a insegurança continuada de pobres e classe média para
explorar e estruturar o desejo primitivo de vingança, e a fantasia
reparadora arcaíssima de que o sacrifício, o extermínio direto do mal
social, desde que pobre e excluído, pode dar conta do mal brasileiro
mais amplo.

Cria"-se assim uma cultura complacente de violência prática contra
pobres, a imensa maioria negros, que alimenta e sustenta a posição
perversa de Estado e de governos, de matar e deixar matar e de negar o
extermínio simultaneamente. Estes dois campos, como aliás foi muito bem
diagnosticado no \emph{block} \emph{buster} popular dos anos 2000
\emph{Tropa de elite}, particularmente o filme \versal{II}, são coordenados e
caminham na mesma direção, a da criação de uma hiper cultura da
violência nacional, legitimada, de aberto desrespeito de direitos e de
classe, que investe as polícias em todos os níveis de um excesso de
poder, cultura que é ao mesmo tempo totalmente afirmada, tornada prática
e também negada, por esta gigantesca maquinaria ideológica de Estado.

Trata"-se de uma ativa cultura da violência sádica e compensatória contra
pobres e negros brasileiros, que tenta equilibrar os maus resultados da
integração social nacional, da instabilidade e da segurança, de uma
sociedade mantida constantemente em risco pelo espetacular e imutável
processo concentracionário da renda, plenamente, e cada vez mais,
reafirmado por esta modalidade de democracia danificada, muito efetiva e
ativa. Não por acaso, como medidas políticas coordenadas do governo de
pouca legitimidade atual, mas de grande interesse classista, um governo
que construiu a sua chegada ao poder sobre o aprofundamento da crise
econômica brasileira, que toma medidas recessivas e que ataca a
estrutura acordada de direitos sociais da constituição de 1988, para
aumentar a força do capital frente à do trabalho, também, o mesmo
governo, e ao mesmo tempo, faz grandes investimentos em polícia,
equipamentos e estrutura de segurança pública... Tudo indica que a real
descompensação social do poder, que aumenta o seu já fantástico poder de
concentração no Brasil, deve ser compensada por uma integração social na
violência, uma ação repressora generalizada de força constante sobre a
vida social nacional. Um exemplo empírico: desde que o governo de
direita chegou ao poder, sem passar por uma eleição, todas as
manifestações públicas criticas ao governo, simplesmente todas, sempre
terminam com o ritual simbólico da repressão violenta afirmada pelo
governo e sua polícia, sinalizando que tais direitos democráticos estão
sobre tensão, não são inteiramente desejáveis, e, no limite, podem
deixar a qualquer momento de serem aceitáveis... É a forte relação
existente, em todo lugar, entre o aumento da ação política econômica de
tipo neoliberal e o aumento simultâneo das práticas repressivas ativas
de governo.

Estes são elementos históricos sociais prévios para o entendimento da
manutenção do espírito de extrema direita brasileiro, as ações decisivas
do fundo autoritário nacional, espírito político baixo que foi
espetacularmente reativado na crise política de 2015 e 2016. Porém,
antes de pensarmos mais detidamente a natureza velha nova deste fenômeno
de politica contemporânea brasileiro, quero evocar um retrato da
convocação pública à direita que se mobilizou naqueles anos de crise
para a derrubada do governo petista em seu recém ganho quarto mandato.
Escrevi o texto que se segue em meados de 2015. Ele fazia parte de um
conjunto mais amplo, que buscava entender as várias forças de
desestabilização entrópicas que acabaram por produzir o impeachment de
Dilma Rousseff, com atenção para a mobilização espiritual, prática e
linguageira da nova direita tomando as ruas, o espaço público politico
visível nacional, sua busca de produção de força politica e atuação, uma
verdadeira novidade no período democrático. Chamei este texto, que fazia
parte de um ensaio sobre a destruição, e a autodestruição, do ultimo
governo petista, de ``Anticomunismo, antipetismo'':

``Estas tensões políticas, clivagens e afastamentos sociais do governo
de Dilma Rousseff foram a base da convocação de um outro tipo de agente
social, que acabou por ser a fera de ataque mais dura, organizada e
eficaz, para a corrosão atual da mística petista. Com o realinhamento
gradual e real do grande capital contra o governo, \emph{o homem
conservador médio}, antipetista por tradição e anticomunista por
natureza arcaica brasileira mais antiga -- um homem de adesão ao poder
por fantasia de proteção \emph{patriarcal e agregada}, fruto familiar do
atraso brasileiro no processo da produção social moderna -- pode entrar
em cena como força política real, deixando de expressar privadamente um
mero ressentimento rixoso, carregado de contradições, contra o relativo
sucesso do governo lulo"-petista, que jamais pode ser verdadeiramente
compreendido por ele.

Com as eleições, e o apoio senhoril assegurador do grande dinheiro, que
voltava a ser genericamente antipetista, este povo se manifestou em
massa. Com a bomba atômica da corrupção na Petrobras revelada,
explodindo no colo da Presidente logo após a reeleição -- a verdadeira
ficha do desequilíbrio político final -- esta camada média, que havia se
organizado ao redor de um candidato e que não se conformara com a sua
derrota, ganhou o instrumento definitivo, agora de fato \emph{real},
que, junto com a sua própria nova organização, de produção midiática de
espetáculo de massas, e de muita estratégia na internet, gerou a nova
paixão política conservadora pósmoderna brasileira. O desequilíbrio mais
profundo da política no capitalismo de consenso geral brasileiro (...)
tendia a se desequilibrar fortemente para a direita, \emph{nova velha}.

Assim, antipetistas indignados com a corrupção do outro, e
anticomunistas do nada, tomaram as ruas para produzir o texto para os
grandes conglomerados de mídia nacionais repercutirem, o que ocorreu, em
tempo real. Estas forças \emph{herdaram as ruas} a partir dos levantes,
originalmente críticos ao governo, mas à esquerda, ocorridos em 2013, se
apropriando da legitimidade política e simbólica do que era um outro
movimento.

Embora esvaziado em todo o mundo, e particularmente no modo de conceber
o poder da até ontem bem sucedida esquerda democrática brasileira, a já
tardia ideia de ``comunismo'' parece ainda ter uma vigência imaginária
importante no Brasil, e está bem presente, surpreendentemente, no fundo
da ação na rua desta grande fração das classes altas brasileiras. Onde
as coisas são assim, pode"-se afirmar com alguma certeza um fracasso de
racionalidade do vínculo entre pensamento e política.

Construção que vem de bem longe, ponto de apoio e ideia central para a
instauração de duas ditaduras \emph{parafascistas} no difícil século \versal{XX}
brasileiro, foco de uma guerra mundial pela hegemonia de Impérios, o
anticomunismo sobrevive magicamente no Brasil de hoje como uma espécie
de imagem de desejo, para a grande simplificação interessada da política
que ele de fato realiza. Ele mantém o discurso político em um polo muito
tenso e extremo de negatividade à qualquer realização democrática ou
popular de governo; ou melhor, ele é contra qualquer realização que
desvie a posse imaginária do Estado de seus senhores, imaginários, de
direito.

Para antipetistas, movimento de desfaçatez do velho anticomunismo, basta
atribuir ao governo o epiteto de estalinista, ou bolivariano -- e gritar
nas ruas que `aqui não é a Venezuela', como se algum dia o Brasil o
tenha sido -- para poder se livrar de explicar todo o sentido real da
política brasileira. Trata"-se de um sortilégio, da redução da política
ao maniqueísmo interessado mais simples, na esperança de desfechos já há
muito impossíveis, do tipo guerra fria.

A dinâmica democrática e viva entre as classes e o governo é
transformada deste modo em um gesto de desejo imediato, em uma luta
imaginária limite, contra os comunistas inexistentes. E, me parece, isto
apenas quer dizer que o governo deve ser derrotado \emph{in extremis}. O
anticomunismo é estratégia extremada -- ancorado no arcaico liberalismo
conservador brasileiro, com fumos de fidalguia, as famosas raízes do
Brasil, de origem ibérica e escravocrata -- de resgatar o governo de
compromissos populares quaisquer, mesmo quando estes compromissos, como
no caso dos governos Lula e Dilma, sejam de fato os da inserção de
massas no mercado de consumo e de trabalho, evidentemente pró mercado,
capitalista.

E, de fato, é necessária uma fantasmagoria limite, exatamente por isso:
foi o governo de esquerda que deu uma certa solução política para o
avanço capitalista bem paralisado no Brasil do neoliberalismo periférico
dos anos 1990, dirigido pela grande elite econômica nacional. Bem ao
contrário da alucinose dos homens que ainda usam os termos próprios da
guerra fria, como se sabe, o governo de esquerda dinamizou intensamente
o capitalismo de mercado interno brasileiro, alcançando de fato um
virtual estado de pleno emprego no Brasil.

A taxa de desemprego caiu sem parar durante os governos petistas, de
12,4\% em 2003 para 4,8\% em 2014, enquanto, de 2009 a 2014, nos Estados
Unidos, origem da crise mundial, ela oscilou de 10\% para 7\%, na Itália
ela foi de 7 para 13\%, na França de 8,5 para 10,2\% e na Espanha..., de
18 para 27\%; e por isso mesmo, nos valores hegemônicos de uma cultura
total de mercado, tal governo só poderia ser vencido se lhe fosse
projetado o velho desejo autoritário brasileiro, o mais puro
anticomunismo com toques de moralismo neo"-udenista, que, mais uma vez,
nada tinha a ver com o caso.

Por isso, inimigos políticos paralisados pelo sucesso mais geral do
governo Lula foram revolver os porões psíquicos do passado: após a
vitória de Lula com Dilma, Fernando Henrique Cardoso propôs, de modo
envergonhado, mas convicto, que o \versal{PSDB} guinasse à direita e José Serra
utilizou"-se abertamente de retórica anticomunista em sua campanha contra
Dilma Rousseff. Justo eles dois, um dia vítimas da prática de ódio
político com que agora flertavam. Essa linguagem já se tornara quase
óbvia na campanha de Aécio Neves, campanha derrotada, provavelmente,
pelos pobres empregados do Brasil de 2014.

Vejamos os termos sociológicos, e a janela de oportunidades, de Fernando
Henrique Cardoso, para esta guinada do partido, contra um discurso
político ``visando o povão'', a favor do que chamou de \emph{novas
classes possuidoras}, que deveriam ter os próprios interesses aguçados
por uma nova política à direita; e a favor do acento do discurso
moralista de elite, que fatalmente encontraria a velha estratégia
retórica do anticomunismo brasileiro:

``Enquanto o \versal{PSDB} e seus aliados persistirem em disputar com o \versal{PT}
influência sobre os `movimentos sociais' ou o `povão', isto é, sobre as
massas carentes e pouco informadas, falarão sozinhos. Isto porque o
governo `aparelhou', cooptou com benesses e recursos as principais
centrais sindicais e os movimentos organizados da sociedade civil e
dispõe de mecanismos de concessão de benesses às massas carentes mais
eficazes do que a palavra dos oposicionistas, além da influência que
exerce na mídia com as verbas publicitárias. (...) Existe toda uma gama
de classes médias, de novas classes possuidoras (empresários de novo
tipo e mais jovens), de profissionais das atividades contemporâneas
ligadas à \versal{TI} (tecnologia da informação) e ao entretenimento, aos novos
serviços espalhados pelo Brasil afora, às quais se soma o que vem sendo
chamado sem muita precisão de `classe c' ou de nova classe média. Digo
imprecisamente porque a definição de classe social não se limita às
categorias de renda (a elas se somam educação, redes sociais de conexão,
prestígio social, etc.), mas não para negar a extensão e a importância
do fenômeno. Pois bem, a imensa maioria destes grupos sem excluir as
camadas de trabalhadores urbanos já integrados ao mercado capitalista
está ausente do jogo político"-partidário, mas não desconectada das redes
de internet, Facebook, YouTube, Twitter, etc. É a estes que as oposições
devem dirigir suas mensagens prioritariamente, sobretudo no período
entre as eleições, quando os partidos falam para si mesmo, no Congresso
e nos governos. Se houver ousadia, os partidos de oposição podem
organizar"-se pelos meios eletrônicos, dando vida não a diretórios
burocráticos, mas a debates verdadeiros sobre os temas de interesse
dessas camadas. (...) Seria erro fatal imaginar, por exemplo, que o
discurso `moralista' é coisa de elite à moda da antiga \versal{UDN}. A corrupção
continua a ter o repúdio não só das classes médias como de boa parte da
população. Na última campanha eleitoral, o momento de maior crescimento
da candidatura Serra e de aproximação aos resultados obtidos pela
candidata governista foi quando veio à tona o `episódio Erenice'. Mas é
preciso ter coragem de dar o nome aos bois e vincular a `falha moral' a
seus resultados práticos, negativos para a população. Mais ainda: é
preciso persistir, repetir a crítica, ao estilo do `beba Coca Cola' dos
publicitários. Não se trata de dar"-nos por satisfeitos, à moda de
demonstrar um teorema e escrever `cqd', como queríamos demonstrar. Seres
humanos não atuam por motivos meramente racionais. Sem a teatralização
que leve à emoção, a crítica moralista ou outra qualquer cai no
vazio.''~\footnote{``O papel da oposição'', Revista Interesse Nacional,
  no. 13, abril/junho 2011.}

\versal{FHC} simplesmente sinalizou, em um discurso estranho e novo à leitura
política nacional, muito assemelhado aos cálculos sociais de
marqueteiros americanos, a brecha possível para a emergente \emph{tea
partização} do espaço público da política brasileira, um movimento
apaixonado de busca de submissão extrema de tudo ao mercado e sua
estrita produtividade -- \emph{jacobinos do mercado} -- que também
animou, em outro círculo do conservadorismo, o delírio arcaico do velho
anticomunismo brasileiro. Anticomunistas do nada, velhos autoritários
anti"-populares e novos \emph{tea"-partistas} em busca de um Estado
estrito para a multiplicação de seus negócios, iam de mãos dadas. E
incluíam também na foto, feliz, pela primeira vez como ator democrático,
não por acaso, a problemática Polícia Militar paulista.

Também, no período de ascensão e queda petista, atacar com a máxima
retórica, isenta de responsabilidade, em jornais, blogs ou revistas, o
comunismo imaginado do governo, tornou"-se um dos modos mais fáceis e
oportunos de ganhar dinheiro no mercado dos textos e das ideias no
Brasil. Era suficiente reproduzir a rede de ideias comuns e fixadas, com
sua linguagem agressiva, indignada artificial, que sustentassem todo dia
o mesmo curto circuito do pensamento. Simplificação espetacular e ponto
certo no imaginário autoritário, jornalistas, articulistas, programas de
tv e de rádio e revistas inteiras passaram, durante anos, a ler as
atividades do governo do ponto de vista extremo, limitado, do
anticomunismo imaginário. Além de anacrônico, havia algo de
verdadeiramente preguiçoso neste processo mental político. Antigos
artistas, verdadeiros comunistas dos anos 1960 -- os nomes são
conhecidos de todos -- se prestavam a vender opiniões imediatas,
atacando faceiramente o aberto \emph{estalinismo} dos governos de Lula e
Dilma. Surgiram os muito duvidosos heróis intelectuais do gênero.

Embora a imprensa fosse absolutamente livre, a Polícia Federal, o
Ministério Público e a Justiça trabalhassem como jamais no Brasil, e
desde o segundo ano do governo Lula a cúpula petista estivesse sobre
processo criminal aberto e acabasse de fato inteira na cadeia, durante
anos homens muito inteligentes nos garantiam todos os dias nos jornais a
natureza ditatorial fixada -- alucinose -- do governo petista.

O delírio interessado, farsesco, não conhecia limite, uma vez que se
desobrigava radicalmente de checar realidades. O fato de, contrariando a
opinião garantida destes estranhos pensadores, sempre dada por certeza,
Lula não ter se aventurado por nem um segundo na busca de um terceiro
mandato, como era previsto -- bem ao contrário do comportamento de \versal{FHC}
quando na Presidência -- também não os sensibilizou para os compromissos
democráticos do Presidente petista. E gradualmente, se abria mais e mais
o espaço para este tipo de regressão, \emph{wishful thinking}, da
leitura da ordem da política, impingindo o delírio apolítico, trabalho
mágico obsessivo, como a medida real das coisas brasileiras.

No limite, chegamos a conviver cotidianamente, em grandes jornais, com
articulistas que atacavam qualquer ideia ou projeto progressista, de
interesse coletivo, solidário ou, até mesmo, apenas meramente humanista.
Os novos modernos anticomunistas liberais do mercado concentracionário
brasileiro tangenciavam o fascismo, um tipo muito próprio de fascismo de
consumo, como dizia Pasolini. Daí a emergência lógica de um discurso
final, atual, baseado no mesmo jogo grosseiro de redução da política, da
ideia apoteótica de extermínio definitivo do \versal{PT}...

O fato do governo Dilma ser obrigado a convocar, algo contra a vontade,
uma Comissão Nacional da Verdade, após o Brasil, no apagar das luzes do
governo Lula, ao final de 2010, ser enfim condenado na Corte
Interamericana de Direitos Humanos da \versal{OEA}, também mobilizou a ira de
velhos torturadores aposentados, amigos e parentes de torturadores e
saudosos brasileiros de ditadura de todos os tipos, que, em tal
panorama, puderam falar contra a tardia Comissão da Verdade da
democracia brasileira, e o governo, sem sofrerem nenhum constrangimento
de opinião pública, ou legal.

Como se sabe, tais homens bons foram cruelmente perseguidos pela sanha
revanchista dos comunistas derrotados, que haviam tomado o poder de
assalto em 2003 e, assim, estes homens bons estavam legitimados, pelos
próprios interesses, a retornarem ao ideário de 1970, época em que
torturavam, matavam e desapareciam com brasileiros... Era preciso manter
a paranoia alimentada.

Os anticomunistas, agentes reais de ditadura, foram convocados pela
mínima política reparatória forçada à esquerda, pois foram incomodados
em suas aposentadorias especiais e premiadas. Pela estratégia geral da
luta política contra o governo eles foram cinicamente tolerados.

Assim se produzia o campo extremo, algo delirante, em que a luta
democrática antipetista encontrava a velha tradição autoritária
brasileira. E, por isso, agora que o país, em seu neo"-transe, se levanta
contra os comunistas inexistentes, em uma ritualização do ódio e da
ideologia, elegantes socialites peessedebistas e novos empresários
\emph{teapartistas} convivem bem, nas ruas, fechando os olhos para o que
interessa, com bárbaros defensores de ditadura, homens que discursam
armados em cima de trios elétricos, clamando por intervenção militar
urgente no Brasil e sonhando com o voto em Jair Bolsonaro. Não por
acaso, em regime de farsa verdadeira, se vislumbrou nas passeatas de
março o semblante das velhas marchas conservadoras de 1964.

Assim, todo o campo dos anticomunistas do nada, incluindo elegantes
estadistas e cientistas sociais, prestou desserviço à qualificação do
debate público brasileiro para a vida contemporânea, que ainda é
seduzido e obrigado a pensar, por estes homens, regressivamente, com
parâmetros vencidos de mundo, construídos em 1959. Este campo também é
movido, em uma certa facção da elite que o anima, por uma verdadeira
política \emph{identitária de} \emph{classe}, cujo lastro organizador de
mundo é o ódio antipopular brasileiro\emph{. }

Tal grosseria imatura e interessada seria simplesmente inaceitável por
alguma vida política minimamente informada; se não se apoiasse em
espetaculares erros reais do governo, que talvez, imaginariamente,
entenda que a crítica às suas práticas graves seja apenas a ideia fixa
delirante do anticomunismo do nada, e não um gradual e verdadeiro
afastamento de suas bases políticas.

O anticomunismo atrasado brasileiro é regressão da política. Regressão
aos argumentos de força e redução da diferença, e implica gozos baixos,
do ódio que poderia se alçar ao sadismo, da simplificação da toda vida
pública e social e do direito ao desprezo a respeito do destino da vida
popular. É uma política do direito ao ódio fixado, frente à vítima
escolhida.

Ele tende, como pode se observar facilmente no Brasil hoje, a reduzir a
linguagem mediada dos problemas ao gesto de força, na panela, ou no
corpo do inimigo.\footnote{\emph{Dilma Rousseff e o ódio político}, São
  Paulo: Hedra, 2015, pág. 35-44.}

Assim tentei retratar o movimento, a retórica e a paixão da nova
direita, em meio ao próprio tempo de suas crescentes passeatas nacionais
de 2015. Se o quadro e o diagnóstico tiverem algum valor, sob a pressão
do presente que o produziu, fica claro no texto como toda a ordem de
força popular da chamada nova direita, incluindo setores sociais
diferentes, estava baseada e ganhava força na não discriminação, na
perda de limites simbólicos claros, entre uma supostamente existente
direita moderna e democrática e uma explícita e apaixonada direita
arcaica e autoritária brasileiras. Esta vinculação de novos liberais
radicais, de tipo americano \emph{tea party}, e velhos autoritários
saudosos de ditadura, de tipo brasileiro \emph{ordem e progresso}, se
deu sob o signo geral, unificador de diferenças e ele mesmo referência a
um passado imaginário de violências que tenta ser reativado, falsificado
historicamente, do que tenho chamado \emph{anticomunismo do nada}.

De fato os movimentos de 2015 e 2016 puseram na mesma rua sob uma
bandeira comum, novos empresários liberais, ricos e socialites tucanos,
antipetistas genéricos, classe média moralista anticorrupção,
evangélicos conservadores e homens radicais de extrema direita, inimigos
declarados dos direitos humanos, apologistas da violência policial e de
Estado, desejosos de uma intervenção militar entendida de modo
onipotente e mágico. Esta convivência ampla em um mesmo espaço das
várias direitas brasileiras, diferentes, que foram unificadas pelo
movimento de massa da eleição -- em que seu candidato antipetista foi
vencido por 54,5 milhões de votos -- e pelo mote estratégico --
perverso, imputado ao campo adversário e recusado e protegido
cuidadosamente no próprio, em um jogo evidente de dupla moral -- do
moralismo político anticorrupção, produziu esta forma de união em algum
ponto do desejo político, e isto não se deu por nenhum acaso. Para
vencer em uma crise de paralização do governo e institucional o campo
social majoritário que acabara de ganhar de fato a eleição no país era
necessário a união de todas as forças restantes no outro prato da
balança nacional, forças que haviam sido derrotadas, mas por uma
diferença pequena, de apenas 3\% dos votos. A recusa aberta do resultado
da eleição, uma surpresa política impensada até então, já sinalizava
fortemente para a construção de uma postura radical de desejo e
particularidade, na sua raiz já extra legal, como fonte legitima de ação
política. Era necessário rapar o taxo da vida social representativa no
limite de todas as forças encarnadas próprias da direita, para manter a
força de uma recusa histórica extra legal. E a organização de novos
grupos de direita, a intensa mobilização e construção de redes de
repercussão e de influência na internet e a ação midiática geral
espetacular, \emph{neutramente} \emph{a favor} das posições manifestadas
nas ruas, logrou manter o campo unificado e mobilizado naquela ação
política de fundo.

Qualquer ponto produtivo da \emph{realidade} pública partilhada,
discursivo, falacioso, mentiroso ou mesmo real, interessava também e foi
mobilizado para a produção do campo oposicionista que ultrapassava os
limites da legitimidade de uma eleição, em fúria, e por desejo, frente à
oportunidade histórica de derrubar o governo petista em seu quarto
mandato consecutivo. Ainda mais com a verdade de uma crise de corrupção
real e imensa revelada, cujo valor foi inteiramente imputado ao governo,
embora pertencente ao sistema geral da política, e os efeitos de uma
crise econômica real e mundial, que finalmente alcançava o país, uma
crise que podia ser aprofundada ao infinito com as próprias práticas
radicais de paralização do governo em que a oposição se lançara, na
busca ativa da construção do impeachment.

Na rua, era mesmo necessária a energia total do movimento popular da
direita, que desse ares de amplo consenso para a violência política e
institucional que de fato se buscava produzir. Dos desejosos de
impeachment, aos de intervenção militar ou de retorno da monarquia,
todos eram muito bem vindos. Deste modo, a socialite cosmopolita de
havaianas brasileira tida por moderna deveria \emph{ir de mãos dadas}, e
de olhos bem fechados, com o ex"-torturador autoritário que se sentira
lesado pela politica reparatória mínima dos governos de esquerda -- de
populismo de mercado interno de Lula e de desenvolvimentismo de Dilma.
Invertia"-se o valor amoroso, o conteúdo da promessa social, da imagem
clássica de união e solidariedade do poema moderno de Drummond. E
realizava"-se a aproximação da direita mais dura por velhos
peessedebistas, preconizada alguns anos antes por Fernando Henrique
Cardoso.

Isto se deu deste modo por necessidade da produção de poder. Assim foi
porque assim se produzia a energia politica máxima de um movimento que
devia aparecer como \emph{de massa}, \emph{como sendo a opinião pública}
e que também necessitava, em um nível importante de produção de
política, da força \emph{de uma paixão}, de um \emph{pathos} próprio, um
\emph{delírio ativo}, unificador e produtivo, que tivesse força de mover
o desejo amorfo do todo. Era a produção daquilo que Fernando Henrique
Cardoso, como um verdadeiro marqueteiro político, chamou de
\emph{teatralização que leva à emoção}, ou seja, os rituais simbólicos
de grupo e de massa que expressam, sustentam e produzem a \emph{emoção},
o \emph{pathos} ou a energia política viva, que gera a disposição para a
ação -- no caso, passear domingo à tarde com fascistas brasileiros na
Avenida Paulista, como se todos fossem pessoas honestas e boas, diante
do inimigo maior e mais monstruoso do que o próprio fascismo nacional
tradicional, que se reapresentava desde um passado necropolítico mal
recalcado para um presente re"-encantado pelo próprio transe.

Este movimento social, com sua representação política em figuras
nefastas e criminosas da democracia, como o peemedebista evangélico de
direita, fascista de consumo, deputado Eduardo Cunha, que para muitos
levou a uma quebra institucional, para outros fabricou um impeachment
artificial e para outros ainda escancarou a falácia dos jogos de força e
lei em uma democracia capitalista contemporânea, simplesmente firmou e
legitimou a voz do fascista nacional no espaço público, ao utilizar"-se
plenamente dela, dando solução de compromisso ativa entre os novos
liberais do mercado total e os velhos autoritários antissociais
brasileiros, em busca de violência real na vida política e social. Os
fascistas foram de fato convocados e utilizados, pelo campo político
interessado, como grandes produtores de mentiras históricas e de energia
passional disponível para a passagem ao ato que são. De resto, foi a
velha aliança ditatorial entre autoritários arcaicos da formação
nacional antipopular e liberais interessados na máxima exploração da
vida produtiva brasileira, a mesma que de fato sustentou ideologicamente
o regime civil militar de 1964-1984, que foi reeditada e rediviva nesta
nova festa pública da direita apoteótica pós"-moderna brasileira, que até
então, por dentro do jogo vigente, apenas perdera quatro eleições
consecutivas.

Em um importante trabalho apresentado em 1995 na Universidade Columbia a
respeito dos elementos fixos da mentalidade e do comportamento político
fascista, do que chamou de Ur"-fascismo, com o seu forte caráter
politicamente \emph{transcendente} de formas de produzir sentido que
neguem radicalmente o processo e o valor da história, Umberto Eco
apontou também o vínculo eletivo temporalmente especial entre classes
médias, direita e fascismo que estou tentando evocar aqui como tendo
acontecido no processo recente brasileiro, da derrubada do último
governo petista. Assim ele descreveu precisamente este ponto:

O Ur"-Fascismo provém da frustração individual ou social. O que explica
por que uma das características dos fascismos históricos tem sido o
apelo às classes médias frustradas, desvalorizadas por alguma crise
econômica ou humilhação política, assustadas pela pressão dos grupos
sociais subalternos. Em nosso tempo, em que os velhos ``proletários''
estão se transformando em pequena burguesia (e o lumpesinato se auto
exclui da cena política), o fascismo encontrará nessa nova maioria seu
auditório.\footnote{http://blogacritica.blogspot.com.br/2016/11/umberto-eco-ur-fascismo-o-fascismo.html}

Seguindo esta razão social das coisas da emergência do fascismo, os
jogos de pressão e de poder entre a vida imaginativa, as realidades
sociais históricas e as relações diretas entre as classes, que de fato
se expressaram fortemente no Brasil da derrubada do pacto social de alto
e baixo lulo"-petista, podemos chegar a dimensões psíquicas importantes
da estruturação de toda posição fascista de ação política. Em seu ensaio
Eco elenca os pontos de constituição da posição fascista na vida
psíquica e sua produção política -- culto da tradição, recusa da
modernidade, irracionalismo e culto da ação pela ação, ódio à crítica,
recusa da diferença, busca de unidade identitária, sentido de humilhação
e indignação social pela história, antipacifismo, elitismo, heroísmo,
populismo qualitativo e produção de \emph{novilíngua} -- e, dentre eles,
aquela referência a dinâmica histórica das crises, e seu efeito
subjetivante, que dispara a articulação de poder produzida entre as
classes \emph{que dá poder ao fascista}. Esta produção histórica
determinada, envolve violências imaginárias na sua raiz, risco e
humilhação, que já se estruturam como gesto de poder, desejo de
reescalonar as diferenças e marcar pela força real os lugares desejados
de poder.

Do mesmo modo, a emergência da condição psíquica do fascista e de seu
grupo em meio à ordem conservadora de classe que agora o aceita no
Brasil, gera e produz poder para o próprio projeto geral de direita.
Noutras palavras, as classes médias de direita, e o senhorio
economicamente mais forte que deve chegar ao poder através do movimento,
\emph{usam} os fascistas, dispostos a tudo pela ordem de um psiquismo
delirante em seus motivos transcendentais, plenamente voltados ao
direito à ação, ao prazer imediato do desrecalque da violência na
política, enquanto, no mesmo movimento, os fascistas \emph{usam} as
classes médias para conquistar um invólucro politicamente aceitável, uma
estrutura de defesa que proteja a sua real ilegitimidade, posição
verdadeiramente ilegal em um Estado de direito, para poderem ocupar e
agir no espaço público, espaço que, sem esta solução de compromisso
entre vida social média e terror baixo, tenderia a rechaçar a presença
fascista.

Como na formação de um sonho freudiano o elemento neutro da imagem do
sonho, indiferente, serve à estruturação defensiva do sonho para que
nele possam se disfarçar e aparecer as matérias relevantes e carregadas
de intensidade pessoal que dá de fato energia para o sonho -- o neutro
visível, e o intenso invisível, formam igualmente o sonho -- no
desrecalque social de uma direita de classe média tendente ao fascismo,
o grande cinturão de massas \emph{neutras} e \emph{de direita}, honestas
e normais se articula, protege, oculta e dá existência simultaneamente
ao núcleo fascista ativo, à extrema direita, que com sua convocação
delirante política gera o \emph{pathos}, a energia política necessária,
para que o próprio homem comum desmobilizado e medíocre enfim saia da
frente da televisão e se mova.

A energia produzida pela ação simbólica e pública dos pequenos fascistas
pós"-modernos é extrema e pode ser deslocada, por ter origem em um
sistema de razões absolutamente irresponsável, fundamentalmente uma
produção de força política comprometida com a mentira na sua raiz, um
cinturão significante que pode ser abandonado por outro a qualquer
momento, por ser totalmente falso. Falso na forma, e real na energia,
este é o segredo e a contribuição formal do fascista ao todo do
movimento da direita. Estudando as estratégias discursivas da extrema
direita brasileira na internet, e seu modo de ser afetivo e cognitivo
para a política -- para o desenvolvimento de um documentário que realizo
com Rubens Rewald sobre o fenômeno -- foi possível observar \emph{a
criação de uma ficcionalização radical da história}, um sistema fabular
fantástico e autônomo, que escolhe mínimos pontos de contato com fatos
reais, os isola e hiper"-investe, para fazer emergir o conto político
fantástico interessado como a verdade do processo histórico. Esta
construção de narrativas ficcionais, e de processos intensos de vivencia
imaginária desta história fabulada, tem a característica importante de
pressionar o sujeito, de fato assujeitado a esta produção, a um
\emph{sentido de urgência}, da \emph{iminência da destruição do mundo e
da reação necessária da guerra redentora}, que por um ato de violência
total e de sacrifício da parte má da vida social deve finalmente
reconquistar a estabilidade, a ordem e a paz, que seriam os motivos
desejados no horizonte de toda esta excitação política verdadeiramente
deformada.

Ficção, imaginação aterrorizada pela presença do objeto mau, iminência
da catástrofe são os fundamentos psicopolíticos que se desdobram em ato
continuo, necessário, urgente, legítimo, irreprimível porque portador
dos valores perdidos de toda a civilização já destruída pelo inimigo, do
gesto de força, da guerra e do sacrifício social. Assim trabalhou
politicamente, e convocou psiquismo e razões fabuladas, para esta forma
primitiva de \emph{sonhar pensar} a vida, e de exigir da politica uma
resposta imediata, a extrema direita brasileira contemporânea, que
encontrou no imediatismo e na representatividade direta da internet, de
plena aceitação de linguagem superficial ao extremo, um amplo campo de
liberdade e de matéria tecnológica à favor de sua própria performance
alucinada política.

Este é o movimento geral, da estrutura do continente psíquico e da
lógica produtiva de sentido, que se produziu e se reproduziu no espaço
público da extrema direita na internet ao longo de anos -- que
certamente já estava mobilizada, mas era ainda mínima, para os ataques
organizados e as manifestações que ocorreram contra os trabalhos da
Comissão Nacional da Verdade, no ano de 2011 e 2012. No ano de 2013 se
deu o \emph{break down} entre uma estrutura de pensamento político e
busca de ação ligada às ruas, de movimentos sociais independentes, e o
governo, um movimento originalmente disparado pela esquerda autônoma e
propositiva de ações socializantes mais fortes à favor da classe
trabalhadora, notadamente a demanda por transporte social gratuito nas
grandes cidades brasileiras. Esta ruptura simbólica ampla, que se tornou
uma semana de eventos críticos com massas na rua frente ao déficit de
qualidade social dos governos brasileiros, iniciou a quebra do
encantamemto da hegemonia política simbólica lulo"-petista e abriu o
espaço público para dois movimentos estruturantes novos do campo da nova
direita: a tomada das ruas como espaço de atuação política à direita,
pela primeira vez no período democrático, e a aceleração do trabalho de
fabulação, em muitos níveis, mais ou menos irresponsáveis ou reais,
sobre o sentido da experiência do governo de esquerda, culminando com a
luta simbólica diante da crise econômica e, em um segundo momento, a
crise de corrupção, para a derrubada da sequência de governos petistas
no Brasil.

Qual era a fabulação limite da extrema direita, construída coletivamente
em espaço de velocidade e irresponsabilidade gozosa total, ao longo dos
anos de 2010 a 2015, em grupos e chats na internet, e que alimentava a
disposição extremada para a ação, produzindo energia real para o
movimento mais amplo da nova direita no Brasil? Nestes casos sociais de
produção de sintoma e regressão ativa como força política real, o
movimento detalhado do próprio material narrativo onírico, fixado, é tão
importante quanto suas razões gerais de fundo. Segundo a extrema direita
espalhada em centenas de comunicações e grupos de internet -- em um
discurso comum que era falado, ou falava, pessoas como Olavo de
Carvalho, velhos militares de pijamas, jovens empresários brasileiros em
Miami até o ex"-rockeiro libertário Lobão... -- o processo politico
histórico lulo"-petista era essencialmente o seguinte: \emph{desde a
existência de um encontro de partidos e políticos de esquerda latino
americana acontecido em meados da década de 1990, chamado Foro de São
Paulo, se estabeleceu um plano amplo para a tomado do poder pela
esquerda em toda a América Latina, visando a criação de uma grande
pátria unificada socialista, cujo nome primeiro era Unasul; Lula e o \versal{PT}
eram agentes avançados desse processo e estavam em contato com forças
revolucionárias e movimentos de guerrilha latino americanos, como as
Farc da Colombia, de modo a investir e ajudar no processo revolucionário
mais amplo, e importá"-lo para o Brasil; o vínculo com o intervencionismo
chavista na Venezuela era real, orgânico e meta final do lulo"-petismo no
Brasil; as mínimas políticas de reconhecimento e identitárias
contemporâneas do governo visavam a criação de uma hegemonia política e
cultural da esquerdas no país, com vistas a facilitar a revolução
comunista que estava no horizonte próximo; as mínimas políticas de
recebimento de imigrantes pelo Brasil, refugiados haitianos, palestinos,
africanos e os médicos cubanos convidados pelo programa mais médicos do
governo federal eram na verdade a real importação de um exército
guerrilheiro internacional, que receberia armas enviadas pelas Farc e
pela China pelas fronteiras desprotegidas do país para fazer a guerra
revolucionária no Brasil; os acordos comerciais com a China e uma
plataforma genérica de intenções para a construção de uma estrada de
ferro ligando o Brasil ao Pacífico eram na verdade um acordo de
submissão e entrega do Brasil à China, que, após a revolução comunista
lulo"-petista, entregaria o país a milhões de chineses que chegariam
através da estrada de ferro prevista, que ocupariam as nossas casas;
armas estavam entrando no país e sendo estocadas em fazendas do
interior, para alimentar o exército do \versal{MST} e dos guerrilheiros
estrangeiros trazidos ao país pelo governo; a presidente Dilma estava
prestes a deflagrar a ofensiva revolucionária do governo; o roubo e a
corrupção petista era de fato a produção de dinheiro necessário para a
guerra revolucionária em curso; o exército brasileiro era a única
alternativa real e eticamente solida para barrar a revolução comunista
iminente, e já em curso, que corroeu as instituições, a cultura e a
política brasileira visando a desestabilização para a constituição de um
país socialista; a corrida contra a iminência do ataque comunista era
urgente e o exército brasileiro, única salvaguarda moral e institucional
disponível, interviria e não permitiria a destruição da nação pela
revolução esquerdista; a intervenção militar era iminente, e todos
deveriam se preparar para ajuda"-la politicamente e sustenta"-la nas ruas
e nos espaços públicos necessários; a guerra de salvação nacional
aconteceria a qualquer instante, e de fato ela já estava acontecendo...}

É possível perceber claramente com o desenvolvimento da ficção política
paranoica apocalíptica da extrema direita -- a ilusão unificadora de um
grupo, como dizia Winnicott -- tomada por real e produtora de ação
politica real, os movimentos psíquicos mais íntimos da nova ordem
pequeno fascista da política entre nós. Com base em um único ponto da
história -- a existência de uma reunião retórica de partidos e
movimentos de esquerda nos anos 1990, que não tinha nenhuma
correspondência com o sentido da ação política do governo lulo"-petista,
essencialmente um governo de desenvolvimento e expansão do mercado de
consumo interno brasileiro e de pacto desenvolvimentista com o grande
Capital nacional -- tomado por abstrato e desligado de todo o resto, com
a fetichização negativa forte da ideia antiga de comunismo, produz"-se
toda uma narrativa fantástica, que utiliza os elementos escolhidos da
história e só eles, como a vinda dos médicos cubanos para o Brasil por
exemplo, para a geração de uma pressão subjetiva presente, uma pulsão
para a guerra e uma disposição odiosa para a ação. Deste modo há ruptura
radical com os circuitos de entendimento ordenados e acordados da
história, recusa da sua existência, como o fato do governo petista ser
pró"-mercado por exemplo. Este descarrilamento dos termos historicamente
orientados leva a um escorregamento do pensamento rumo ao passado, uma
atração pelo passado, de fato um passado imaginado desejado para uma
estratégia política visando o presente, o anticomunismo do século \versal{XX} e
da guerra fria dos anos 1960, passado arcaico e delirante tido como a
verdade do presente e poroso a uma pressão desejante urgente, com a
configuração de um discurso paranoico final a respeito da ruína iminente
do que se tem por civilização e a inevitável guerra total, reparadora,
que tal degradação civilizatória necessariamente provocará.

Assim são três os movimentos psicopolíticos em jogo, configurando de
fato uma única produção de força e atuação delirante: recusa dos
elementos históricos complexos do presente, regressão imaginária radical
a um modo antigo de ordenar a história, que já é a escolha paranoica e
de ódio e pressão urgente por ação de violência, sacrifício e
restauração da civilização. Este sistema de delírio da extrema direita,
que em sua raiz e de modo total já justifica toda ação da
\emph{pós"-verdade} da direita em geral, tinha a função de pressionar e
forçar, transferir sua energia política, por uma ação imediata e sem
respeito algum pelo campo adversário, a todo o amplo movimento da nova
direita brasileira, que de um modo ou de outro participou deste tipo de
paixão, cujo próprio delírio de caráter sádico socialmente sustentado já
era realização sintomática, regressiva, do prazer imaginário da
liquidação do mal e do inimigo. Em um degrade de posições mais ou menos
estruturadas, que consideravam mais ou menos pontos históricos, todos se
utilizaram da energia política e do delírio intenso, para a produção de
política e pós"-verdade, da qual a extrema direita é uma espécie de fonte
pura do princípio da coisa. Daí os gritos, estertores de ódio, violência
comezinha e grosseira nas ruas, panelas batendo, pela deslegitimação
urgente de um campo político que botava toda a civilização em risco...
Tal regressão satisfeita à práticas políticas de pequeno sadismo, como
ocorre em todo movimento fascista, era também baixa produção de prazer,
legitimada pela idealização sublime de se estar salvando o país e o
mundo do mal absoluto. Este é o sentido preciso da ideia de regressão, e
da política como produtora de regressão, utilizada aqui.

Além das tradicionais formas transcendentes de uma configuração
psicopolítica delirante, fetichista, paranoica, rumo ao passado e à
evocação de alguma falsa tradição, já bem indicadas por Umberto Eco na
lógica do Ur"-fascismo, a direita e a extrema direta brasileiras tem sua
produção subjetiva e política, biopolítica, fundada na longa tradição
nacional do direito à exclusão radical dos direitos e da riqueza de
amplas massas de brasileiros, que não precisariam receber nenhum
reconhecimento da nação. De fato, historicamente nossas direitas sempre
trabalharam com a exclusão social tolerada, produzida e desejada -- a
diferença da direita para a extrema direita é a firmação ou não de tal
gozo: a extrema direita o afirma, a direita o nega, mas pratica -- com
os correlatos racismo, direito à ação violenta, aceitação inquestionável
do poder hierárquico ou econômico com adesão imaginária a ele, e ação
simbólica radicalmente desistoricizada e auto"-referida .

Conforme nossa longa tradição original escravocrata, uma parte
significativa do mundo do trabalho no Brasil não tem legitimidade e nem
necessita ser inscrita e reconhecida plenamente no plano dos direitos e
da vida material social. Nossa direita, e isto é um fato ideológico
único em qualquer nação industrial desenvolvida, pode desprezar e
excluir ativamente massas de trabalhadores brasileiros, pobres e negros
em sua maioria, do processo da produção e acumulação de riqueza
nacional. Este elemento simbólico forte, o desprezo antipopular e o ódio
pelo povo brasileiro e pelos pobres, é uma força particular única da
direita brasileira, que acrescenta um ponto a mais, que diz respeito à
história particular de nossa formação moderna como sociedade
escravocrata, nas posições de abstração da história e de regressão
elitista, anticrítica e anti"-intelectual, pronta para a passagem ao ato,
de toda formação fascista. E este elemento histórico forte, o desprezo
radical pela vida popular e pobre nacional, também é o responsável pela
nossa sólida ideologia de ataque contra os \emph{direitos humanos
universais}, e hoje, pelo ataque \emph{a toda a vida crítica e
intelectual} brasileira.

O \versal{PT}, e o governo Lula, que ousaram dirigir o processo histórico
brasileiro para uma expansão de mercado e riqueza com um grau mínimo de
partilha com os muito pobres, foram deslocados pela extrema direita para
este lugar único no mundo político contemporâneo: o ódio de uma elite
nacional por seu próprio povo. Com a força desta paixão arcaica se
conseguiu deformar todo o processo democrático brasileiro, e se pôs no
poder um radical governo de tipo neoliberal, real representante do
sistema da corrupção de Estado brasileiro, que em poucos meses atingiu
os direitos da classe trabalhadora no Brasil de modo que, em anos de
processo eleitoral e governos sucessivos, a direita jamais conseguira
realizar. É o sentido de urgência, delirante e antidemocrático,
transferido da extrema direita para o governo antipopular vencedor, e do
governo antipopular para todo o campo da nova direita brasileira.

\chapter{Democracia de extermínio?}

Há poucos dias Ricardo Silva Nascimento, um carroceiro reciclador de
lixo, negro, foi friamente assassinado pela polícia militar paulista,
com um tiro no corpo e dois na cabeça, ao pedir comida em um restaurante
do bairro de classe média de Pinheiros. A mesma polícia paulista que o
matou alterou as provas materiais do local, na frente de todos, retirou
o corpo ilegalmente e apagou a força celulares de quem filmou o crime.
Quando a tragédia de incompetência e desumanidade aconteceu, uma mulher
que estava no supermercado em frente à cena gritou: ``Tem que matar
mesmo'', legitimando como opinião o assassinato e revelando o nosso mal
geral como um universo de problemas muito amplo.

Não é possível negar que o Brasil passa por um momento grave. Um dos
aspectos que vai se tornando claro de nossa crise que se aprofunda é o
incremento de violência social, e de Estado, sustentada por um espírito
extremado que ganhou nova representação e estranha presença pública dita
democrática nos últimos tempos. O processo político e social brasileiro
dos últimos três anos, que determinou a atual composição insólita do
poder entre nós, tem vínculo com o grau de dissolução e de negação de
princípios civilizatórios fundamentais, de fato básicos, de que nos
aproximamos, e passamos a viver cotidianamente, nas bordas da lei. O
modo com que se produziu no Brasil a tomada do poder executivo por um
grupo muito duvidoso, sem o ato real de legitimidade das urnas, através
da construção de um impeachment, que também pode ser lido como um golpe
de novo tipo no processo institucional democrático, liberou e se
utilizou de forças sociais arcaicas brasileiras, abertamente
comprometidas com a violência como prática social aceitável, forças que
se imaginava terem sido superadas na busca de consensos médios da sempre
defeituosa democracia nacional.

Os processos da política pública e do incremento da violência social
como política estão ligados. Na representação social dos interesses que
chegaram ao poder em 2016, na construção da opinião pública anti"-governo
Dilma e na mobilização popular de grandes estratos da população
brasileira no ano que antecedeu o impeachment foi possível observar um
movimento político estratégico importante, que restaurou e usou posições
políticas extremadas, simplesmente falsas e irracionais, da tradicional
direita autoritária brasileira, trazendo"-a à cena de um momento de
contemporaneidade no qual, de fato, ela tem pouco a dizer, mas, de fato,
bastante a atuar, como ação direta de violência social primitiva,
sacrificial, tendente à exceção. Tal movimento, falsamente democrático,
de condescendência com o autoritarismo antissocial brasileiro,
reintroduziu e se utilizou abertamente da força do ódio na política, que
foi cinicamente tolerado como meio de interesse mais amplo, como todos
pudemos claramente ver desde então.

A democracia que se viu abalada politicamente, com a falsa solução
parcial anti"-petista para a crise do sistema geral de corrupção da
política brasileira, se viu abalada novamente na construção de um ataque
forte a direitos firmados na constituição de 1988 a partir de um pacto
de alinhamento pró capital, produzindo uma nova recusa dos direitos que
determinavam o horizonte e os parâmetros civilizatórios a serem buscados
pelos governos brasileiros, os parâmetros da lei simbólica coletiva, que
diziam respeito ao que se deveria sonhar como sociedade. E, em um
movimento contínuo, a democracia também se vê abalada constantemente
pela ampla emergência de violência social real de tipo tradicional e
conservadora, cada vez maior, que atua diretamente no mundo da vida
contra princípios de humanidade e de direito vigentes. Ataques a espaços
de direitos, como o direito de tradição iluminista da liberdade de
cátedra ou o direito indígena, constituição de milícias autoritárias,
ações cotidianas de constrangimento, afirmação de racismo, homofobia e
demofobia, ataques policiais gratuitos ou forjados à manifestações
democráticas de direito contra o estado do poder, tolerância com graves
ilegalidades policiais, em geral antipopulares, se tornam práticas
políticas presentes na vida brasileira de uma democracia que, vista daí,
apenas se degrada. Este tipo de vida política, que tende à ação
sacrificial que retorna em momento de crise aguda, que marca a vítima
para reintegrar a comunidade imaginária na afirmação do poder direto,
tendente a exceção, é um dos subprodutos da saída às ruas de massas
brasileiras à direita, para a derrubada do governo eleito em 2014. De
fato, nas manifestações pró impeachment vimos moralistas políticos,
falsos moralistas seletivos, liberais verdadeiros, neo"-liberais
autoritários e tradicionais autoritários antissociais brasileiros
caminharam de mãos dadas, espetacularmente, em nome de causa mais nobre
do que as próprias diferenças.

O processo de força social à direita, que para muitos produziu uma real
violência institucional, foi acompanhado desde sempre de modo grave da
afirmação e da legitimação de práticas de violência real como
alternativa à crise do sistema da política e da economia mundial que
envolve o Brasil. Não é estranho que agora, no momento de máxima
descaracterização da legitimidade de um governo que, tomando o poder
contra a corrupção da política -- mesmo que em um jogo movido por homens
sabidamente corruptos --, se revelou completamente enredado em corrupção
como não poderia deixar de ser, há ainda um aumento da pressão social
conservadora pelo direito à violência anti"-cidadã e antipopular.
Tratou"-se, em termos amplos, como fitas vazadas na televisão já disseram
claramente, da produção de um espaço politico real orientado por aquilo
que os psicanalistas chamam de lógica perversa, que diz que a lei que
vale para o outro não vale para mim, no qual se manipula e se constitui
poder por esta diferença. Esta é também a fonte de todo movimento
contemporâneo da construção de pós"-verdades, da real e satisfeita
cultura política da mentira, que é igualmente ato radical de violência
contra a ordem de sentidos acordados, em nome do desejo exclusivo do
poder. Assim, do mesmo modo, no mesmo campo, é coerente que já se fale
agora em alterar as regras do jogo institucional, duplicando e
configurando o golpe antidemocrático em definitivo.

Desta perspectiva mais radical, a crise estabelecida pelas próprias
elites brasileiras na gestão de um país destruído por elas próprias deve
ser paga pelo sacrifício social dos pobres e dos excluídos, em uma
espécie de solução para"-fascista para a nossa real incompetência
histórica. Um sacrifício que se dá tanto na exclusão da produção de
direitos, na qual os pobres não se representam, quanto na exclusão real
no mundo da vida, onde violências ilegais, de Estado ou não, acontecem
mesmo com liberdade. Não podemos nos esquecer nunca do tradicional e não
regatado fundo de cisões sociais e do direito ao sadismo antipopular de
nossa formação de quatrocentos anos de espaço social escravocrata,
colonial e nacional, que, como fantasia política de fundo, ainda modula
o desejo de extermínio, tortura e suspensão dos fundamentos da
democracia próprio de nossos autoritários, ressuscitados hoje pelo que
há de pior no Brasil, que também estão no poder.

Os atos de violência de Estado, covardia e incompetência técnica e
social, como a morte banal de Ricardo Nascimento, ou a água fria jogada
em miseráveis em um dia gelado em São Paulo, realizados por agentes
públicos e a serviço do público, representam a tomada do poder e a
guinada do Estado para a visão bárbara, ilegal e abertamente contra os
direitos humanos universais firmados pelo país, de setores da nova velha
direita nacional. É também o mesmo movimento de mentalidade que
multiplica de fato as ameaças aos direitos democráticos com o sonho,
pesadelo, da eleição de um ex"-militar de extrema direita, defensor
declarado de ditadura, de tortura, de armamento da população e de
assassinato de adversários políticos. Assim faz sentido que ao final do
ato de reação democrática e cidadã contra a violência de Estado, quando
da missa de sétimo dia do carroceiro Ricardo na Catedral da Sé, que
estava cheia, na qual os bispos auxiliares do Cardeal de São Paulo
denunciaram a real política de execução do brasileiro pobre que acontece
hoje no Brasil, durante as manifestações dos vários grupos de direitos
humanos que lá estiveram em conjunto com o povo da rua que acorreu à
cerimonia tenha surgido um grito popular espontâneo, de chamada à
responsabilidade e de nomeação de algum responsável pelo que de fato
está ocorrendo entre nós em relação aos direitos fundamentais à vida. A
população ali presente gritou, para quem quisesse ouvir: ``Geraldo
assassino'', sinalizando com clareza o nível de barbárie com que a
política social brasileira está se confundindo.

Não haverá pós"-verdade que possa apagar o sentido dos atos de crueldade
que a tomada do poder pelo vínculo de neo"-liberalismo e autoritarismo
brasileiro estão produzindo no país hoje, na degradação real da
democracia em um campo de liberdade para a força direta e para o mal,
regressivo e incapaz de dar conta verdadeiramente da vida contemporânea.

\chapter{O Carnaval da Tortura}

\begin{center}
\versal{I}
\end{center}

Vivemos tempos estranhos e de risco. Risco de degradação do espaço
democrático de direito no Brasil. E produtores desta degradação podem
ocupar cargos de responsabilidade de Estado.

Na última semana a Juiza Daniela Pazzeto Meneguine Conceição, da 39a
Vara Cível do Tribunal de Justiça do Estado de São Paulo, negou o pedido
em caráter liminar do~ Ministério Público de São Paulo para que o
desfile do bloco de carnaval "Porões do Dops'' fosse suspenso, caso o
bloco insistisse em usar esse nome e fazer aberta apologia da tortura e
do terror de Estado no Brasil. O grupo organizador do bloco, Direita
Brasil, ameaça homenagear o Coronel Carlos Brilhante Ustra, reconhecido
como torturador em Ação Declaratória de Culpa pela Justiça brasileira,
e~o Delegado Sérgio Paranhos Fleury, torturador explicitamente nomeado
pela Comissão Nacional da Verdade.

É absurdo, e diz muito da época, termos que lembrar os fatos legais de
fundamento de civilização que este ato de promoção de violência deseja
negar. O Brasil é signatário da Convenção Internacional da \versal{ONU} para
prevenção ao crime de tortura, bem como do Protocolo adicional à
Convenção, de 2007, que projetou a existência de um Sistema Nacional de
Prevenção à Tortura, que funciona no país desde 2015.

A Lei Federal 9455/97 especifica e criminaliza a omissão diante de casos
de tortura, seja ela realizada por cidadãos comuns seja por agentes de
Estado, crime passível de pena de detenção de um a quatro anos (Art. 2,
parágrafo primeiro). Um agente de Estado, por lei, e por justiça, tem o
dever conspícuo de evitar que o crime de tortura aconteça. Pela
liberdade de expressão do que fere o direito de todos, a Juíza se omite
frente a esta responsabilidade universal, entendendo que o perverso
direito de folia sobre matéria humana grave se sobrepõe à defesa da
vida.

Na prática, ao negar o pedido do Ministério Público, a Juiza legitima a
relativização da interdição simbólica do crime pela coletividade,
primeiro passo para a sua efetivação. Mesmo os militares brasileiros do
período ditatorial jamais assumiram a tortura como prática, nem a
defenderam. Não o fizeram para evitar a associação oficial à imoralidade
humana radical do crime, universalmente condenada.

A \versal{ONU}, e o Brasil que assinou sua Convenção, entendem que o crime de
tortura é um crime de oportunidade, quando não de terror de Estado. Ele
ocorre em locais e situações onde a supervisão social não está presente.
Daí a necessidade de que os espaços sociais de restrição de liberdade
tenham ``paredes de vidro''; de modo que a lei coletiva os adentre
praticamente, como luz, impedindo o ato de violência covarde. Muito mais
que buscar carrascos e sociopatas, que são efetivamente criminosos,
entende"-se que a condenação coletiva da tortura é que realiza a
prevenção. Relativizar a condenação coletiva do ato de tortura
representa imediata possibilidade de propagação da prática.

Trata"-se de um ultraje à democracia e à garantia dos direitos de todos.
O crime de tortura é mundialmente compreendido como crime de
lesa"-humanidade. Ele fere o cerne do que nos une, nossa condição humana:
se um de nós passa por isso, estamos todos feridos. Todos os
participantes e organizadores do ato abjeto e hediondo devem ser
responsabilizados, assim como aqueles que relativizam e facilitam a
existência do crime bárbaro de tortura entre nós.

\begin{flushright}
\emph{Tales Ab´Sáber e Aldo Zaiden}
\end{flushright}

\begin{center}
\versal{II}
\end{center}

Este pequeno texto foi escrito por mim e por um colega psicanalista com
importante atuação pública e governamental no espaço dos direitos
humanos e enviado a um grande jornal de São Paulo, para ser publicado na
sessão de debates, nas vésperas do carnaval de 2018. Não recebemos
nenhuma resposta, nenhum comentário, nenhuma mínima sinalização sobre o
destino do texto. Afora a incompetência rebaixada, de desconsiderar
agentes sociais legítimos e matéria social grave de modo sumário e
mal"-educado, o caso de conivência cultural com os elementos de barbárie
da nova direita brasileira pelo jornalismo da grande mídia brasileira
não é um acaso.

Toda crítica da violência liberada nas ruas, e nas ações de repressão da
política do próprio governo de direita pós impeachment, em geral, esteve
ausente das páginas frias e calculadas dos jornais brasileiros no
período. Jornais e revistas preferiram se concentrar nas tarefas de
desmontagem dos direitos do trabalho e de aposentadoria, para qual o
governo instável e autoritário da nova direita brasileira foi de fato
escalado. Todos pareciam saber, em um acordo tácito, que a ordem de
violências ativadas no mundo da vida fazia amplamente parte do processo
do choque conservador e neoliberal, tudo o que importava. E como bons
liberais, que não podem se misturar com o serviço sujo dos fascistas na
política, mas contam com ele, era melhor não tocar nesse assunto.

Críticas a violência, a desrespeitos aos direitos humanos, à mentira
excitada e constante dos grupos civis de extrema direta envolvidos no
impeachment, restaram isoladas do todo do sentido da cultura política,
como uma espécie de mania reificada da esquerda. Preferia"-se deixar as
coisas da degradação da vida pública da política correr, legitimar a
possibilidade ridícula de um presidenciável de extrema direita, garantir
o direito ao uso da violência direta do poder para o Capital, se assim
for necessário. E junto com o elogio de tortura e violência política
`direita, vem o silêncio dos grandes produtores de notícias oficiais. O
elogio grosseiro da ditadura e da tortura no novo tempo de transe
brasileiro era efeito colateral, necessário, da mobilização da vida
social para a garantia do novo poder de governo bem posicionado à favor
do capital, e contra o trabalho, no Brasil.

O texto, que é relevante sobre a ordem de riscos que a democracia
brasileira corre de fato, se é que se pode falar, neste estado de
cosias, em democracia, não deixou de ser publicado pelo desprestígio dos
autores -- um que já publicou muitas vezes, e foi entrevistado outras
vezes, pelo próprio jornal em questão, outro que foi Coordenador
Nacional de Combate à Tortura, da Secretaria de Direitos Humanos do
governo Dilma Rousseff. O texto não foi publicado pelo verdadeiro
desprestígio da matéria, de interesse democrático universal, mas algo
desinteressante nos novos tempos brasileiros.

\begin{center}
\versal{III}
\end{center}

No dia 3 de abril, um dia antes do julgamento do direito ao habeas
corpus de Lula no Supremo Tribunal Federal, meu amigo Aldo Zaiden,
psicanalista, trabalhador pelos direitos humanos no Brasil, foi
espancado por um grupo de extrema direita em São Paulo. Ao encontrá"-los
na padaria da esquina de sua casa, vindos de uma manifestação anti"-Lula
e ``pelo fim da corrupção'' no Brasil, Aldo lhes perguntou se eles se
manifestavam também contra a corrupção de Aécio Neves e de Michel Temer.
Foi o suficiente para o grupo partir pra cima, aos gritos de petista
canalha. Um deles usou uma arma de choque -- proibida no Brasil, pois há
risco de morte -- o que derrubou Aldo, que é um homem grande. Quando ele
caiu no chão da padaria o bando, cerca de seis pessoas, incluindo duas
mulheres, começou a chutá"-lo. Pessoas e segurança interviram, e
afastaram o grupo de selvagens políticos do tempo do psicanalista
cidadão. Uma das mulheres do grupo, colocou o celular na cara de meu
amigo, provocando"-o, e dizia: ``Esse é o petista que bate em mulher''.
No dia seguinte a sua imagem estava no instagram dela, e ele passou a
ser ameaçado de morte na rede.

O grupo que espancou Aldo Zaiden, na véspera da decisão sobre a prisão
de Lula, portava camisetas com o dizer Direita Brasil.

\chapter{O Estado não está sendo favorável à vida no Brasil}


\section{Entrevista a Amanda Massuela, Revista Cult}

\noindent\versal{AMANDA MASSUELA}: Você afirma que há uma política da inimizade instalada no Brasil. Pode
explicar esse conceito?

\medskip

\noindent\versal{TALES AB'SABER}: A política da inimizade é uma formulação do filósofo camaronês do campo
crítico ocidental do \emph{pensamento negro} Achille Mbembe para as
longas tendências políticas agressivas e excludentes, ligadas também a
ideia de uma \emph{necropolítica}, a política como soberania a respeito
da decisão sobre \emph{a morte} do outro, que durante muito tempo se
convencionou também, em um processo de metaforização de movimentos
históricos concretos de organização massiva da violência, chamar de
\emph{fascismo}. Mas também, com uma doze de anacronismo no jogo de
palavras, o fascismo, ou a necropolítica, ou a produção histórica da
inimizade universal, já presentes em toda a origem da vida capitalista a
partir dos trânsitos comerciais mundias dos séculos \versal{XVI} e \versal{XVII}, como
\emph{colonialismo}.

O Ocidente tem uma longa história de \emph{política da inimizade}, uma
história profunda, que em uma de suas facetas se confunde com a história
do colonialismo, o termo político sistêmico da emergência de todo
capitalismo mercantil moderno, desde a origem globalizado. È um campo
muito amplo de pensamento político, e de ação politica que tende á
violência direta, reduzindo o espaço entre os direitos e a decisão a
respeito da exclusão e morte do outro. O que importa como dado histórico
escandaloso é que nos últimos três anos vimos a democracia travada ou
danificada brasileira se degradar e passar a ser pautada, distorcida ou
performada, como se queira, pela violência na ação explícita das forças
que produziram o impeachment, ou o golpe, dependendo de onde se vê.

De fato, tais práticas geraram força para a produção de política, o que
em si mesmo é uma degradação da estrutura simbólica de orientação das
coisas. Não por acaso, hoje temos pessoas absolutamente fora dos
sistemas de legitimadade simbólicas públicas historicamente reconhecidos
-- o sistema mundial das universidades -- dizendo o que tem que ser
feito da educação, por exemplo, os tais Escola sem Partido. São pessoas
que simplesmente vêm de sua família, da sua religião, da sua igreja, dos
rituais e práticas mais comuns de sua classe e tentam enquadrar todo o
campo simbólico do outro, e a própria vida prática e ética. A política
da inimizade também cria inimigos ao seu espelho. É a lógica projetiva
de todo fascismo, que em nome do ato sacrificial que libera a barbárie
projeta simplesmente a bárbarie no inimigo, um fetiche negativo. O
judeu, o petista, o negro, o gay, enfim a minha diferença absoluta. São
os monstros que põe em perigo a civilização, baseado em alguma ordem de
deléiro sobre a própria superioridade, que nunca se confirma, e
autorizam qualquer tipo de violência. E essa violência, quando se torna
movimento político, produz poder e produz um movimento integrador,
identificatório, com o poder. Essas pessoas se sentem integradas por
essa violência, que simplifica muito as coisas do mundo. Do \emph{seu}
mundo. Isso as constitui. É uma política da violência que está tentando
responder a uma ordem de sofrimento que não consegue ser pensada de
outro modo que não esse: elegendo o inimigo para o sacrifício. Triste,
violento e estúpido.

\medskip

\noindent\versal{AM}: Quem é esse inimigo, hoje, na política brasileira?

\medskip

\noindent\versal{TA}: Todo o movimento de tomada do poder foi feito fixando artificialmente o
campo petista como responsável pela totalidade da crise do sistema
político brasileiro, o que é uma falsificação. Se estamos falando da
degradação e da corrupção estrutural do sistema político brasileiro -- e
era essa o motivo político real da crise aparelhada totalmente pela
direita -- então o responsável por isso é justamente o sistema político
brasileiro. Mas se tornou o \versal{PT} porque era justamente esse o modo
estratégico de produzir política e de colocar um certo grupo da política
nacional no poder, um grupo que claramente retoma a tradição autoritária
brasileira, além de seu alinhamento cego com qualquer grande interesse
econômico de modo sumariamente antissocial -- e que, em uma democracia
de massas só conseguiria mesmo governar por meio de uma ruptura
controlada como foi o impeachment.

É um grupo antiético, inteiramente envolvido em corrupção -- Geddel
Viana de Lima, ministro de Temer preso por manter um apartamento cheio
de malas e caixas de dinheiro vivo, apenas um entre todos, é o retrato
acabado do tipo -- que usa o poder para se blindar, que cede legislação
e direitos a interesses particulares diretos. É a impressionante
representação atual de uma velha elite predadora que se formou em uma
relação de violência extrema com o povo, numa posição original de senhor
de escravo, um dado histórico e subjetivo particularmente brasileiro. É
o passado sempre incrustrado no processo histórico brasileiros, que
perfura a democracia, e emerge glorioso e terrível no avançado da hora
do capitalismo contemporâneo. E o senhor de escravo só necessita dar um
prato de comida para reproduzir a força mínima, mínima, mínima de
trabalho, ele não precisa garantir a vida, e a sociedade, em nenhum
outro sentido.

É isso que está no poder hoje. Não se trata mais de um projeto, não há
nenhum espaço para a política que importa, de constituição desejada de
civilização. Esse grupo, que é uma facção das classes dominantes, está
numa relação de inimizade com a nação, poderíamos dizer, em que, dado o
ritmo de blitz krieg da tomada do poder de assalto e no grito, a nação
tem que ceder tudo de modo imediato. É uma guinada neoliberal bárbara,
um real assalto aos interesses e direitos que configuram o espaço
público. O Estado não está sendo favorável à vida no Brasil.

E tudo isso sobre a égide de uma figura de democracia, o que permite que
tal violência se configure em um lugar de legitimidade de Estado, que é
a legitimidade de operar a violência. Esse é o paradoxo da situação
institucional da dita democracia formal, e o fundo político autoritário
que se apropriou dela, que produz, desde o poder, um campo de violência,
fascismo e desrespeito público no Brasil, mas está tudo ``normal''. Esse
é o sentido mais duro, e profundo, da politica da inimizade entre nós.

A democracia se tornou uma fachada para a produção de uma violência
antissocial, tida por legítima. Tem alguma coisa muito errada. Se a
democracia serve para isso, que democracia é essa?

\medskip

\noindent\versal{AM}: Quais são os sintomas sociais dessa violência?

\medskip

\noindent\versal{TA}: São muitos. Um dois principais é a perda dos limites da lei para o
Estado policial. O assassinato frio de um carroceiro que pedia comida em
um restaurante, em São Paulo, por uma polícia que acha que pode dispor
do povo brasileiro como bem entende, é um emblema disso. Foi necessária
enérgica reação, pois o fascismo de um Estado condescendente com
práticas de extermínio social banalizadas estava ameaçando totalizar a
vida pública: houve missa protesto na praça da Sé, com apresença de
todos os bispos de São Paulo, como quando do assassinato de Herzog pela
ditadura civil"-militar de 1964/1984...

Também é notória a repressão sistemática, em sistema de controle social
forte e sem limites legais para o uso de armas de efeito moral, que
lesam mesmo corpo e indivíduo, a toda e qualquer manifestação contra o
governo, apanhado em explícitos atos de corrupção com fitas e malas de
dinheiro para lá e para cá expostos da tv. Mas o que valia como espaço
democrático de política de manifestação cívica contra o governo de
esquerda, nunca valeu para o governo repressor e muito agressivo de
direita que emergiu do golpe. Há uma radicalização generalizada da
violência policial, que tende a tratar a sociedade como inimiga, e uma
radicalização do discurso geral de ataque à legislação universal e
internacional dos direitos humanos. Isso é motivo e resultado da tomada
do espaço público e político por essa direita autoritária. Ela insiste
no fato social extremo, brasileiro, de que a policia tem o direito de
decidir a vida e a morte de pessoas por fora da lei. Esses caras estão
literalmente \emph{trocando a lei pela polícia}. Isso é muito grave.

Além disso, o que se verifica no trabalho das clínicas abertas e sociais
de psicanálise -- movimento social do qual participo -- é o avanço da
perda de energia da vida e para a vida, que vai paralisando pessoas em
estruturas de sintomas ideológicos que não dão mais conta: são os 12,5
de desempregados, uma geração inteira de vinte a trinta anos que sabe
que a vida está passando e que não terão nada de significativo para
fazer. É uma clara depressão de valor social. Uma ruína humana, uma
doença social, que também tem o polo de escárnio e o choque traumático
do absurdo diante de um Estado tomado e que radicaliza os interesses
particulares e não dá respostas sociais minimamente acietáveis.

Sua relação com o país é irresponsável e antissocial, e isso é uma real
política da inimizade. Para que isso se sustente são necessárias ações
de força para congelar as tenções sociais. O que acontece, gerando mais
violência. A máquina não para, e vai se tornando infernal. Ao mesmo
tempo, onde não há tensão social existe um arruinamento da vida. Nesse
momento é isso que começa a aparecer. Hoje não temos nenhum pacto social
acordado, que seja um projeto de mediação para a construção de alguma
coisa, porque, do ponto de vista da direita, não há nada a ser
construído. Só reprimido. A vida pública e os cidadão apenas atrapalham
os negócios no Brasil.

\chapter{Neofascismo e o cinema urgente brasileiro }

Existe uma afirmativa ação político cultural neofascista no Brasil de
hoje. Passado um ano e alguns meses da derrubada simbolicamente violenta
do último governo petista eleito -- com o grande choque sobre uma ordem
democrática que produziu 54 milhões de votos para um projeto de país que
foi abortado -- todos os dias precisamos lidar com a real desagregação
democrática pela tendência forte de pessoas, que tem sua subjetividade
referida a grupos ideológicos organizados de direita, em reduzir a
linguagem, o direito e a razão pública da história ao seu sistema de
desejos imediatos e à passagem à ação de força direta contra os direitos
e o corpo do outro que não reitere sua própria paixão. O que em um
passado recente mal disfarçava o ódio e o desejo corpóreo de violência
sobre alguém no ato limite de calar e atordoar \emph{o inimigo} batendo
panelas, hoje ultrapassou esta barreira, e prossegue, como compulsão à
repetição indomável, tentando cada vez mais, cada dia mais, alcançar o
corpo de direitos e o corpo real dos inimigos maléficos, sempre
inventados, para animar esse sistema psíco"-político. Do comunista
imaginado de 2014/2015 chegamos aos artistas pedófilos, e às filósofas
críticas bruxas do presente, sonhados para serem queimados.

O Brasil regrediu de fato, não apenas ao seu grosseiro passado
ideologicamente orientado de autoritarismo e de desprezo antipopular, o
fundo mítico restaurador de nossas clivagens sociais originais para onde
nossos neofascistas querem voltar ilusoriamente, mas talvez até mesmo
para formas humanas ainda muito anteriores à sua própria história. Freud
já relacionava a constituição de grupos que sustentavam ilusões comuns
fascistas à ideia de uma forma hiper"-arcaica, a congregação grupal
original, anti"-intelectual de todo modo, daquilo que ele chamou de
\emph{horda primitiva}. A medida desta avaliação, deste conceito
psicanalítico sobre a regressão coletiva, é a real dissolução do
entendimento comum partilhado, do compromisso e do pacto com a história
acordada, a vida intelectual, e a ocupação e transformação do espaço
psíquico dos homens do grupo ou das massas pelo próprio direito à
violência, uma real política subjetiva da exceção, paranoia e inimizade.
Deste modo, como vemos acontecer no Brasil hoje, o estado de exceção
mais amplo é duplicado em uma política psíquica cotidiana e comum da
exceção. Um estado psíquico de exceção.

Esta regressão psíquica ao real grupo neofascista blinda mesmo os seus
ideais, sem checagem histórica ou racional, criando território nítido e
clivado entre o bem e o mal e liberando a violência, verbal, moral e
real, contra os inimigos inventados, criados por esta própria dinâmica
subjetiva e grupal. Ativa"-se a forma paranoica como a forma fundamental
da política, núcleo original de invenção do poder. A política tende a
regredir à forma da milícia autoritária, pronta para descrever a
realidade como uma guerra imaginária, que supõem real e que libera e
legitima toda violência. Esta real regressão subjetiva acompanhou desde
sempre bem de perto o processo político mais amplo institucional e foi
força importante para o projeto de tomada do governo, e de ataque ao
Estado brasileiro, por grandes interesses econômicos locais e globais.

Se o golpe parlamentar"-midiático de agosto de 2016, construído simbólica
e politicamente ao longo de todo o ano de 2015, ocorreu através do
realinhamento politico de grandes interesses industriais, financeiros,
do agronegócio e de amplos setores das classes dominantes brasileiras --
o verdadeiro poder real, em conjunto chamado tradicionalmente de Capital
-- projetando a tomada do governo para a desestruturação do pacto
civilizatório acordado na Constituição de 2008, para o aumento direto de
sua força e de sua riqueza sobre toda forma de trabalho existente no
Brasil, este movimento real do poder sempre foi bem acompanhado daquele
estado de exceção psíquica atuando nas ruas do país. Um estado excitado
que ocupou o significante vazio das manifestações democráticas contra um
governo tido por corrupto com o seu conteúdo claramente violento,
anti"-histórico. O Estado de exceção projetado pelo grande poder na
própria ordem pública era apoiado e sustentado pelo estado de exceção
psíquica daqueles que em 2015 saiam às ruas apaixonados, e liberados
para a ação, diante da \emph{iminente revolução comunista bolivariana}
-- falsificação grotesca -- que de fato alucinavam ocorrer no Brasil,
alucinose que justificava inteiramente todas as próprias barbaridades
simbólicas. Diante da monstruosidade projetada no inimigo político,
todas as próprias ações bárbaras se justificavam plenamente. Eram os
radicais apaixonados, antissociais e violentos \emph{anticomunistas do
nada} brasileiros, real força popular a favor do real ataque neoliberal
ao Estado, a força verdadeira do poder, imensos interesses econômicos no
Brasil.

Um dos paradoxos de todo este sistema ideológico simbólico, que deságua
no neofascismo medíocre atual, é a destruição do sentido histórico do
governo democrático petista, que dinamizou e deu grande passo no
desenvolvimento capitalista brasileiro ao incentivar o funcionamento
limite do mercado interno de consumo e de emprego, peça macroeconômica
real que nenhum país desenvolvido do mundo pode desprezar, mas o
capitalismo rentista brasileiro pode. Ou seja, um governo que acentuou o
processo da dinâmica capitalista e de consumo brasileira. Este processo
real, que durante o período Lula, por exemplo, fez o país crescer seis
anos acima do \versal{PIB} mundial, em direção ao pleno emprego, foi tratado,
alucinatoriamente, com a referência bem fixada ao passado autoritário da
ditadura e da guerra fria, como um real governo comunista. A desfaçatez
mentirosa, ``versão alternativa'' da história, mundana, sustentada de
fato como vulgar briga de rua, não encontrava limite em seu desabrido
desejo de poder, e sua liberdade de mentir, legitimada pelo poder
simbólico do todo, da ordem capitalista em busca de exclusão brasileira.

Neste caldo de cultura violenta, ativa e nova, da nova direita
neofascista agindo realmente na política, e do golpe geral midiático
parlamentar, a tradição reflexiva e crítica do cinema documentário
brasileiro encontrou, como não poderia deixar de ser, matéria especial e
forte para redefinir as condições históricas do país, cuja face
conservadora ganhou nova visibilidade. Assistindo a ``Intervenção, amor
não quer dizer grande coisa'', filme sobre a ação delirante e radical da
extrema direita na internet, base energética e qualitativa de toda
tendência regressiva da nova direita brasileira, quando da sua polêmica
exibição no festival de Brasília deste ano, o poeta Chico Alvim colocou
a questão fundamental, que o limitado processo de nossa democracia
anterior tentou sempre recalcar: ``O Brasil é isso?''. Sim, o filme
deixa claro, o Brasil é isso. E também, ao mesmo tempo, mais que isso.
``Operação da lei e da ordem'', de Lúcia Murat e Miguel Ramos, demonstra
a criminalização e a redução simbólica dos movimentos sociais críticos e
à esquerda do país sempre ao mesmo ponto repetitivo, de ilegítimos e
reais vândalos, pela grande mídia televisiva oficial. A máquina geral da
comunicação participa assim ativamente da produção da mentalidade
antidemocrática, que vai alimentar em casa o preguiçoso fascista nosso
de cada dia. Enquanto isso, jovens corajosos, informados e que lutam
pela própria liberdade subjetiva e erótica, invadem as próprias escolas
para defendê"-las do desejo de liquidação neoliberal por um governo
conservador, e chegam a vencer a poderosa máquina ideológica de
aceitação de tudo do governo tucano do Estado de São Paulo. E vemos esta
história inteiramente contada por seus personagens concretos, os jovens
que a realizaram, filmando desde as suas próprias ocupações, na nova
obra prima do documentário político ``Escolas em luta'', de Eduardo
Consonni, Rodrigo Marques e Tiago Tambelli. Em conjunto os filmes
radiografam em profundidade o sistema ideológico e subjetivo do poder no
Brasil, os movimentos simbólicos concretos que também são o golpe
político vivido pelo país, e as verdadeiras forças vivas de reação
histórica presentes no mesmo país.

O Brasil está de fato conflagrado. A direita nos choca a cada dia com a
sua política da violência e de insolência antidemocrática, contra os
direitos humanos e agora contra a ideia mesma da crítica, ao mesmo tempo
em que a força e a velocidade avassaladora do golpe, das classes
dominantes brasileiras contra o trabalho no Brasil, consuma a derrota de
um projeto de democracia e sociedade longamente cultivado. Neste quadro
o cinema brasileiro responde com energia de urgência e inteligência que
se remete aos seus bons momentos de outros tempos de nosso grande mal.

\chapter{O tempo é mal e o país partido: estilhaços do Brasil}

\begin{flushright}
\emph{26 de setembro de 2016}
\end{flushright}

O ministro da justiça skin head anunciou com antecedência a prisão de um
adversário político, dias antes das eleições municipais. A polícia do
governador, padrinho político do skin head, se torna ativa na provocação
de violência entre as manifestações políticas contra o governo. Estamos
de volta ao tempo das polícias políticas? A direita quer aparelhar a
democracia com violência policial?

\begin{flushright}
\emph{5 de novembro}
\end{flushright}

O mundo está em convulsão e há questionamento por todos os lados: de
gênero, ambientais, raciais, sobre a outridade do outro e as modalidades
afetivas do poder. No entanto, para muitos, basta a reprodução geral do
sistema das mercadorias para tudo encontrar o seu lugar.

\begin{flushright}
\emph{24 de outubro}
\end{flushright}

O pau está comendo no Brasil da direita que perdeu a vergonha.
Autoritários, a"-históricos e violentos. Atacam o direito de
manifestação, jogam bombas em adolescentes e querem passar projetos de
controle da política econômica do país por vinte anos. Negam os direitos
humanos, atacam os direitos sociais e desconhecem os direitos das
minorias. Rompem todos os pactos sociais acordados na constituição de
1988. E os ditadores eram os petistas eleitos.

\begin{flushright}
\emph{2 de novembro}
\end{flushright}

A polícia prende atores, juízes autorizam tortura, policiais apontam
fuzis para adolescentes e ônibus inteiros são cheios de estudantes
presos. Determina"-se agora os limites dos próximos cinco governos.
Intervém"-se na estrutura da educação por decreto. Se corta o
investimento em ciência, se aumenta o gasto com a política e com o
judiciário. E com a polícia. Prepara"-se uma anistia ao próprio sistema
da própria corrupção. O governo classista de direita se torna a cada dia
mais ilegítimo. Se não foi um golpe o que ocorreu no Brasil o semblante
ditatorial cada dia mais acentuado demonstra o verdadeiro caráter do
movimento da nova direita: ruptura com os limites da lei e da política,
ação violenta antissocial, criminalização das ações críticas contra o
estado das coisas. Ataque real à democracia. A direita golpista não
consegue atuar sem imprimir um estado geral de arbítrio, transformando o
espaço público em um espaço violento, a política em polícia. Uma
vergonha continuada o mundo do governo Temer e seus ridículos ditadores
de massa e de elite, em busca de violência direta. Um desejo, e uma
estética, de ditadura se disfarça em uma estrutura institucional
democrática que permite e alimenta as ações violentas do novo velho
poder no Brasil.

\begin{flushright}
\emph{15 de novembro}
\end{flushright}

Dois movimentos sociais confrontam a regressão geral conservadora da
nova direita: os jovens adolescentes da escola pública que aprenderam a
exigir direitos, porque, por vezes, livres do trabalho e atentos aos
interesses recusados da própria classe, e os movimentos por direitos de
reconhecimento e política estética e existencial lbgtq. Em algum lugar,
que não é pequeno, estes mundos se encontram. Além de se encontrarem nos
céus das possibilidades utópicas humanas, que de um modo ou de outro, e
hoje deste modo, também sempre descem à terra.

\begin{flushright}
\emph{18 de novembro}
\end{flushright}

A violência é uma defesa rápida e mágica da dor, a verdade, que oculta.
E também é uma prisão.

\begin{flushright}
\emph{23 de novembro}
\end{flushright}

Afeto micro, política macro: afeto macro, política micro.

\begin{flushright}
\emph{2 de dezembro}
\end{flushright}

Crise mundial, crise central, crise nacional. Crise do mundo do emprego
e da economia. Crise da política degradada. Crise ambiental. Crise da
cidade. Crise do caráter. Crise do sonho. Crise do amor e da vida. Quem
é o sujeito desta grande liquidação?

\begin{flushright}
\emph{4 de janeiro de 2017}
\end{flushright}

Há um estranho clima de fim do mundo no ar. Certamente o Brasil está
liquidado por todo tipo de erro, político, econômico, ético, estético.
Como era previsível a histeria regressiva à direita não produziria
civilização digna. A direita brasileira é apenas condescendente ao
máximo com a violência de toda ordem que seus esquemas sociais
simplistas produzem. Definitivamente estão fora do tempo, mas fazendo o
tempo em falso, com o seu próprio deslocamento arcaico. Os horizontes
mundiais de aposta autoritária para repor o capitalismo em rota também
não convidam. Teremos que suportar o mal geral, que não é pequeno, em
nosso próprio corpo socialmente nu, e em nosso próprio espírito
constantemente impactado e reduzido pela cultura geral do choque e da
estupidez satisfeita. Vamos nos aproximar como heróis do nada e no nada,
e nos dedicar a algo melhor, enquanto o céu cai sobre todos nós.

\begin{flushright}
\emph{1 de fevereiro de 2017}
\end{flushright}

Estamos em meio a uma guerra degradante. Cujo sentido é nos degradar.
Uma guerra em que todos sabemos da perversão do poder e do processo da
política no Brasil, todos sabemos da ruína, da mentira, da pura
exploração, da manipulação, do desrespeito e da irresponsabilidade
generalizada do poder. Tudo está péssimo, violento e degradado. Tudo é
uma grande mentira, uma máquina de violência simbólica e real, nas ruas,
plenamente visível. Precisamos de imensos cuidados, de imensos gestos de
amor e respeito. Precisamos muito, demais, nos cuidar, em meio a tanta
violência e estupidez. Precisamos muito, demais, de amor, força e
inteligência.

\begin{flushright}
\emph{8 de fevereiro de 2017}
\end{flushright}

Esta classe de homens cuja relação com o país é formal e corrupta, pura
má fé, tem o direito de nos impingir a sua incompetência social,
regressão simbólica e violência protetora dos privilégios de Estado e
dinheiro? Precisamos ir pra praça Tahrir.

\begin{flushright}
\emph{12 de fevereiro}
\end{flushright}

Tive na paulista na manifestação contra o Alexandre de Moraes no \versal{STF}.
Não tinha mais de trinta pessoas. Onde estávamos todos nós?

\begin{flushright}
\emph{2 de março}
\end{flushright}

Há muita violência disponível nos próprios sujeitos, muita destruição à
espreita, muita guerra desejada entre os próximos que necessitam ser
reconhecidas e trabalhadas, cuidadas por um ponto de vista que permita a
circulação do amor, em um trabalho que, se não antecede, também não cede
ao trabalho que deve ser feito com o mal social mais geral. Maltratamos
nossos amigos, maltratamos nossos irmãos e maltratamos nossos amores,
mas nos iludimos sermos melhores quando desejamos um mundo melhor para
todos. Vamos importar nossa violência desconhecida para o mundo mesmo
que sonhamos como melhor, se não soubermos suspende"-la, reconhecê"-la, e
transmutá"-la nas relações imediatas com o mais próximo. E também
transformá"-la diante do ainda mais próximo de todos, que cai sobre nosso
terror e submissão, e que necessitamos reaprender a sonhar: nós mesmos.

\begin{flushright}
\emph{6 de abril}
\end{flushright}

O avançado da hora do golpe de força da direita no Brasil estabeleceu um
país muito cínico e sádico, que convive com o extermínio e a corrupção
generalizadas, desde que de direita, com a desfaçatez, o desejo de ação
direta pelo poder e o direito satisfeito de violência. Destruir o
processo institucional e a leitura da história no grito trás o grito,
imediato, autoritário e onipotente para o primeiríssimo plano da vida
política pública. Esta foi a transformação da política que a direita
operou no Brasil. A loucura maior desta violência é que ela sequer
precisa apresentar resultados, a culpa de tudo será sempre do comunista
inexistente, produto do próprio delírio sacrificial primitivo. Que
bárbaros ignorantes defensores de ditadura, de extermínio de índios e de
destruição de direitos falem em clubes de elite e da classe média alta,
recebendo aplausos dos nossos novos sádicos públicos, revela a profunda
estrutura psicopolítica autoritária e burra brasileira, que a
democratização apenas encobriu com uma cortina de fumaça e a esquerda no
poder não transformou de nenhum modo, mas, com suas opções políticas e
de pleno poder ao consumo também jogou água neste moinho.

\begin{flushright}
\emph{8 de maio}
\end{flushright}

Transformação de importantes energias críticas e disponibilidade para a
vida e a luta em briga direta fácil e imaginária, eu contra ele, minha
cor contra a sua, meu sexo contra o seu, meu gênero contra o seu, minha
linguagem contra a sua... um ganho do poder. Pode"-se mimetizar assim o
modo de ser das formas estéticas sexuais burguesas e sua arrogância,
apenas querendo se impor, como poder pessoal sobre todas as demais
formas existentes. Pouca diferenciação do modo de ser mais íntimo do
poder. As violências sofridas pelas pessoas representam questões e
lugares políticos, que podem se expressar como dores absolutamente
pessoais e de reparação narcísica. É preciso cuidar da dor da pessoa,
mas é preciso lembrar a congregação mais ampla de todos. Onde a
perspectiva crítica de toda violência entre os homens se unifica? Pensar
com mais afinco e rigor sobre esses pontos.

\begin{flushright}
\emph{19 de maio}
\end{flushright}

No Brasil um governo não precisa nem ser nem parecer ser honesto. Desde
que desmonte as leis trabalhistas, garrote o país ao mercado, despreze
os direitos sociais e reprima com eficácia. Então as malas de dinheiro
dos homens do presidente podem voar daqui pra lá e de lá pra cá, mas o
governo está garantido. Foi assim com \versal{FHC}, que comprou com dinheiro a
própria reeleição, é assim com Temer, de quem todo o governo são
vendedores baratos do Estado. Essa é a real moralidade pública de uma
elite afeita a golpes, ditaduras e escravidão. Vergonha da falsa
democracia e do Estado seletivo de direito. Em qualquer lugar do mundo o
governo se tornaria ilegítimo com um presidente flagrado manipulando a
justiça e vendendo benefícios. Mas o mundo não é aqui.

\begin{flushright}
\emph{25 de maio}
\end{flushright}

Contra a estupidez, os cuidados. Contra o desprezo, o compromisso.
Contra a grosseria, a delicadeza. Contra a violência, amor e
inteligência. Viva o Hotel Laide.

\begin{flushright}
\emph{28 de maio}
\end{flushright}

Há algo de verdadeiro escárnio na ordem do poder hoje no Brasil. Uma
classe de senhores contra tudo e contra todos. Animados assassinos de
pobres, grotescos vampiros da classe trabalhadora e inimigos do espaço
público, o mal está à luz do dia hoje no Brasil, sem nenhum pudor e
nítido como poucas vezes foi. Nossa democracia profundamente danificada
praticamente não se diferencia da violência própria das ditaduras
íntimas brasileiras. O Brasil vai se tornando novamente a pesada cruz
dos brasileiros, sempre sacrificados no altar da riqueza antissocial dos
imbecis ricos dessa espécie de terra. Bem duro, um tempo obscuro
disfarçado de estado de direito.

\begin{flushright}
\emph{6 de julho}
\end{flushright}

Política como modo de enriquecimento de homens duvidosos, incapazes de
produzir em qualquer esfera, ética, estética ou mesmo técnica. Política
como venda da política para expansão dos interesses do capital sobre o
Estado. Pacto pela riqueza e pelo poder como luta pelo poder e
servilismo barato do dinheiro. Homens do poder que podem extorquir, que
podem encomendar mortes e silêncios, que podem falar de reis da Suécia
quando na Dinamarca e de Shakespeare como famoso filósofo alemão. Há
algo de podre no reino... de fato, tudo. É preciso dedicar cada fibra do
próprio corpo e espírito à própria degradação. É preciso ter fé absoluta
que nada vale nada, e que tudo e qualquer coisa vale a venda de tudo, da
política, do país e da vida do outro. Fé absoluta que não há sentido nem
porque lutar, que não seja o jogo do poder de cobrar caro pela venda da
própria posição. Nossa vida sobre nenhum aspecto deve ser determinada
por isso.

\begin{flushright}
\emph{13 de julho}
\end{flushright}

Um dia passei por um carrinho de um catador que era muito lindo. Flores,
máscaras, brinquedos. Fiquei impressionado demais, e uma quadra depois
resolvi voltar atrás e dizer ao moço lindo que o carinho dele estava
muito bonito. Ele ficou satisfeito. Não aceitou o meu dinheiro, porque
"não estava precisando". Ontem ele foi assassinado friamente pela
polícia. Pedia comida em um restaurante.

\begin{flushright}
\emph{19 de julho}
\end{flushright}

Missa por um brasileiro morto pelo Brasil hoje ao meio dia na Catedral
da Sé. De Herzog ao Negão a história continua uma máquina de sacrifícios
e mentira. Um outro sacrificado um dia já associou a ideia de progresso,
que é o progresso do lucro, com o aumento constante da catástrofe
humana. A história avança como necropolítica, e há uma tendência
universal de todos devirem o "negro". Além das vidas desperdiçadas,
impressiona muito, muito, o sadismo e a normalidade satisfeita de uma
certa humanidade que convive e passa cotidianamente sobre os cadáveres
que alimentam a própria vida. Afinal só podemos existir assim? Matando
constantemente alguém, na fantasia, ou na Real? Desejo de paz, e de que
a pesquisa humana se inverta da destruição para a vida, da vida sobre o
trabalho alienado e destruição do valor do vivo para a pesquisa das
modalidades e camadas mais ricas e complexas do amor. A missa, ato de
amor por todos nós, do Ricardo será ao meio dia, na Sé.

\begin{flushright}
\emph{17 de agosto}
\end{flushright}

Nazismo de esquerda: o gozo da mentira histórica, o desprezo pela
diferença, desejo degradado como poder, anti"-intelectualismo cínico,
insolência conservadora espancando as balizas históricas, tomada da vida
simbólica por verdadeiros calhordas. A boçalidade do mal. Novilíngua.
Como combater a irracionalidade que desrespeita tudo, a não ser o ato de
poder da própria e única fala, de fato ridícula? Nazismo de esquerda é a
fala do grau zero de subjetivação política: são os nazistas, que queimam
o Reichstag, e dizem que foi a esquerda.

\begin{flushright}
\emph{18 de agosto}
\end{flushright}

A boçalidade do mal.

\begin{flushright}
\emph{5 de setembro}
\end{flushright}

A estrutura e a vida imaginaria do espaço público se alteraram nos
últimos 20 anos. Crise do partido, o continente ordenador, popular e
democratizante. Perda da sua referência, do seu norte simbólico.
Reorganização da direita na internet e, depois de 2013, nas ruas. Recuo
do campo social organizado e desejante, das políticas graduais de
Estado, para as redes sociais. Governo sob assalto direto do capital em
todas as suas formas e modos. Tempestade simbólica de pósverdade, ações
ilegais e gestos de violência consentidos, como cortina de fumaça
pública para o livre movimento do Capital no Estado. Espírito
progressista em recesso imaginário na internet, em estado de gozo
especular, vendo tudo ser desmontado e aguardando 'democraticamente' a
derrocada da política neoliberal, como se se tratasse de democracia.
Espaço público confundido com narcisismo e espetáculo, crise da esquerda
e tomada do poder por violência simbólica e poder material concreto.
Democracia como simulacro e ordem real do capital sobre o Estado. Quem
somos nós?

\begin{flushright}
\emph{11 de setembro}
\end{flushright}

Liberais autoritários brasileiros: censura e intolerância modernas: tudo
pelo mercado com controle moral da cultura: elogio do dinheiro com beija
mão de governo ilegítimo: perseguição de direitos contra o comunismo
inexistente: fascistas de consumo agregados de qualquer poder
antipopular: senhor de escravo liberal brasileiro

\begin{flushright}
\emph{12 de setembro}
\end{flushright}

Um grupo de ignorantes radicais, mentirosos compulsivos e promotores de
violência contra direitos básicos pautando a cultura no Brasil.
Capitalismo e fascismo de mãos dadas novamente. Seus ídolos, Olavão de
Carvalho, Azevedo, Pondé, e outros neoliberais para quem tudo é
permitido para a instalação do império do mercado, que formaram por anos
a pratica política cultural dessa gente, devem estar muito satisfeitos.
E falsos liberais que saíram às ruas com fascistas para destruir uma
eleição também são responsáveis pelo avanço do mal de seus aliados
objetivos pelo poder.

\begin{flushright}
\emph{29 de setembro}
\end{flushright}

O que se observa é a cultura da ignorância, que se expressa como gesto
de força -- censura, recusa do outro, bater panela -- agindo diretamente,
como lhe é próprio, sobre os espaços de problematização social,
diferença, respeito democrático e arte. O golpe de força à direita foi
baseado na recusa do trabalho de entendimento da história, e trouxe a
ignorância e a estupidez para o primeiro plano da cultura. Pessoas que
simplesmente abrem mão de pensar, de mediar sua própria existência pelo
outro e de aprender com a experiência. Banalidade do mal era isso,
segundo Hannah Arendt. No Brasil ela também é uma grotesca boçalidade do
mal. Mal e burrice, é o resultado do progresso da democracia brasileira?
Assassinato e ignorância resplandecente é o nosso destino? Isso é o
Brasil, me perguntava Chico Alvim, depois de ver " Intervenção -- Amor
não quer dizer grande coisa ", Com a palavra todos os elegantes
modernos, liberais, neoliberais, consumidores universais, ricos e cultos
dessa espécie de nação, que insiste em ser burra, brega e violenta.

\begin{flushright}
\emph{1 de outubro}
\end{flushright}

Pessoalmente vejo a performance que foi atacada no \versal{MAM} como inofensiva
sobre todos os aspectos. Anódina. Com as pernas nuas Cris Bierrenbach
fez coisa muitíssimo mais interessante no encontro de performance da
Galeria Vermelho. O que move a direita destemperada e desaforada não é
nenhum conteúdo de verdade. Como todos sabemos estes são os homens que
consideram a verdade totalmente ultrapassada: homens e mulheres que
vivem em Estado de pos verdade. O que eles atacam é o semblante em bloco
dos conceitos e mundos que desconhecem e lhes exige trabalho e humilha.
Arte contemporânea. Política do Sexual. Sistema institucional da
cultura. Arte, sexo e cultura. São estes os reais alvos da ação
violenta, que é verdadeira política da ignorância e da repressão, no
sentido mais amplo possível do termo. É evidente que este mesmo povo, já
viciado em ação de choque para constranger direitos, é o que deseja
ditadura no país, que eles próprios desorganizaram. E sua política é
identitária e de ação direta. Ela não tem razão, verdade nem argumentos.
Transmite diretamente, de modo mimético, o convite a identificação com o
próprio gozo regressivo, niilista e autoritário do próprio grupo.
Psicopolitica real, contra todo trabalho desejante e múltiplo de
civilização que implique trabalho da linguagem.

\begin{flushright}
\emph{3 de outubro}
\end{flushright}

Eles precisam e buscam uma nova ordem de choque, convocação de ódio, ato
de produção de grupo para ganho político, diante do real fracasso do
governo golpista que impuseram ao país. Na ausência de comunistas,
atacam arte e sexualidade. Nova rodada de política da violência e de
sacrifício simbólico do inimigo unificador. Política da inimizade.
Mimética, baseada em grandes idealizações e na redução do outro ao lugar
de inimigo absoluto, frente ao qual o gozo do extermínio justificado,
mesmo que simbólico, permite o desrecalque prazeroso do direito à
porrada. E do cinismo, de saber que sua estratégia vazia de verdade se
transmite automaticamente, por adesão do indivíduo ao grupo, como
formação psíquica regredida que é.

\begin{flushright}
\emph{3 de outubro}
\end{flushright}

E mais ainda. E se os que estão salvando a criança da pedofilia do
artista estiverem fazendo de fato um mal a ela? Expondo"-a.
Sexualizando"-a. Explorando"-a. Gozando com a sua disponibilidade para
todo tipo de uso de sua vida. ... uma massa grosseira despedaçando uma
criança simbolicamente pelo seu jogo e prazer ideológico descontrolado?
... os que a defendem também gozam um próprio gozo... quem disse que
inimigos da arte, de museus, e da sexualidade dos outros seriam amigos
das crianças dos outros?.. De onde vem o mal?

\begin{flushright}
\emph{5 de outubro}
\end{flushright}

O \versal{MBL} despreza os direitos humanos, apoia golpes, deseja fim de
direitos, constrange pessoas, mas... gosta de arte bonita

\begin{flushright}
\emph{5 de outubro}
\end{flushright}

70 000 gritam likes na internet para fechar um museu, por que não
gostaram de uma obra de arte... o que mais vão gritar para fechar?

\begin{flushright}
\emph{16 de outubro}
\end{flushright}

Na impossibilidade de discutir o avanço dos direitos universais à vida,
o que há muito está disponível à conquista técnica da humanidade,
estamos discutindo posições, narrativas, interesses, direitos do poder,
controle às artes, direito à violência, terra plana, bolsonaro,
alexandre frota... O barramento à vida e a fragmentação da cultura das
disputas particulares, multiplicação das cisões de um mundo
estruturalmente cindido, sociedade de classes, nos leva para a Babel
regressiva, que o primitivo fascismo explora. O meu desejo: um
redespertar da cultura crítica contemporânea para o direito geral à
liberdade e à vida, universal, incluindo aí o ambiente e as vidas não
humanas. Oxalá!

\begin{flushright}
\emph{20 de outubro}
\end{flushright}

Temer, sua casta de senhores mafiosos, facilitando a escravidão. De
vomitar.

\begin{flushright}
\emph{20 de outubro}
\end{flushright}

... nenhuma nação.

\begin{flushright}
\emph{21 de outubro}
\end{flushright}

Modos automáticos de reprodução da violência. O momento da técnica
permite uma performance direta ``signo, imagem violência'' numa cultura
geral do choque, tanto quanto do risco. A lógica do mercado, de fundo e
primeira, fragmenta, coisifica e espetaculariza toda a dinâmica social,
ao mesmo tempo que a força e exige, como gesto compulsivo de existir aí.
Em rede os fascistas pescam. Excelente texto da Silvia Viana.

\begin{flushright}
\emph{24 de outubro}
\end{flushright}

Teatro Oficina destombado pelo Condephat:

``Foi muito ruim a reunião. Toda família está arrrazada. Eles vão fazer
as torres destruir a janela árvore. O teatro ficará espremido e com toda
certeza sofrerá muitas coisas. É muito injusto. O teatro vai entrar com
recursos precisa ter muita divulgação disso ele está precisando de muito
apoio. Precisamos fazer alguma coisa. Me ajudem. É terrível. Ninguém
consegue dormir direito. O teatro fora do Brasil foi premiado e aqui
querem acabar com ele. Vamos inventar qualquer coisa para ajudar.
Precisamos de ajuda de todos.''

\begin{flushright}
\emph{28 de outubro}
\end{flushright}

Comissário nazista sobre arte, 1937: ``Vivemos a loucura, do
descaramento, da incompetência e da degeneração''.

\begin{flushright}
\emph{30 de outubro}
\end{flushright}

As demandas setoriais, que cindem o campo geral e a ideia da
universalidade da classe, e que projetam o dado particular na história -
a experiência e a história negra, o lugar da mulher, as vidas \versal{LBGTQ}, o
terror dos refugiados e dos palestinos -- em múltiplos vetores críticos
históricos, que ainda podem articular posições de intersecção
particulares em relação a cada sistema de sentido, e que pedem reparação
para a dimensão única do seu percurso, podem ser vistos como movimentos
que tem duas dimensões frente o todo integrado do poder, e frente a
ideia de uma razão crítica universal -- hegeliana, marxista,
frankfurtiana -- : como aprofundamento da consciência histórica das
violências, em um plano de democracia projetada e suposta de fundo, e
como um possível ceder da crítica universal à gestão da vida pela venda
geral de força de trabalho no mercado e subjetivação para o consumo. O
problema político estaria no fato de que a demanda de uma justiça única
para uma posição de violência histórica real particular, pressupõe
estabilidade democrática para a leitura da diferenciação dos casos, o
que em democracias liberais maduras pode ser o caso, mas, no ponto que
estranho, talvez anule a necessidade de consciência pública e ação
política contra a real cultura do poder capitalista: a da subjetivação
para o mercado. Cada um dos críticos particulares pode remeter sua
crítica ao pano de fundo geral do mercado universal, dado como
transparente e quase natural, de modo a produzir novas posições de poder
dentro dele, e não para além dele. Por isso, importantes pensadores
críticos das condições de violência de suas comunidades históricas, como
Achille Mbembe, Angela Davis, Nancy Fraser ou Roswitha Scholz, não
dissociam de nenhum modo a crítica da situação particular histórica do
movimento real de produção do poder capitalista. Este se resolve e se
oculta no mais difícil dos inimigos, a vida sob o regime, perverso, do
feitiço e do segredo -- a própria ordem do poder -- da forma mercadoria.
Por isso Fraser pode falar em demandas sociais neo"-liberais, para
aspectos acríticos dos movimentos de particularidade histórica...
Militantes, é preciso ainda um último esforço... Um bom livro,
desconhecido entre nós, sobre a produção de injustiça embutida na
fragmentação universal da crítica, que tomou a esquerda desde os anos
1990, é ``The trouble with diversity -- how we learn to love Identity and
ignore ineguality'', do crítico literário americano Walter Benn
Michaels. Precisamos saber articular os valores e direitos das histórias
particulares com a crítica da violência universal, a reprodução infinita
da sociedade de classes, pacificada na subjetivação para o consumo e a
produtividade produtora de mais valia.

\begin{flushright}
\emph{31 de outubro}
\end{flushright}

O Estado de S. Paulo: ``Contra Lula mercado já fala em Jair
Bolsonaro''... Projeto civilizatório do mercado. Nazistas tupiniquins
que trabalham para Wall Street temem que o seu dinheiro ``leve um
susto'' e por isso vão tacar o terror geral. Os homens bons do dinheiro.

\begin{flushright}
\emph{2 de novembro}
\end{flushright}

Oswald, O Rei da Vela, 1933:

``Há um momento em que a burguesia abandona a sua velha máscara liberal.
Declara"-se cansada de carregar nos ombros os ideais de justiça da
humanidade, as conquistas da civilização e outras besteiras! Organiza"-se
como classe. Policialmente.''

Então, estamos vendo o que acontece: ela estoura os tratos, rompe os
limites do sentido, relança o fundamento primitivo da acumulação, refaz
um ato sacrificial fundamental, aproxima a linguagem do ato puro da
violência, goza no sadismo do extermínio de alguém ou de algo e reproduz
e reafirma a sua real ordem de poder. Que se forma assim.

\begin{flushright}
\emph{7 de novembro}
\end{flushright}

As ideias de Butler, produzindo histeria conservadora no Brasil, repõe,
cem anos depois, o mesmo impacto crítico e político que a emergência da
sexualidade infantil, polimorfa, freudiana teve ao seu tempo vitoriano
burguês. Agora em um momento universal de impasse da relação entre
democracia e capitalismo, novamente a política do sexual -- teoria
performática de gênero, pluralidade do desejo... -- se torna centro de
uma crise sobre os limites da democracia. Freud deve estar se divertindo
muito, a respeito da reposição da relação poder , repressão, contra a
riqueza múltipla e a priori indefinida do sexual humano. Foi o mesmo
campo, em outro momento histórico, no qual ele foi atacado também. Mas
ele não tirou dessa crise uma política crítica da cultura, como faz
Butler -- e como fizerem Adorno e Marcuse, cada um ao seu modo, com sua
obra -- mas sim uma política clinica do cuidado e da diferença, a
Psicanálise.

\begin{flushright}
\emph{11 de novembro}
\end{flushright}

Amiga se pergunta, diante das agressões à Butler: que país é esse? Penso
que é o país em que este tipo de gente, com essa política de censura e
ódio, de redução da linguagem à violência direta, saiu às ruas junto com
pessoas boas indignadas com a corrupção da esquerda e com neoliberais
prontos para assaltarem direitos para destruir uma eleição em que alguém
foi eleito com 54 milhões de votos. E agora, esses que bateram panela
pelo que hoje está aí, a direita antidemocrática brasileira, se sente
forte e legítima para perseguir e bater em filósofos, artistas,
funcionários de museus, professores. É o preço que pagamos por legitimar
a política autoritária para os interesses neoliberais. Esses mesmos
homens que hoje atacam a corrupção da Judith Butler, mas não a do Michel
Temer. Um país que pôs os autoritários no poder, sem votos, para
destruir leis trabalhistas.

\begin{flushright}
\emph{17 de novembro}
\end{flushright}

O capitalismo se desresponsabiliza. Como pode destruir o ambiente, pode
destruir mundos. Países, sociedades, homens. O que ocorre no Brasil é a
dissolução de tudo para o aumento de qualquer lucro. Um lucro qualquer,
uma satisfação qualquer, já dizia Machado de Assis. Se pararmosmpara
pensar, tudo já foi destruído. Democracia. Eleição. Constituição. Leis e
qualquer garantias sociais. Qualquer coisa pode ser destruída.
Universidade. Contratos sociais. Só o Estado mercado global vai advir.
Bem"-vindos ao devir negro de Mbembe. O mercado teme por seu rico
dinheirinho, mas não por nós, nem por ninguém. Flerta com Bolsonaro.
Necropolitica da estupidez e do desrespeito a tudo. Uma coisa é certa,
não há Deus, nem justiça, nem nenhuma ordem de valor para o Capital. Ele
é a única modernidade total. Todos nós estamos aquém. Somos muito
lentos. Muito fracos. O pânico de todos nós, brigando por migalhas,
restos de mundos liqüidados, enquanto a máquina de milionários se
apropria de tudo, é o maior sinal da derrota. A derrota é a paralisia de
todo vínculo com a vida pública, coletiva e social. O capital vai vender
caro para muito poucos. Universidades para quê? O ganho é mais valia
absoluta absoluta. E senhores dementes prontos para caírem da última
pinguela. Não há modelo para o que devemos viver, bem lembrou Priscila
Figueiredo. O assalto não vai mais parar, nem o crime satisfeito do
poder. A direita prescinde de história, compromisso e verdade. O capital
goza, aqui é lá fora. Segurem"-se, o piloto sumiu, o sadismo é o limite
final. Para muitos ele nunca existiu. Agora para todos. O capital se
desresponsabiliza.

\begin{flushright}
\emph{20 de novembro}
\end{flushright}

Pondé e a dor dos outros: ``Se os `movimentos progressistas' não
tivessem eles mesmos virado um `mercado de impacto' de milhões de
dólares, (quase) ninguém estaria nem aí pra vítimas de racismo e
sexismo. A própria luta da Inglaterra contra a escravidão foi um
business em si.''

Folha de S. Paulo

\begin{flushright}
\emph{6 de dezembro}
\end{flushright}

Pensamento sobre gênero proibido no ensino médio.

\begin{flushright}
\emph{10 de dezembro}
\end{flushright}

Sobre mentira e neoliberalismo. Hoje chegamos a pós verdade e fascismo.

\begin{flushright}
\emph{31 de dezembro}
\end{flushright}

A esquerda institucional no poder não deu resposta política, equação
simbólica possível, de esquerda, para o problema do sistema brasileiro
de corrupção da política, e para a sua adesão a ele. Essa lacuna real no
pensamento da esquerda também a lançou para fora do próprio tempo. A
revolta conservadora dos moralistas anticorrupção também era legítima. A
esquerda preferiu ir presa e negar até o fim do que de fato falar do
assunto. Exatamente como a direita faz. Apenas nega até o fim o inegável
enquanto tenta comprar a lei, para não cumpri"-la. Grande fraqueza
simbólica e, consequentemente, política.

\begin{flushright}
\emph{5 de janeiro de 2018}
\end{flushright}

Brasil? África. Achille Mbembe.

``Apoiando"-se na desregulamentação e na privatização de economias
outrora controladas pelo Estado, esses grupos conseguiram enxertar seus
poderes nas redes de finanças e de extração em escala global, o que lhes
confere uma relativa imunidade. Agindo dessa forma, eles não são mais
responsáveis diante de suas sociedades. Por outro lado, eles puderam
comprar potentes apoios no seio delas. Colocadas juntas, essas forças,
cujas dimensões internacionais não são negligenciáveis, defendem o
status quo. Elas são os meios organizados e dispõem da força das armas,
do dinheiro e de sólidas redes locais e internacionais. Elas editam leis
que lhes são favoráveis e dispõem de tribunais para aplicá"-las ou,
quando apropriado, para ignorá"-las e contorná"-las. Esse bloqueio ao
poder não é destituído de contradições internas. Com as elites que se
fizeram donas e "capturaram" o Estado, tal bloqueio se torna em parte o
resultado da economia de extração à medida em que esta ingressa nos
circuitos da financeirização.''

\begin{flushright}
\emph{27 de janeiro}
\end{flushright}

Conversa com reacionária, anticomunista do nada:

Em seu país livre e democrático morrem 60 000 pessoas assassinadas por
ano. Todos os anos. Dez por cento dos assassinatos do mundo, segundo a
\versal{ONU} acontecem perto de você. Do outro lado do muro. A polícia técnica e
respeitosa dos direitos universais do Brasil mata 6 000 pessoas por ano.
Não tem guerra que se compare a isso. 5 homens desse país decente e
democrático -- que garante os seus direitos, mas não o dos jovens negros
descendentes dos 6 milhões de escravos trazidos para cá -- detém 50 \% da
riqueza, desse país que precisa de mais desemprego e menos leis
trabalhistas. Um banho de sangue, com muita gente irresponsável e muito
poucos donos do poder. Você não precisa ficar tão nervosa com a
estupidez geral que o golpismo autoritário estabeleceu por aqui, e as
gotas de ódio de classe média que respingam em sua página. Os facistas
que atacam em bando por aqui fazem bem pior. E em relação à real do
Brasil, essa brincadeira de empurra empurra e xingamentos mútuos no Facebook
não é realmente nada. O genocidio brasileiro não alcança a sua
classe. Você pode dormir tranquila.

\begin{flushright}
\emph{24 de janeiro}
\end{flushright}

Psicologia de massas do golpismo.

\begin{flushright}
\emph{27 de janeiro}
\end{flushright}

Le Monde, Brasil: uma democracia derretendo

L'élite de Brasilia baigne dans un climat d'impunité de nature à écœurer
le peuple. A quelques mois de l'élection présidentielle, le Brésil, pays
parmi les plus inégalitaires au monde, renvoie l'image d'une société de
castes où les dirigeants n'obéissent pas aux mêmes lois que les
miséreux. C'est indigne et dangereux pour la plus grande démocratie
d'Amérique latine.

Ou seja: a elite de Brasília está envolvida em um ambiente de impunidade
de modo a lesar o povo. A alguns meses da eleição presidencial, o
Brasil, país dentre os mais desiguais do mundo, reafirma a imagem de uma
sociedade de castas na qual os dirigentes não obedecem às mesmas leis
que os miseráveis. É indigno e perigoso para a maior democracia da
América Latina.

\begin{flushright}
\emph{5 de fevereiro}
\end{flushright}

Política do ódio, política da mentira, política da violência. A
subjetivação à direita do golpismo brasileiro.

\begin{flushright}
\emph{19 de fevereiro}
\end{flushright}

É isso. Esse é o problema maior. A desmobilização e desarticulação
política democrática real no espaço público atual. Ela tem muitas
frentes. O alinhamento total do Capital, com seus vários aparelhos
ideológicos de Estado, para a destruição do governo e da ideia da
esquerda democrática no Brasil. A criminalização da esquerda, sem
resposta política efetiva da própria esquerda. A fragmentação do campo
crítico em identitarismos imaginários simbólicos, que abandonam as
políticas universais, e fazem política de grupo e pessoalidade. A
redução do espaço crítico à internet, e ao teatro imaginário narcísico,
especular, do face"-book... por isso tudo, e por uma profunda articulação
maniaca da própria esquerda no poder no tempo Lulo -petista, com a força
imanente endógena da mercadoria representando os próprios feitos do
poder carismático, e sendo assim conservador de mercado -- tendente ao
liberalismo de consumo -- no meu estudo sobre o poder de Lula na
presidência pensei que se criava um mundo político anticritico desde
aquela época, exatamente o que você aponta agora, um mundo político sem
mobilização política efetiva, sem sequer o senso de política. Cultura
anticritica.

\begin{flushright}
\emph{20 de fevereiro}
\end{flushright}

De 15 anos para cá, tem havido, em várias partes do mundo, uma
reativação dos nacionalismos, dos fundamentalismos e dos racismos. É um
movimento que envolve a crise gigantesca do capitalismo que ocorreu a
partir de 2008 e desorganizou a integridade econômica e ideológica do
processo de globalização, que ordenava a produtividade econômica e
ideológica. Começam a surgir as soluções psíquicas simplistas para
substituir a análise de um processo contemporâneo complexo. Isso ocorre
porque eles, os fascistas, são incapazes de investir nessa elaboração
complexa. A saída é encontrar um inimigo para justificar tudo: o negro,
o estrangeiro, o árabe, o petista. A culpa é de alguém que eu conheço,
de um inimigo imediato contra quem eu estou autorizado a cometer uma
violência. O extermínio desse inimigo imediato repõe a ordem.

\begin{flushright}
\emph{3 de março}
\end{flushright}

Por vezes estamos tão perdidos em nós mesmos quanto achamos que o mundo
está perdido nele próprio. Tão arruinados que é melhor não querer tirar
nada daí. Tão vazios que desconfiamos que só teremos uma nova chance
sendo inteiramente outro ou algo definitivamente novo por fim. Se
chegarmos a ter nova chance diante de tal panorama realmente fantástico.
Nos aproximamos da morte já a conhecendo com uma certa intimidade em
vida. No fundo tudo isto é bom. Deve ser bom. Mas, na forma existente,
desesistente, dói, e apenas dói demais.

\begin{flushright}
\emph{10 de março}
\end{flushright}

A redução desestruturante dos Estados Unidos e da Europa a Homem, e do
Capitalismo a Branco é política na escala do corpo

\begin{flushright}
\emph{15 de março}
\end{flushright}

Execução. Mulher, negra, de comunidade, de esquerda. Política. É preciso
uma união antifascista já. E parar o país contra a cultura política do
ódio, da direita criminosa.

\begin{flushright}
\emph{16 de março}
\end{flushright}

\begin{flushright}
\emph{17 de março}
\end{flushright}

Ela foi assassinada. Uma morte política. Quem relativiza o sentido deste
crime apenas participa do seu desejo. Grandes canallhas justificando o
espírito criminoso que estabeleceram no país. Quem matou Marielle? Quem
comemora sua morte.

\begin{flushright}
\emph{17 de março}
\end{flushright}

Marielle é algo mais profundo do Brasil. Mais profundo do que o sistema
contemporâneo de sentidos que a envolviam, e em nome dos quais falava.
Ela é a voz mulher e negra do povo brasileiro que ganhou força de falar
a partir de si própria. Na política, como política. Na vida, como povo.
E por traz dela existem dezenas, centenas de outras Marielle, Bubas, e
tantas outras. O rolê tá só começando.

\begin{flushright}
\emph{18 de março}
\end{flushright}

Muita gente de bem, classe média golpista, inimigos dos direitos
humanos. O Brasil é o país em que mais se mata defensores dos direitos
humanos. Bárbaros cínicos golpistas tendem para o fascismo.

\begin{flushright}
\emph{18 de março}
\end{flushright}

Quem comemora a morte de Marielle a matou. Quem tenta despolitizar o
crime é cúmplice. Qual é a sua?

\begin{flushright}
\emph{19 de março}
\end{flushright}

Necropolitica.

Necrogoverno.

Necrogovernistas.

\begin{flushright}
\emph{20 de março}
\end{flushright}

Brasil denunciado na \versal{ONU}. Desrespeito sistemático aos direitos humanos
do Brasil, contra negros e pobres, um problema de todo o mundo

\begin{flushright}
\emph{20 de março}
\end{flushright}

Existem pessoas que entendem a luta política como o direito de eleger
outras pessoas para atacar e tentar humilhar. Projetar dores é
suficiente e é tudo. Essas pessoas estão em todos os espectros dos
circuitos ideológicos contemporâneos. Esta política imaginária do poder
direto de um sobre o outro, ``uma supremacia qualquer'', dizia Machado
de Assis, é conservadora da ideia do poder como real gozo imediato,
vitoria sem mediação sobre a existência de alguém, que pode e deve ser
reduzido a nada, ao mal, ao objeto do escárnio. É o escárnio real que
importa, não o outro, seu suporte. Só assim, na operação de um poder
qualquer sobre um outro, de fato existo, diz essa política, em um mundo
em que só o poder de fato importa. Micropolítica do gozo imediato. Da
vitória sobre o outro mais do que imediato. Micro fibra, pequeno sonho,
da microfísica do poder. Pequena alegria sádica, que dispensa todo
``trabalho do outro'', disseminada pela vida e pela cultura, exército
rarefeito da malidicencia e do real desejo de machucar, na
impossibilidade de se atingir o poder de fato, que determina todas as
vidas. Na impossibilidade de se destruir o Banco Itaú, a Rede Globo, a
classe dominante mundial, se tenta destruir qualquer outro aí, que
esteja bem próximo de mim, no plano dos pequenos narcisismos, criado
imaginariamente para ser o portador de todos meus males. Pequena
política odiosa, muito feliz na produção de inimigos nessa vida que já é
tão imensamente ruim.

\begin{flushright}
\emph{21 de março}
\end{flushright}

Por outro lado, a violência nas relações entre pessoas de caráter
político contemporânea, a agressividade e violência das perspectivas dos
grupos que defendem suas comunidades de compromisso, é também e
certamente um modo de sinalizar desníveis de poder e naturalizações de
gozos de privilégios Em um nível civilizatório simbólico, ou na crítica
de estruturas que são objetivas da reprodução do poder, a agressividade
dos múltiplos grupos sociais e históricos -- mulheres, negros
brasileiros, gays ou trans... -- também revela a inércia do poder em
reconhecer a sua própria violência. Em conjunto são partes do processo
crítico à vida social do poder, aspectos estratégicos da luta de classes
e luta por outra ordem de vida e sentido, recalcada e deformada no
presente. A agressividade do oprimido é sempre correlata a força de
naturalização do modo de ser do opressor, uma agressividade imensa
sublimada repressivamente como regra do jogo cultural e social.

\begin{flushright}
\emph{26 de março}
\end{flushright}

A sete anos atrás Denise Ferreira Da Silva escrevia isso, de Londres.
Hoje, com o Brasil confraglado em políticas de militarização da crise
social -- racistas, antihumanistas e de desprezo por pobres e seu destino
- e de cortes neoliberais em todos os níveis -- muito fortes na
educação... -- enquanto se aumenta abertamente os gastos com polícia e
exército nas ruas, tudo piorou. Tudo apenas piora. Há uma crise mundial
do Capitalismo em curso. Há uma ilusão bárbara, num mundo de
fragmentação e privatização de tudo, de que apenas o poder explícito
sobrevive. É o neofascismo e a mentalidade neofascista, bossalidade do
mal, que parece eclodir em toda parte. Inclusive em nós mesmos. Para
onde vamos?

``A beautiful day, today, in London! We need many more good days, more
on the streets everywhere! Too many held by the forces of global
capitalism, the dispossessed everywhere -- in jails, prisons, immigration
detention centers, refugee camps, occupied territories, you name it -
already have their future taken away from them. Now these forces
threaten to steal the hope to retrieve the future for everyone! No
pasaran!''

Um lindo dia hoje em Londres! Precisamos de muito mais dias bons, nas
ruas de todos os lugares! Muitos presos pelas forças do capitalismo
global, os despossuídos em toda parte -- nas cadeias, nas prisões, nos
centros de detenção de imigrantes, nos campos de refugiados, nos
territórios ocupados, como se diz -- já têm seu futuro sequestrado. Agora
essas forças ameaçam roubar a esperança de recuperar um futuro para
todos! Não passarão!

\begin{flushright}
\emph{20 de março}
\end{flushright}

Existem pessoas que entendem a luta política como o direito de eleger
outras pessoas para atacar e tentar humilhar. Projetar dores é
suficiente e é tudo. Essas pessoas estão em todos os espectros dos
circuitos ideológicos contemporâneos. Esta política imaginária do poder
direto de um sobre o outro, ``uma supremacia qualquer'', dizia Machado
de Assis, é conservadora da ideia do poder como real gozo imediato,
vitoria sem mediação sobre a existência de alguém, que pode e deve ser
reduzido a nada, ao mal, ao objeto do escárnio. É o escárnio real que
importa, não o outro, seu suporte. Só assim, na operação de um poder
qualquer sobre um outro, de fato existo, diz essa política, em um mundo
em que só o poder de fato importa. Micropolítica do gozo imediato. Da
vitória sobre o outro mais do que imediato. Micro fibra, pequeno sonho,
da microfísica do poder. Pequena alegria sádica, que dispensa todo
``trabalho do outro'', disseminada pela vida e pela cultura, exército
rarefeito da malidicencia e do real desejo de machucar, na
impossibilidade de se atingir o poder de fato, que determina todas as
vidas. Na impossibilidade de se destruir o Banco Itaú, a Rede Globo, a
classe dominante mundial, se tenta destruir qualquer outro aí, que
esteja bem próximo de mim, no plano dos pequenos narcisismos, criado
imaginariamente para ser o portador de todos meus males. Pequena
política odiosa, muito feliz na produção de inimigos nessa vida que já é
tão imensamente ruim.

\begin{flushright}
\emph{21 de março}
\end{flushright}

Por outro lado, a violência nas relações entre pessoas de caráter
político contemporânea, a agressividade e violência das perspectivas dos
grupos que defendem suas comunidades de compromisso, é também e
certamente um modo de sinalizar desníveis de poder e naturalizações de
gozos de privilégios Em um nível civilizatório simbólico, ou na crítica
de estruturas que são objetivas da reprodução do poder, a agressividade
dos múltiplos grupos sociais e históricos -- mulheres, negros
brasileiros, gays ou trans... -- também revela a inércia do poder em
reconhecer a sua própria violência. Em conjunto são partes do processo
crítico à vida social do poder, aspectos estratégicos da luta de classes
e luta por outra ordem de vida e sentido, recalcada e deformada no
presente. A agressividade do oprimido é sempre correlata a força de
naturalização do modo de ser do opressor, uma agressividade imensa
sublimada repressivamente como regra do jogo cultural e social.

\begin{flushright}
\emph{24 de março}
\end{flushright}

Leonardo Antunes agaradece o prêmio:

Agradeço esta distinção do prêmio,

que muito me emociona, por motivo

de que esse meu livrinho de sonetos

trata de coisas muito dolorosas:

de condições precárias de trabalho,

de depressão, de dor, de suicídio,

da exploração do corpo da mulher,

de uma desesperança incontornável.

Fico um pouco sem graça e constrangido

de receber um prêmio por um livro

que narra tanta dor, tanta injustiça.

Melhor seria nunca tê"-lo escrito,

num mundo que não carecesse dele.

Enfim, muito obrigado e boa noite.

\begin{flushright}
\emph{27 de Março}
\end{flushright}

Tiros contra os onibus da caravana de Lula no Sul do país. Tiros. Após
as pedradas, espancamentos e chicotadas na militância que queria vê"-lo,
atos de violêncioa real meio tolerados, meio celebrados -- dependendo de
onde se vê a coisa, de quem vê, de que rscista, site ou jornal, noticia,
daqui oi dali -- tiros... Tiros.

\begin{flushright}
\emph{29 de Março}
\end{flushright}

As ambiguidades e as tendências favoráveis dos conservadores liberais -
tidos até segunda ordem por democratas -- e seu apoio genérico e cinico
aos reais fascistas, autoritários e criminosos políticos que se
expressam com total liberdade hoje, não são apenas dubiedade e falha
moral e de caráter, de pessoas com dificuldade de compreender o
significado da violência na política, porque abandonaram o senso
histórico, e do valor dos direitos, a começar pelos humanos, em uma
democracia. Muito pelo contrário. A tendência dos liberais em assobiar
para o alto, e fingir normalidade política, enquanto seus pares da
direita apaixonada espancam pessoas, chicoteiam adversários políticos,
atiram em ônibus e ensaiam o lixamento real da esquerda, como mataram e
comemoraram o assassinato de Marielle Franco, toda essa dança cínica que
com um passo recusa a gravidade do estado bárbaro a que chegou o espaço
público político pós impeachment, e com outro acena de modo liberal para
a ação de reprodução do ódio e ameaça de extermínio, é de fato uma
política, uma estrégia para o poder, e um desejo que as coisas sejam
assim dos próprios liberais. Como todos sabemos, todo o processo anormal
de virada de mesa política do impeachment foi construído desde o
primeiro momento e sempre com o pacto social autoritário e conservador
entre direita liberal ``moderna'' e extrema direita arcaica e amante da
violência direta na política. A ponto que muitos dos grupos ``liberais''
que organizaram as estratégias políticas da tomada das ruas pelo
impeachment em 2015 não poderem de nenhum modo serem discriminados dos
atos e interesses fascistas do tempo, e são liberais econômicos radicais
que propõe estado ativo na violência contra pobres, censura de cultura,
arte e professores, destruição dos elementos críticos da cultura em nome
de entidades e valores que estão em crise fixados como a natureza das
coisas sociais. São liberais do dinheiro e autoritários e interventores
da cultura, da liberdade de expressão e crítica e do direito de
organização social democrática. De fato estes falsos liberais do alheio,
como dizia Machado de Assis sobre os seus antepassados escravistas,
sempre aceitaram, toleraram e andaram de mãos dadas com a turba
enlouquecida por ódio irracional, pronta para cometer qualquer crime
antidemocrático por sua ideologia do direito à violência simples,
primitiva como um sacrifício ritual. Os liberais se utilizaram
fortemente da paixão fascista. E os fascistas se utilizaram do cinismo
liberal para se legitimar e passar a agir à luz do dia na cultura
política brasileira. O que não podiam fazer desde 1984... Todas as
frentes políticas da direita, do neoliberalismo de liquidação total dos
compromissos sociais do Estado Brasileiro, ao fascismo bocó burro e
grosseiro do bonossaurismo, de fato se mobilizaram e andaram felizes
juntos, foram passear na Avenida, para alimentar a estratégia do
desmonte da democracia e da derrubada do governo de esquerda pela carta
coringa, perversa porque parcial, da responsabilidade infinita da
esquerda pela corrupção do sistema geral da política brasileira. Foi um
pacto do autoritarismo do dinheiro, democrata até a página dois, com o
autoritarismo de violência direta antissocial, que produziu a força
política para o impechment. Neste pacto se produziu a real força da nova
direita. Estes homens maus e violentos, dispostos a qualquer crime, uns
engravatados em suas empresas, aguardando novas oportunidades de
negócios com a quebra dos direitos do trabalho no Brasil, outros
ensandecidos com seus destinos ruinosos, e alimentados pelo espelho
odioso na internet, tem de fato, até segunda ordem, um projeto comum
para o Brasil, que passa pelo aberto aumento e uso estratégico da
violência na política e na vida. Todo processo da crise do impeachment,
do golpe da democracia, foi movido por uma real liberalização da
violência, e pela instrumentalização da política do ódio dessa nova
direita antissocial brasileira. O autoritarismo de liquidar a
legitimidade do processo do poder de Dilma Rousseff batendo panelas e
gritando enlouquecidamente nas ruas contra o ``comunismo do nada'' da
esquerda democrática e capitalista brasileira evoluiria fatalmente para
o próximo passo, bater nos homens, espancar o corpo, calar, ferir e
barrar a ação política do adversário com violência direta. O próximo
passo, como nos diz a história, é a organização de máquinas de
assassinato e extermínio, apenas o deslocamento das máquinas que já
fazem isso com pobres negros, para a esquerda existente. Facistas e
neoliberais andam juntos desde a origem de toda a crise e dividem o
serviço da violência, nas ruas e no Estado. Por isso meios de
comunicação, homens elegantes ricos, mulheres finas da alta sociedade,
empresários em busca de mais valia absoluta e direta, e de militarização
da crise social que criaram, desconversam, enunciam sempre um ``mas''
balbuciante, e fazem críticas tão tênues às ações de violência na
política que nem sequer se configuram

\begin{flushright}
\emph{30 de março}
\end{flushright}

Racista, homem branco cis hétero, comunista, esquerda branca,
intelectual branco, esquerda institucional, esquerdomacho, petista,
anarquista, vagabundo, bolsominion, fascista, feminista, feministas
americanas, feministas francesas, radfem, feminazi, identitário, negro,
mulher negra, neo"-negro, evangélicos, \versal{LBGTQ}, golpista, coxinha, tucano,
gente de bem.

\chapter{Um político preso, um preso político}

\section{Entrevista a André de Oliveira, El País Brasil}

\noindent\versal{ANDRÉ DE OLIVEIRA}: Do ponto de vista simbólico da Política, o que significará, daqui para
frente, o ex"-presidente Lula preso?

\noindent\versal{TALES AB'SABER}: É provável que, mesmo preso, em meio a um cenário estranho e excêntrico
de judicialização da política que vem ocorrendo no Brasil -- focado até
agora quase exclusivamente no \versal{PT} e no Lula --, os brasileiros que
viveram e melhoraram de vida sob os dois primeiros Governos do
ex"-presidente, continuarão sonhando com ele. Talvez, a partir de agora,
sonharão até mesmo com mais intensidade. Já que além da memória de uma
época em que mudaram de vida, agora há o aspecto sacrificial de seu
percurso e de perdas de perspectivas a que esses mesmos brasileiros
foram lançados nos últimos anos -- como decorrência de uma política
econômica neoliberal, um esquema de corrupção por enquanto inatingível
pela Justiça e o desprezo pela vida popular do governo pós"-impeachment.

\smallskip

\noindent\versal{AO}: De quem você fala especificamente quando diz desses brasileiros que
continuarão a sonhar com ele? Apenas a esquerda militante que simpatiza
com ele?

\noindent\versal{TA}: Não só, mas também daquele extrato da sociedade de pobres conservadores
que o consideravam um risco lá atrás e que hoje são os 30\% de
brasileiros que lembram de Lula e seu bom Governo e simplesmente o
elegeriam novamente presidente a qualquer momento. Eles, depois das
políticas sociais que beneficiaram sua vida e depois de um trabalho de
20 anos de investimento de uma legião de intelectuais e homens de
esquerda para legitimar a figura de Lula, passaram a confiar plenamente
no ex"-presidente. Tornaram"-se, assim, parte da narrativa de sucesso, que
stá muito longe de ser apenas uma mera narrativa, que os pobres
alcançaram durante os Governos de Lula.

\smallskip

\noindent\versal{AO}: Como você viu o discurso de Lula no Sindicato dos Metalúrgicos, logo
antes da prisão?

\noindent\versal{TA}: O ponto central foi a ideia de que, a partir de agora, ele é uma ideia.
Quando ele fala em ser uma ideia, que pertence a todos, ele evoca termos
universais para repor a força do desejo político de cada um. Lula não é
um demagogo comum, nem um manipulador regressivo, antidemocrático, como
existem tantos por aí hoje. Ele é a encarnação viva, numa potência de
corpo única, do conflito entre capitalismo e democracia, em um país
periférico e incompleto. Como representante democrático e negociador do
conflito de classes que sempre foi, ele tem razão ao dizer que é uma
ideia. Isso porque essa ideia de um mediador entre os excluídos e a
ordem econômica capitalista é algo que não morre, porque é estrutural do
problema da vida social. Essa ideia é uma das faces da própria
democracia e ele encarna isso. Lula está no núcleo estrutural dessa
questão: a democracia em sociedade de mercado responde ou não responde
aos interesses de pobres e excluídos? Ou é democracia apenas para os
detentores do poder de mercado? A ideia da dialética política dos
conflitos de classe, de modo democrático, que é a defendida por Lula, é
uma virtualidade civilizatória da própria democracia. Então, se de fato
houver democracia, essa ideia é necessária, e não morre.

\smallskip

\noindent\versal{AO}: Apesar da barba e da fala enérgica do líder sindical, Lula nunca foi um
radical, não é estranho que essa imagem seja colada a ele agora?

\noindent\versal{TA}: Essa imagem só vale para a parte da direita mais grosseira brasileira, a
que tem feito tanto estrago na vida nacional e constrangido todo o
espírito democrático. É claro que Lula sempre se moveu em uma corda
bamba política, na qual ele se equilibrava bem. Sempre ficou entre
assumir e convocar os interesses sociais populares, que foram muito
tardiamente representados no poder executivo brasileiro, e negociar
condições políticas e força institucional real por meio da interlocução
com os interesses reais do capital nacional. Seu projeto, muito longe de
qualquer radicalismo, foi um modelo de capitalismo nacional, integrado
aos fluxos globais. Um plano de desenvolvimento de economia produtiva e
de mercado interno, que empregava, produzindo aumento constante da
inserção no trabalho e no consumo -- a única moralidade que o
capitalismo conhece, segundo Keynes.

\smallskip

\noindent\versal{AO}: Mas o que explica, então, a eleição de Lula por parte da sociedade como
o pior mal da nação?

\noindent\versal{TA}: O que se viu no Brasil nos últimos quatro anos -- desde quando a crise
econômica mundial se agravou durante o Governo de Dilma Rousseff, dando
sinal para a guerra aberta que vimos em seu segundo mandato -- não foi
apenas um ataque de parte da sociedade ao Lula, mas algo que visou
degradar toda a esquerda ao atacar a sua imagem. Estamos vivendo um
estado de guerra total em que há uma política de ódio paranoica muito
primitiva em que o alvo é a esquerda democrática contemporânea. Não à
toa, do nada, foi reinventado um anticomunismo delirante. Há uma massiva
metafísica do mal, muito violenta, um desejo gnóstico negativo, que
permite o desprezo total pelo inimigo imaginado que, no caso, é Lula,
representando toda a esquerda. Foi assim que a direita brasileira, em um
estado de paixão que, entre outras coisas, produziu toneladas de
mentiras infindáveis na internet, alcançou um grau de intolerância
elevado. Lula é amado demais, e odiado demais também. Torna"-se a
obsessão de todos. E mesmo sendo um imenso democrata, como Lula é, será
odiado. É difícil para o todo do poder de classes no Brasil ser
confrontado diretamente pelo poder de um único homem. O sacrifício de
Lula é social, realizado por grandes conflitos simbólicos de classe. O
poder instituído não pode tolerar um homem que sozinho tenha tanto poder
pessoal, carismático. E o resultado da sua ausência, e do \versal{PT}, do espaço
político brasileiro é a liberação de todo poder, e a real face
antissocial, do capital brasileiro.

\smallskip

\noindent\versal{AO}: Você definiu o Lula como um ``herói da luta de classes'' no Brasil, mas
sua imagem de radical ficou para trás faz muito tempo. Por que essa
definição?

\noindent\versal{TA}: Lula fez um Governo entre as bolsas sociais para os brasileiros muito
pobres, as desonerações de impostos básicos e o financiamento de consumo
de bens fundamentais para boa parte da população que não podia consumir
em um mundo cujo centro simbólico mais forte é o próprio consumo. Por
outro lado, houve sempre o enriquecimento dos muito ricos, com as
garantias econômicas de um ativo governo desenvolvimentista. Nenhum
contrato foi rompido, o superávit fiscal foi mantido e até ampliado, a
inflação controlada e a vida econômica, com trabalho formal e com
direitos trabalhistas, disparou o crescimento país. O cenário
internacional era favorável a essa política interna mais rica. Assim,
depois de seus dois mandatos bem sucedidos, Lula ficou imensamente
poderoso. Era celebrado pelos pobres, que se sentiam contemplados de
modo digno e raro no Brasil. Mas também foi celebrado pelos mercados que
estavam aquecidos e que enriqueciam. Além disso, sua política sempre
conciliadora aceitou o arcaísmo e a regra do jogo da corrupção
universal, que foi mantida durante todo seu governo. Assim, nessa época,
ele era uma solução para todos.

\smallskip

\noindent\versal{AO}: Um ``herói'' pelo fato de ter conseguido conciliar diferentes brasis
durante oito anos?

\noindent\versal{TA}: Sim. Projeto político de conciliação dos contrários. Mas também pela
imagem que se acabou criando ao seu redor. Durante este período
importante da vida de Lula, o de seus dois Governos, seu valor nos
corações e mentes dos brasileiros cresceu muito a seu favor. Com a
investidura simbólica do cargo de presidente e com a sua linguagem e
performance pessoal hábil e rica, de caráter moderna e popular, além do
seu \emph{habitus} de classe renovado por sua própria história de
ascensão e experiência, Lula se tornou um verdadeiro super"-herói
carismático, no país e no mundo. Ele se elevou ao nível mais fluido e
universal do carisma político contemporâneo, o \emph{carisma pop}, que
aproximava e ligava o seu sucesso econômico, o seu charme pessoal, a sua
habilidade de linguagem e presença de espírito com a própria excitação
da vida animada do mercado mais comum que acontecia no Brasil. Tudo
confluía para o seu poder e a sua propaganda, natural industrial. Nenhum
político brasileiro jamais conseguiu isso, e talvez nenhum outro jamais
o consiga neste nível.

\chapter{Fascismo comum, sonho e história}

Uma das realizações de regimes fascistas efetivamente operando em seu
mundo é a produção daquilo que George Orwell chamou de \emph{novilíngua}
em \emph{1984}. O regime fascista sempre pesa sobre a
língua e a própria linguagem, como pesa originalmente sobre o psiquismo
disponível ao passado do fascista. Ele completa e torna densa a relação
de cisão e de poder existente entre a linguagem e a realidade social.
Fixada por violência e pelas balas e bombas do poder, no fascismo a
ideologia tende a se tornar o \emph{real}, fazendo efeito mesmo como
outra \emph{coisa} sobre o sonho, dando à linguagem a concretude da
pedra, a que se atira sobre o inimigo e a que esmaga e paralisa a
possibilidade de circulação da diferença.

Atirar pedras, espancar, torturar ou fazer barulho, ou bater panelas...,
para assustar o inimigo evocando um estado de guerra primitiva,
imaginária ou real, são traços e operações de poder arqueológicos, que
deixaram a marca de horror que pressupunham na própria linguagem do
futuro, reduzindo o sabido voo do espírito ao ato material sobre o corpo
do outro. São traços do passado distante que podem voltar, como memória
\emph{da forma}, do ato e da coisa, e não do sentido, trabalho do
pensamento que não existe aí.

O passo final das clivagens fascistas, das suas certezas que legitimam a
violência e o extermínio, a tortura e o escárnio dos adversários
políticos, seus gozos de massa, de sua falsa identidade de uma
superioridade qualquer, de sua vida prática que busca a ação e que
recusa fortemente qualquer conhecimento meditado, criativo ou crítico de
algum modo é uma ampla curvatura descendente no plano da linguagem, o
carregamento excitado das palavras que tende ao concreto de seu valor, o
desprezo aberto por outras palavras que devem ser recusadas, negadas, o
deslocamento do plano do léxico e da semântica para outro centro
gravitacional cuja natureza política é interessada, e imensamente
triste.

``A partir de 1939, o carro de corrida foi substituído pelo tanque, e o
motorista de automóvel foi substituído pelo \emph{Panzerfahrer}
{[}\emph{motorista de tanques}{]}. (...) Durante doze anos, o conceito e
o vocabulário do heroísmo estiveram entre os termos prediletos, usados
com maior intensidade e seletividade, visando a uma coragem belicista, a
uma atitude arrojada de destemor diante de qualquer morte em combate.
Não foi em vão que uma das palavras prediletas da linguagem nazista foi
o adjetivo \emph{kämpferisch} {[}\emph{combativo, agressivo,
beligerante}{]}, que era novo e pouco usado, típico dos estetas
neorromânticos. \emph{Kriegerisch} {[}guerreiro{]} tinha um significado
muito limitado, fazia pensar somente em assuntos de \emph{Krieg}
{[}guerra{]}. Era também um adjetivo claro e franco, que denunciava a
vontade de brigar, a disposição agressiva e a sede de conquista.
\emph{Kämpefersch} é outra coisa! Reflete de maneira mais generalizada
uma atitude de ânimo e de vontade que em qualquer circunstância visa a
autoafirmação por meio de defesa e ataque, e não aceita renúncia. O
abuso da palavra \emph{kämpferisch} corresponde ao uso excessivo, errado
e próprio do conceito de heroísmo. (...) Desde o primeiro dia de guerra
até a queda do Terceiro Reich, todo heroísmo em terra, ar e mar usou
uniforme militar. Na primeira guerra ainda existia um heroísmo civil por
tráz da linha do \emph{front}. E agora? Até quando haveria um heroísmo
ali? Por quanto tempo ainda haveria vida civil?''\footnote{Victor
  Klemperer, \emph{\versal{LTI}, a linguagem do terceiro reich}, São Paulo:
  Contraponto, 2009, p. 42.}

Em seu estudo sobre a degradação e a produção de linguagem própria do
nazismo alemão e seu regime o primeiro ponto que Victor Klemperer
destaca e recorda é a busca de uma fusão da ideia de belicosidade comum
e desabrida, agressividade na vida, com a política ampla da
transformação de tudo o que existe e vive no mundo em ``guerra''.
Configurando uma construção em que a guerra deve se tornar total,
\emph{interna aos sujeitos}, ato de subjetivação e de ser, e o horizonte
de todo o mundo externo existente, da cultura, o mundo do nazismo era a
substituição da vida civil pela vida como batalha sem sobreviventes.
\emph{``O carro de corrida foi substituído pelo tanque, e o motorista
pelo panzerfahrer}''. \emph{Kämpefersch}.

Assim do velho mundo esportivo e de espetáculo técnico, ligado à cultura
liberal, o mundo cotidiano do elogio da competição e do desempenho --
desempenho de mercado, e de guerra, ``cujo princípio é o mesmo'' dizia,
simplesmente, Marcuse --, do prazer da vulgaridade agressiva cotidiana e
comum na vida moderna, sublimada na forma da própria técnica, o bólide
do carro de corrida, objeto fálico de um gozo que voa rápido e vai na
frente, ultrapassando a cultura que o segue irremediavelmente como
empuxo e como vácuo, o progresso, a cultura nazista definitivamente tira
o peso da façanha individual, desrealizando"-a, e a esquece, poderíamos
dizer, concentrando todas as intensidades inteiramente na língua da
façanha técnica de Estado, das divisões panzer maciças e pesadas, dos
tanques que ocupavam imenso espaço concreto no mundo, reais corpos
lentos do sentido mas totalitários na torção que fazem do próprio espaço
com sua presença, ocupantes imensos de espaço vital também na própria
língua.

O deslocamento é expressivo, habita os significantes e os sintagmas da
vida, configurando uma espetacular regressão tópica, temporal e formal
no interior da própria linguagem: do indivíduo, da competição, da
técnica e do mercado, como sonho comum do mundo liberal burguês, à massa
social fundida ao Estado, a ocupação concreta do espaço, a tecnologia
bélica e a guerra, como espírito comum do tempo.

A restrição e a alteração da vida imaginária, e do universo de palavras
disponíveis era uma realidade política clara do fascismo, e do aberto
aventuroso do mundo excitado do brilho individual e burguês, chegávamos
ao fechado, invasivo, bélico, destrutivo e pesado como o Estado do
tanque de guerra alemão, ou italiano. Renuncia"-se a vida do espírito,
que a senhorita valorizava...'', diz Klemperer a uma amiga em plena
ascensão do nazismo, nova convertida que justifica tudo.

Semântica e léxico sociais estavam alterados, na direção da restrição,
comunhão orgânica e do peso, além da belicosidade como cultura.
\emph{Panzerfahrer}. \emph{Kämpefersch}. Era o espírito do tempo, de uma
solução -- ou \emph{dissolução} -- em violência da crise aguda do
capitalismo da época, que falava, encarcerando e aproximando as palavras
da luta desabusada e direta e do bando em busca de confusão e
sacrifício, mais baixo e comum. O bando que renunciou à linguagem.

Em seu estudo sobre a ideia do \emph{ur} \emph{fascismo}, das condições
de irresponsabilidade, transcendência e ativação da violência presentes
em todo movimento histórico de tipo fascista, Umberto Eco também anotou
algo a respeito da vida das palavras em um regime de ordem e progresso
muito autoritário, centralizado no líder do Estado:

``Em 1942, aos dez anos, eu ganhei o primeiro prêmio do Ludi Juvenelis
(concurso de livre participação forçada para jovens fascistas italianos,
a saber: todo jovem italiano). Havia discorrido com virtuosismo retórico
sobre o tema: `Devemos morrer pela glória de Mussolini e do destino
imortal da Itália?' Minha resposta foi sim. Eu era um garoto esperto.
Depois, em 1943, eu descobri o significado da palavra `liberdade'.
Naquela época, `liberdade' ainda significava `libertação'. (...) Na
manhã de 27 de julho de 1943 foi"-me dito que, de acordo com as
comunicações lidas no rádio, o fascismo havia caído e Mussolini havia
sido preso. Minha mãe mandou"-me comprar o jornal. Eu fui à banca mais
próxima e vi que os jornais estavam lá, mas os nomes eram diferentes.
Além disso, após um breve olhar pelas manchetes, percebi que cada jornal
dizia coisas diferentes. Comprei um ao acaso e li uma mensagem impressa
na primeira página assinada por cinco ou seis partidos políticos, como
Democratas Cristãos, Partido Comunista, Partido Socialista, Partido da
Ação, Partido Liberal. Até aquele momento de minha vida eu acreditava
que havia apenas um partido para cada país e que, na Itália, havia
apenas o Partido Nacional Fascista. Eu estava descobrindo que no meu
país poderiam existir jogos políticos diferentes, simultaneamente. Não
só: como era um garoto esperto, logo percebi que era impossível que
tantos partidos houvessem surgido de um dia para o outro. Entendi que
eles já existiam como organizações clandestinas. A mensagem celebrava o
fim da ditadura e o retorno da liberdade: liberdade de expressão, de
imprensa, de associação política. Estas palavras, `liberdade',
`ditadura' -- Deus meu -- foi a primeira vez na minha vida que as li. Em
virtude dessas novas palavras eu tinha renascido como um homem ocidental
livre.''\footnote{Umberto Eco, ``Ur Fascismo (O Fascismo Eterno)'',
  https://groups.google.com/forum/\#!topic/livros\_online/NFN0ye-94xA}

O menino do interior da Itália -- como Fellini recorda também em
Amarcord -- vive, em julho de 1943, um rápido movimento avesso àquele do
espírito pesado que tomou o mundo moderno muito fixado de Klemperer.

Após passar a sua vida sob o regime fascista de Mussolini o menino sabe
escrever bem, aliás muito bem, sobre a submissão da vida à pátria e ao
líder controlador. Palavras e pensamentos convergem unidas, na criança,
para o poder. Ele sabe participar corretamente da convocação
\emph{livremente forçada de todo garoto fascista, ou seja todo menino
italiano}, para reproduzir e aumentar o poder do Estado e seu guia. Ele
sabe ser esperto e mobilizar a língua de algum modo, como sempre
saberia, para reconhecer e ser reconhecido pelo regime que o formou, que
o formou tanto quanto a própria mãe, evocada por um segundo no relato,
um fio de continuidade de si mesmo em um mundo que se revolucionava e se
abria em um cenário histórico de possibilidades.

No entanto, o futuro linguista e romancista pós"-moderno desconhecia
completamente o significado de certas palavras da própria língua quando
públicas e políticas, e não sabia do movimento da vida de amplos
aspectos da história, imagens da vida, que se elidiam em conjunto com o
esvaziamento da vida das palavras banidas. Toda uma semântica da vida
social lhe fora realmente ocultada, subtraída, toda uma matéria de sonho
lhe fora duplamente recalcada, em seu mundo fascista quase por natureza
das coisas. Toda uma estrutura da emergência mesmo do sentido lhe era
desconhecida. Ocupada por outra ordem de sonhar, e natureza de desejo,
que nomeava tudo de um outro lugar, \emph{livre forçado}, em que ser
italiano era idêntico a ser fascista, a vida dos conceitos básicos da
política moderna, e suas palavras, lhe eram exotéricas.

Suas energias de vida foram condensadas no sistema de sentido das coisas
políticas em que crianças de dez anos deviam escrever nas escolas de
toda a Itália sobre o valor de se morrer pela pátria, concentrada na
figura do líder que a enuncia com o próprio corpo. Outra vez, os
sentidos fortemente restritos, muito menos do que uma capacidade de
sonhar, aproximavam abertamente a subjetividade da capacidade de morrer,
e de matar... O sonho fixo das palavras que emanam do corpo do líder e
do projeto político extenso na cultura dissipa nuvens mais amplas de
sentido, de praias e de passagens da dinâmica política, de imagens, de
direitos, de experiências, mas também \emph{do próprio nome dos direitos
perdidos} e já não mais sonhados. Essa destruição, forçada livre, também
constituía uma prisão na própria ordem das palavras, como deixa claro o
menino linguista.

Para além da violência direta, o sistema político que estreita os
espaços entre a esfera pública e o braço excitado de quem espanca,
atira, tortura e mata, o fascista queria imprimir no campo da
representação pública um conjunto de palavras aproximadas da coisa mesma
que representam, enquanto também extirpa, como um cirurgião carniceiro
do simbólico, mundos e mais mundos de possibilidades de sentido e de
experiência, que desfalecem em conjunto com a morte programada do outro
na cultura. A cultura programática da morte e do extermínio, é cultura
da morte de palavras, e com elas, de sentidos.

Liberdade de expressão, de imprensa, de associação política, múltiplos
partidos, um espaço público concebido como plural, de múltiplos jogos
simultâneos e ocorrendo em múltiplas temporalidades sociais. Foi neste
espaço de outra forma que o menino fascista foi lançado subitamente.
Tudo deve ter sido vertiginoso, uma onda de erotismo na cultura, que
trouxe de volta palavras e modos de viver \emph{que estavam sobre
ocupação fascista}. Sim, pois sabemos desde Sade que a grande maquinaria
necessária do gozo sádico, sua catedral de posse e tortura, é apenas
assessório, necessário, para o controle e o uso absoluto do corpo do
outro: as palavras coincidem com a máquina de tortura que coincidem com
o gozo fascista. Qual terá sido o choque de uma certa ideia de
liberdade, no sentido da possibilidade da vida se mover em variados
pontos e sistemas de sentido, partidos políticos e zonas de linguagem,
versus o termo, também inexistente no sistema da restrição social da
força fascista, ``ditadura''?

De um lado, Eco descreve uma descompressão social, uma explosão de
sistemas de vida e de linguagem, outras apostas sobre o campo político,
outros desejos, articulados à ideia de um campo social \emph{livre}.
Outra produção de vida, outras palavras. O peso do líder Estado, do seu
desejo de morte restritiva do nome das coisas que existissem sem ele, de
seu tanque de guerra universal do sentido, de sua cultura do insulto, da
belicosidade e da organicidade -- vimos bem em Amarcord, com sinal de
liberdade e ridículo a posteriori -- de um senso de historicidade em que
muitos agentes disputam o sentido das coisas humanas, que deviam ganhar
um ponto em dialética, o que implica na sua real abertura para a
história. \emph{Ditadura}, \emph{liberdade}.

Podemos intuir bem no relato como a cultura fascista é o negativo
realizado de um espaço de vida entendido como multiplicidade, da
pluralidade mínima dos direitos liberais coordenados pela sociedade de
classes, ao que poderia chegar a ser ainda a pluralidade máxima ``de
cada um segundo as suas capacidades, a cada um segundo suas
necessidades'', de um virtual socialismo democrático realizado. Assim,
de fato, o cinema italiano do pós guerra era popular, livre, aberto à
rua, humanista e revolucionário. Se sua real esperança socialista foi
barrada no processo de redemocratização visando ao mercado mundial -- o
que levou Pasolini ao final dos anos de 1960 a falar em um novo
fascismo, \emph{fascismo de consumo --}sua força de experiência e valor
desejante de humanidade de fato revolucionou o mundo do cinema, e o
cinema mundial, nos anos de 1950 e 1960. Eco nos dá a medida de
ressubjetivação da expansão forte do mundo das palavras, seu rápido
desdobramento do plano da cultura em seus novos termos, que representam
práticas do público e do político, o mesmo fenômeno de expansão
humanista democrático e formal maravilhoso que vemos a vida no cinema
italiano do pós"-guerra. O menininho viveu a mesma emoção e expansão da
vida que vemos, a própria forma, em um filme de Rosselini ou de Sica,
que contavam aquela mesma história. Um cinema que se expandiu com tal
força e de tal forma que criou, a partir da sua expansão antifascista,
todo os cinemas nacionais e modernos do mundo, a partir dos anos de 1950
e 1960.

Vejamos o impacto mais forte desta diferença, entre a cultura da
concentração e do peso, organizada para a guerra, e a cultura da.
multiplicidade, organizada para a ideia de fundo moderna de alguma
liberdade. Sonhos podem nos dizer ainda melhor a natureza dessa relação
política, de choque, sobre o corpo simbólico de uma pessoa em uma
cultura que se restringe à violência política que a cerca. As camadas
políticas e concretamente sociais, históricas, que sempre habitam o
sonhar humano -- como já nos diziam Roger Bastide e também Theodor
Adorno, e como Freud foi o primeiro a mostrar, na série de sonhos
políticos de \emph{A interpretação dos sonhos}, conhecida como os seus
\emph{sonhos romanos --} nessa hora histórica limite se representam
ainda com maior nitidez.

Porque o sonho é o limite simples da resistência, a fonte da mobilidade
psíquica, o único resto da ideia de liberdade, o que o fascista visa é
de fato dominá"-lo, paralisá"-lo, reconfigurá"-lo mesmo como forma: de sua
negociação civilizatória fundamental, da metáfora, da distância e da
poesia do sonhar, do exílio humano sonhado em sentido, à ação direta de
descarga e de recusa da existência do outro. \emph{Kämpefersch}. Um dos
alemães antinazistas, que eram obrigados a viver sobre Hitler, sonhou em
1934:

``A \versal{SA} instala arames farpados nas janelas dos hospitais. Jurei para mim
mesmo que não admitia isso em minha seção, caso chegassem com seu arame
farpado. Mas acabo permitindo que o façam e fico ali, a caricatura de um
médico, enquanto eles quebram os vidros e transformam um quarto de
hospital em campo de concentração com arames farpados. Mesmo assim, sou
demitido. Porém, sou chamada de volta para cuidar de Hitler, pois sou o
único no mundo que pode fazê"-lo: fico tão envergonhado de meu orgulho
que começo a chorar.''\footnote{Charlotte Beradt, \emph{Sonhos no
  terceiro reich}, São Paulo: Três Estrelas, 2017, p. 78.}

O sistema fascista de linguagem, de cultura, é um sistema de ações. Um
sistema de ocupações ativas de sentido do espaço da vida simbólica
pública, das subjetivações e, no limite, dos próprios sonhos. O mais
íntimo, e aquilo que resiste, como dor, a toda violência. Toda mentira e
toda linguagem fascista é uma ocupação de choque do real, uma mudança de
sentido das coisas do mundo: arames farpados nas janelas do hospital, o
hospital torna"-se o campo de concentração, revela toda a agressividade e
política que ele costuma ocultar e sublimar na ordem do mundo liberal. O
campo do simbólico, espaço de movimento e vida do próprio sonho, é
invadido pela coisa mesma de uma cultura que se torna farpada e
violência em expressão. \emph{A} \emph{coisa penetra o espaço do
símbolo}. Ao final é o próprio Hitler que faz exigências ao sonhador,
porque de fato é o fascista que faz exigência simplesmente a tudo.

A cultura da mentira fascista é cultura de inversão do valor e dos
sentidos das próprias coisas, uma ação invasiva e violenta para que as
coisas mudem de nome, não sejam mais o que são. Não por liberdade elas
devem alterar sua substância, não por erotismo ou por criação. Mas por
desejo do poder. Para que hospitais se tornem prisões, e ``profissionais
liberais'', ou homens públicos sirvam ao poder real, se alinhem com o
seu desejo, se tornem ``médicos de Hitler''. A mentira pública
sistemática do poder busca com insistência a invasão ativa dos espaços
concretos, a destruição das \emph{fronteiras} \emph{significantes} da
democracia sempre claudicante, de modo a degradar a natureza dos objetos
existentes, das coisas e seus sentidos, a favor de seu núcleo de força,
produtor puro de poder. Como veremos, os próprios limites significantes
das palavras entram em crise. Elas estão de fato sendo dissolvidas, para
ganhar nova configuração desde a estrutura do desejo fascista, que
penetra o mundo, que mais \emph{quebra os cristais} das palavras do que
respeita algum pensamento que possa de fato atravessá"-las. Mentira é
ação, ocupação, e a ocupação das coisas e desde o espaço da política vai
gerar a nova cultura da mentira, com seus novos termos: novilíngua.

O exemplo de quem sonha é dramático. Ele encena o terror político
cultural frente o espaço social que se torna o peso da máquina de
produção fascista. Como o tanque e a belicosidade que tomaram a cultura,
o hospital se tornava também máquina de guerra, e o sujeito liberal de
alguma personalidade democrática sente a invasão completa de seu espaço
subjetivo pelas mesmas formas pesadas. Ele ainda resiste, mas se sabe
tomado de assalto pela ordem da violência, que é prática, que ganhou
força na cultura e que é sonho. A resistência pessoal e subjetiva está
no limite, na fronteira, contra a transformação do próprio sonho
traumatizado, não há como barrar, no sonho e na vida, a transformação do
hospital em campo de concentração \versal{SA}. Antes de ser expulso do espaço da
violência, desejo de negá"-lo, antes de ser demitido, o sonhador se torna
uma \emph{caricatura de um médico}: o movimento da ocupação do mundo e
de si mesmo pelo terror é o movimento da desrealização de si próprio. O
eu e suas ilusões na ordem liberal se tornam progressivamente
irrelevantes, praticamente \emph{de papel}. O circuito da linguagem do
poder se apropriou do espaço público, e vem do todo, do continente das
coisas e símbolos, intensamente para dentro do sujeito, que se
desestrutura com ele, como um veneno psico"-ideológico, como \emph{espaço
coisa, real,} contra o próprio sonho. Sua mentira é eficaz porque ela é
ação real, real poder. Poder de deformar as próprias coisas. Ela mente
sobre o hospital, mas também não mente mais, porque o hospital não é
mais um hospital, é uma prisão e um campo de violências fascista. Como a
cultura. E o sujeito que sonha? Ele não é um fascista enquanto ainda se
aterroriza, reconhece a violência e o absurdo, e oscila diante do risco
da própria adaptação totalitária.

A subjetividade está sitiada, pela conversão exigida pela cultura da
mentira e da violência, que vem dos horizontes do mundo, o sujeito terá
que decidir, entre a verdade da própria negatividade e a conversão à
máquina de guerra, agressividade, desprezo e poder. Ele se tornará
médico de Hitler? Pela pressão identificatória do todo, e pelo princípio
de conservação, seu desejo será esmagado pela força de vida e de morte,
o terror da mentira, muito ativas no poder fascista? Ele será convertido
ao desejo simbólico do poder, pela ação de mentiras públicas, ameaça
concreta e poder de Estado? O sonho faz a pergunta política de raiz. A
pergunta do eu diante da identificação com o elemento totalitário do
todo.

Nos sonhos, aquele homem na fronteira de toda violência histórica sobre
si próprio \emph{jurou que não}. Mas ao fim do sonho, após ser
expatriado da cultura do poder, por ainda saber o que é um hospital e o
que é uma prisão, ele é convocado, ele precisa se colocar diretamente,
frente a frente com Hitler. Há desejo e há trauma nessa relação. É assim
que se enfrenta a cultura fascista, de frente e negativamente. Ele terá
que olhar para a realidade do poder, porque ela já não é mais recusável,
de nenhum modo. O limite trágico e agonístico do fascismo e da
subjetividade está colocado aí: ele é o único no mundo que pode salvar
Hitler, ou seja, do ponto de vista político, aceitar a sua máquina de
violência e de mentira. E como médico, que é o único que pode salvar o
ditador, ele é também aquele que pode deixa"-lo morrer... Ao converter
totalmente a cultura em guerra o fascista exige de cada um uma decisão
de vida e morte diante dele próprio.

Este é o maior horror, o paradoxo final de quem vê a vida como violência
e desprezo pelo outro: apenas a guerra liquida, ou tranforma, quem faz
da vida uma guerra real. Uma guerra com a \emph{forma} do fascismo, que
implica negá"-lo na raiz e sempre. Uma guerra para fora da guerra fria do
sonho fascista. Outra formação. O sonho mostra o quanto é difícil este
trabalho social em si mesmo.

O sonho do médico alemão não nazista põe em ação no espaço da própria
subjetividade aquela tragédia cultural política ridícula de
aprisionamento do mundo, vivida e compreendida por Victor Klemperer.
``Por quanto tempo ainda haveria vida civil? A Doutrina da guerra total
se voltava contra os seus criadores de maneira terrível: tudo é
espetáculo bélico, o heroísmo militar pode ser encontrado em qualquer
fábrica, em qualquer porão. Crianças, mulheres e idosos morrem a mesma
morte heroica, como se estivessem em campo de batalha, com frequência
usando o mesmo uniforme desenhado para jovens soldados no
front.''\footnote{Victor Klemperer, \emph{op. cit..}} A vida civil se
tornara a norma da vida limite da paixão autoritária, agressiva e
bélica. De fato, Hannah Arendt lembrava a degradação odiosa de toda a
vida pública europeia em um ar tóxico de desconfiança, e desrespeito
generalizado, que, após a catástrofe do mal imperialismo da primeira
guerra mundial, tomou a Europa, e preparou o terreno profundo para a
ascensão do totalitarismo fascista. E os homens mobilizados,
paramentados, uniformizados, invadidos pela estrutura de desejos do
próprio mundo do poder não morriam mais ao seu modo, como dizia Freud,
no seu grande comentário metapsicológico aos efeitos da primeira guerra
mundial sobre todos, mas morriam ao modo \emph{do desejo do poder}. A
vida civil tornou"-se apenas o inferno de sua própria supressão. E
generalização da cultura da inimizade. O motorista do tanque de guerra,
a agressividade e belicosidade comum, que andavam nas ruas e nas
cervejarias, o hospital como prisão e a morte em estado de guerra
permanente. A morte da cultura, e a morte como cultura.

Charlotte Beradt prossegue na leitura do sonho do médico, paradigmático
da degradação subjetiva e incorporação ao poder, \emph{conversão ao
poder}, que prossegue sendo sonhado pelo sonhador atormentado: ``O
médico acordou totalmente acabado, como acontece frequentemente quando
se chora em sonho. Durante a madrugada, pensou sobre o sonho e encontrou
sua causa premente, também muito esclarecedora para o quadro geral: na
véspera, um de seus assistentes fora trabalhar na clínica com um
uniforme da \versal{SA}, e ele, apesar de revoltado, não protestou.''

Aí está a invasão e a degradação do espaço civil pelo desejo e pela
linguagem, pelo espírito, fascista. O jovem médico nazista, com seu
uniforme paramilitar, já ocupa, com acinte e arrogância, certamente
desafiador e agressivo, o espaço \emph{neutro}, o espaço social liberal
e sua ordem de valores, o espaço da vida médica, que, em teoria, não
deveria ser cercado pela política. Não de modo saturado, uniformizado,
tendente a transfiguração da vida ao partido, e das relações à luta
constante e universal pelo poder. Por quanto tempo ainda haveria vida
civil?

Do ponto de vista do fascista, como preconiza a ocupação total do
espaço, também o tempo está esgotado: por nenhum tempo mais deve haver
vida civil despolitizada da luta total, da política do ódio, que deve
ocupar cada hospital. Prossegue, Beradt:

``Dorme de novo e sonha:

`Estou em um campo de concentração, mas todos os prisioneiros passam
muito bem, participando de jantares e assistindo peças teatrais. Penso
que é muito exagerado o que se ouve sobre os campos e então me olho no
espelho: uso o uniforme de um médico de campo de concentração e botas
altas especiais, que cintilam de tão brilhantes. Encosto"-me no arame
farpado e começo a chorar de novo.'

Esse médico precisa da palavra caricatura para definir a si mesmo -- e é
isso o que ele é, uma caricatura traçada precisa e friamente por um
lápis em seu interior, no esforço de conciliar o inconciliável. No
primeiro sonho, ele vê o perigo que existe no silenciar e a relação
entre a inação e o crime. No segundo sonho, sob o lema `Tudo é falso',
ele se tornou cúmplice das forças que odeia: sua imagem no espelho
contradiz a imagem que ele quer ter de si mesmo, no entanto suas botas
altas brilham de forma tentadora. Cheio de vergonha, ele se conduz, em
ambos os sonhos, a uma categoria em que não quer estar: ao mesmo tempo
realiza, cheio de orgulho, o desejo de ser incluído.

O médico conta ainda que, no primeiro sonho, ele se ocupara
obstinadamente da palavra \emph{Stacheldraht} {[}\emph{arame farpado}{]}
(elemento que desempenha um papel tão proeminente em seus dois sonhos;
primeiro ele pensou em \emph{Krachelstaat}, depois em
\emph{Drachelstaat} {[}\emph{palavras inexistentes em alemão, mas que
giram ao redor de Staat}, \emph{ou seja, Estado...}{]}, mas, apesar de
toda a desconstrução joycena da palavra, não pensou em
\emph{Drachensaat} {[}\emph{literalmente `semente de dragão', expressão
que significa `pomo da discordia'; é o `ovo da serpente' de Bergman}{]},
palavra à qual, segundo ele, queria chegar, para mostrar as perigosas
consequências que arames farpados e cacos de vidro poderiam ter para
deficientes visuais.

Como se sabe, a história da \versal{SA} e dos cacos de vidro aconteceu muitos
anos depois, em 1938, na Noite dos Cristais. Esse evento contou com
detalhes que pareciam ter sido tirados do sonho do ofatalmologista:
quando os membros da \versal{SA} destruíram as vitrines de todas as lojas
judaicas, eles também quebraram, no oeste de Berlim, os vidros da
pequena loja de um cego, que foi tirado de sua cama e obrigado a
caminhar de pijama sobre os cacos. Aqui se vê mais uma vez que esse
sonhos se mantinham na esfera do possível, ou melhor, do impossível, que
estava prestes a se tronar realidade.''\footnote{Charlotte Berardt,
  \emph{op. cit.}, p. 79, 80.}

O médico invadido e aterrorizado pelo nazismo em seu próprios sonhos, em
uma política da intimidade e do inconsciente, projetava um saber
histórico sobre o próprio porvir da coisa fascista. Por que ela é
formula fixada da história, transfiguração da razão histórica em ordem
da natureza, dizai Hannah Arendt, e assim pode ser prevista em detalhes.
Como os campos de concentração eram planejamento máximo, em detalhes.
Deste modo, Charlotte Beradt conclui, o sonho do médico é um trabalho de
\emph{uma memória do futuro}.

Em algum momento entre o ano de 2012 e 2013 sonhei o seguinte sonho, que
foi anotado em algum dos apontamentos de meu livro \emph{ensaio,
fragmento}:

```Dentro de um shopping center havia acabado de ser construído um
museu, destes transparentes e modernos. Eu, meio criança, tentava ver,
mas não conseguia, \emph{a última obra de arte da moda da época}: um
filme de fragmentos de sexo explícito, com cenas entre Mira Schendel e
Alexandre Frota.' Por trás do sonho estão: a exposição de uma artista
brasileira \emph{blockbuster}, que vi em um novo museu fora do Brasil,
onde também vi os pequenos documentários de autoexibição da vida sexual
de Tracey Emin.''

Naquele momento apenas anotei o sonho e seu motivo imediato. Hoje se
trata de ir mais fundo no seu sentido histórico, a sua verdadeira
dimensão de \emph{self cultural}.

Mira Schendel é para mim a força de um impulso criador, que envolvia
toda a possibilidade de pensamento de seu tempo, que punha em trabalho
de arte as possibilidades rigorosas de uma cultura aberta ao novo, para
mim, ao melhor e ao bom. Uma trabalhadora em profundidade da cultura,
nas raízes do sentido e da linguagem, no tempo em que o próprio Brasil
era sujeito de tudo, e produzia gente assim.

Uma artista que, ao mesmo tempo em que pesquisa, mantém o cuidado da
própria intimidade, o recato e o senso de integração de não ser
devassado pela presença da arte no mundo, que não é espetáculo,
propaganda ou comércio. Exatamente o oposto de toda ordem de invasão,
vulgaridade e exibição presente no sonho de sua violação, na arte
espetacular dos shopping centers. Sua obra muito fina, em que traços e
fundamentos de letras, palavras, origens visuais do significante, em
espaço branco, base, conceito e vazio em um único ato do artista,
emergência da letra em silêncio, entre o signo e o espaço, é a
delicadeza da força real, de uma cultura que trabalha, enquanto o mundo
dorme, ou explode em redundância espetacular ao redor. O sonho ético do
moderno em trabalho, e do espaço de resgate como fonte de sentido, e não
reprodução do sempre o mesmo. Entre a Suíça, e a Europa despedaçada
pelas próprias intensidades e erros, do capitalismo imperialista do
século \versal{XX} e seu desejo de poder e alienação para a catástrofe,
destruição, que ela fazia ver como opacidade, o \emph{não brilho}
singular de seus primeiros quadros, e o Brasil, que representou terra
virgem, não saturada, aberta para as primeiras inscrições no rarefeito,
Mira contribuiu para nossa vida com afeto, singeleza e rigor, que
ligavam a obra ao gesto, e o gesto à linguagem. Ao mundo. E, também,
para mim ela sempre representou o desejo de um momento moderno em que
\emph{a Europa se curvou ao Brasil,} o que é também uma fantasia
política infantil.

Alexandre Frota, para mim, por sua vez, é o produto acabado de uma
subjetividade desde sempre determinada de modo fortemente heterônomo,
carregada de preconceitos claros e visíveis, que são ideologia expressa
diretamente em seu estado de corpo concreto. Elogio do corpo puro, só
corpo, da força sempre disponível para o constrangimento e a agressão,
seu destino objetivo na indústria da pornografia nacional é apenas a
confirmação do explícito da condensação dos elementos da cultura sem
dimensão do mercado total, que se adensaram plenamente nele, em seus
músculos, em seu cérebro e seu pau, até se tornar a coisa em si da
regressão cultural que representa tão precisamente. Alexandre Frota é um
signo cultural tão pornográfico em ação no cinema, quanto na ação comum
de seu corpo que é pura matéria, sempre disposto a violência e a
agressão, quando em simples repouso na cultura mais comum de seu mundo.
Cultura de seu corpo em busca do fetiche, que se tornou parte dominante
de nosso próprio mundo...

E o sonho, premonitório com os elementos estruturais de nosso fascismo
comum, continua, como efetivo dado histórico... Assim, no dia 29 de
outubro de 2017, cerca de um ano e meio após a retirada de Dilma
Rousseff do poder, Alexandre Frota estava realmente engajado em mais uma
atividade do grupo político de nova direita a que aderiu satisfeito
durante o processo social das manifestações contra o governo da
Presidente petista. Agora, diferente do anticomunismo maníaco, delirante
e politicamente mentiroso, revivescência arcaica nacional de um passado
do pensamento político autoritário que nunca passa, que moveu a facção
da direita mais apaixonada nos processos de manifestação pelo
impeachment, Alexandre Frota e seu grupo estavam atacando outro objeto,
novamente de modo histérico e com ameaças de passagem à violência
direta: uma \emph{exposição de arte}, uma performance artística, que
ocorria em uma temporada, normal e comum, de exposições oficias no Museu
de Arte Moderna de São Paulo. Com seu grupo de classe média
descomprometido dos destinos da cultura qualificada, dos processos de
relação entre arte e crítica e de todos os parâmetros e compromissos
operados pela arte contemporânea, Alexandre Frota atacava abertamente
\emph{a um artista}, Wagner Shwarz, atacava a instituição que o recebia
e atacava os participantes da situação artística e estética proposta
pelo artista. O sonho também era \emph{uma memória do futuro}.

O pit boy, promotor da cultura comum da briga carioca, de baixa classe
média que engajou o corpo na faceta violenta do espírito da competição
do mercado, brigava agora \emph{explicitamente} com a arte, com o
artista e com o sistema cultural que sustenta arte e educação no mundo.
O ator pornô passava ao ato, como dizem os psicanalistas, para de fato
destruir e violentar o trabalho do artista. De modo aproximado com
alguns conteúdos, explícitos, de meu sonho -- de Frota comendo Mira
Schendel, e forçando a artista, e seu mundo, para o seu próprio mundo de
violência e redução simbólica ao explícito da violência direta, assim
como a própria cultura do consumo, que é a sua, do shopping center
também o faz -- de três anos antes, sonhado mesmo na origem da
degradação da cultura política e sua nova violência de direita no
Brasil.

Frota, em conjunto com seu grupo violento, ignorante e grosseiro, dava
vazão ao novo lance político, à nova posição simbólica para a violência
política na cultura, dos grupos que promoveram o impeachment, que de
liberais passavam a ativos promotores de violência pública em nome de um
difuso e arcaizante conservadorismo, organizado como campo simbólico
\emph{antiesquerda}. Após inventarem e celebrarem com rituais públicos o
comunista inexistente de 2014 e 2015 na política petista, para um
agenciamento do ódio como força política produtora real -- de modo que o
inimigo é imaginário e efetivamente falso, mas o ódio simples e concreto
que constela a paranoia é efetivamente real e produtor de poder --
grupos de mobilização social à direita renovavam a sua política
reinventando a sua persecutoriedade e perseguição. Após muitos e muitos
anos de suspensão do direito à intervenção e ao controle da cultura,
desde o auge da ditadura militar no pós \versal{AI} 5 de 1968, setores da classe
média brasileira autoritária, e ativos inimigos da esquerda, propunham
novamente censura à arte, artistas e setores inteiros da vida da difícil
democracia brasileira.

O movimento de expansão do autoritarismo ativo e violento pela cultura,
em busca de uma nova inimizade fundamental e de um novo inimigo
projetado no lugar do monstro -- La Bete, era o nome da performance de
Wagner Shwarz atacada por Frota e pelos novos \emph{anticomunistas
culturais} -- era uma saída brilhante para a desmobilização da política
do ódio, muito importante na encenação e no engajamento da paixão na
política, como um dia disse Fernando Henrique Cardoso tentando animar a
direita para este novo tipo de ação, que fatalmente deveria esmaecer,
perder o objeto fetiche negativo, após a derrubada consumada do último
governo petista. O inimigo comunista imaginário, que deu suporte para a
política do ódio, da projeção liberada da violência, que já não existia
quando das manifestações excitadas do processo do impeachment, ficou
existindo ainda menos quando a esquerda democrática -- e pró capital --
brasileira foi afastada do governo. A direita do agenciamento do ódio
ficava sem objeto para a própria formulação da \emph{forma} de sua
política, excitada, paranoica, delirante e legitimadora da violência na
vida política cotidiana. Era necessário reinventar o inimigo, reanimar a
lógica psico"-política. Como o capital, o ódio como política \emph{não
pode parar de produzir o seu próprio excedente}, a política da
inimizade, a invenção do inimigo civilizatório universal para o
agenciamento da necropolítica dos de baixo.

E a transposição da energia de ódio disponível para o ataque à política,
do comunista inexistente para \emph{o artista existente}, \emph{o
promotor cultural existente}, \emph{o} \emph{professor de humanidades
existente}, \emph{o} \emph{filósofo crítico} ou \emph{a filósofa
crítica} existentes, apresentava imensa vantagem política efetiva. Ela
ligava, produzindo \emph{deslocamento e condensação} na cultura, a
política regressiva, herdeira das tenções históricas do século \versal{XX} e da
Guerra Fria, ao mundo contemporâneo. A violência do passado no desejo de
violência do presente. Se fixava um modelo banal de realidade social,
sempre em crise, como medida trans"-histórica das coisas, o fetichismo
positivo do homem de direita, que projeta o inimigo em uma entidade
viva, que deve fazer o papel do novo inimigo total, o que um dia foi o
``comunista'', atualizando e projetando, para sempre, a presença da
política de ódio regressiva na cultura.

Assim, se operava a transmutação significante da \emph{novilíngua},
tendente ao novo fascismo comum, da nova direita brasileira: após a
esquerda democrática petista, muito comprometida com os destinos do
mercado e do capital brasileiro, se tornar o comunista imaginado dos
anos 1950 que punha a civilização em risco absoluto -- o fetichismo
negativo do anticomunista, de fato \emph{anticomunista do nada}
brasileiro -- artistas e homens de esquerda se tornavam agora
\emph{pedófilos}, a filosofia crítica exigente de democracia, da teoria
de gênero de uma Judith Butler, se tornava promoção da perversão
infantil e familiar, professores críticos de história, filosofia e
ciências sociais se tornavam doutrinadores -- como há muito ideólogos
baratos populares da política do ódio anti"-esquerda e crítica, do tipo
Luiz Felipe Pondé, propunham nos jornais -- e todas as ações de desejo
político que não sejam imediatas com o desejo de alguma força explícita
de mercado, empresarial e de capital, se tornavam apenas o amplo e
indefinido complô comunista universal, a busca de ``hegemonia
gramsciana'' do novo comunismo \emph{cultural}, nos termos da
novilíngua, de tendência pervasiva e ilimitado, porque ele era agora
simplesmente \emph{tudo aquilo que alguém, que se projeta como um homem
de direita, não gosta. }

A política da paranoia, ativada no processo do golpe da democracia,
chegava ao limite da sua extensão, recobrindo a cultura com seu afeto e
marcando a diferença crítica, trabalho da história ou político como o
verdadeiro inimigo, \emph{a face extensa do anticomunismo do nada}, que
agora era simplesmente tudo que devia nomear o mundo. Alexandre Frota
estuprava simbolicamente todo o mundo que tivesse algum grau de contato,
qualquer que fosse, com Mira Schendel...

Novilíngua, sistema geral de erros e mentiras públicas, espetaculares,
da extensão do anticomunismo do nada sobre todo o mundo existente.
Ignorância satisfeita como política de ação direta, violência, contra
qualquer alguém, ou um alguém qualquer. Anticomunismo do nada que se
expande para toda a cultura, política e crítica. Ao prazer imediato do
desejo do inimigo do dia e da hora. Mais uma vez a direita não poupava
ninguém, e ainda menos as palavras. Assim, o muito degradado mentor
``filosófico'' do movimento ensinava cotidianamente as massas no
Youtube, prontas para a ação apaixonada de empastelar exposições de
arte, e nas noções absurdas de um Olavo de Carvalho sobre a realidade
política e cultural do Brasil dos anos de 2010: ``No Brasil de hoje tudo
é comunismo. Menos a economia.''

O ridículo evidente do sistema de conceitos, apenas errados, não é
acaso. Ninguém vai checar a máquina de delírio que procura a violência
do guru fascista da internet com história, com mediação e respeito à
verdade. Esta política é feita muito longe de qualquer desses valores e
ações de sentido, muito longe da subjetivação pelo elemento de verdade
na história. O desejo político injeta a guerra em qualquer palavra, em
qualquer sistema de signos, já diziam Klemperer e Humberto Eco, em
qualquer conceito, retirado do próprio sistema de validação, a ponto de
inventar, para o agenciamento do próprio ódio, onipotente, uma
\emph{hegemonia cultural comunista...} O que significaria uma cultura de
esquerda ativa e dominante no país do Faustão, do Jornal Nacional, da
novela das oito, do império do sertanejo industrial, da Veja, do Estado
de S. Paulo, da Jovem Pan, da Folha de S. Paulo, de centenas de rádios
evangélicas espalhadas por todo o país, muito bem pertencentes às
oligarquias político econômicas locais e onde o governo lulo"-petista
ganhava todos os louros do jogo político local por ter encaminhado as
massas trabalhadoras ao shopping. Um país em que a esquerda real
existente fez verdadeira política do elogio da forma mercadoria, e pacto
de sangue com o capital local, quando no poder.

Assim fala a novilíngua da extrema direita, apenas mentirosa
publicamente e desqualificada conceitualmente, mas quem se importa?,
\emph{tudo isso na cultura do império do mercado e seu fetichismo sobre
a vida das pessoas é de fato o império cultural do comunismo petista}.
\emph{Hoje no Brasil tudo é comunismo}, \emph{menos a economia}. Insiste
outra vez, e outra vez, o indigitado filósofo, acentuando o acinte da
política como política da gestão do absurdo, política de choque da
linguagem contra a linguagem, da violência da mentira de massas contra o
pensamento. É até ridículo, do ponto de vista de qualquer decoro na
ordem do pensamento, termos que nos dedicar à vida da mentira e a
grosseria com as noções e ignorância com a história, neste grau de
regressão espetacular que fez política real no Brasil. Tudo isso seria
dispensável, se esta política não fosse realmente \emph{eficaz}. Choque,
confusão, absurdo e tomada do poder no grito, por corruptos e
antissociais. A política da gestão do absurdo e do acinte chegou ao
poder.

Assim, se explica o novo líder da direita anti"-cultural brasileira, em
um grande jornal do Brasil, em outubro de 2017, não por acaso como se
estivéssemos em 1967, que tenta nos ensinar a nova ordem,
\emph{novilíngua}, da cultura da mobilização total da nova direita, cujo
lastro político real é a ação universal do ódio na vida. Assim ele pode
explicitar, com singeleza, e uma espécie de legitimidade cínica ao
redor, de todo campo liberal, essa máquina de absurdo das noções e
palavras, que agencia o direito à ação de ódio direta, um modo primitivo
de produzir poder:

``A Revolução Bolchevique, às vésperas de seu centenário, pôs em prática
pela primeira vez um método direto e efetivo de tomada do poder pelos
comunistas. Em 1917, uma elite dirigente foi a ponta de lança de um
movimento que, usando primeiro a força, mais tarde o terror, ditou os
rumos da antiga Rússia pelas décadas seguintes, até o regime desmoronar,
no início dos anos 90, sob o peso de sua ineficiência, injustiça e
isolamento.

Os comunistas aprenderam, com o fracasso da primeira experiência real de
socialismo, a como não fazer uma revolução. Hoje em dia está
ultrapassado o conceito de uma vanguarda partidária que age em nome do
povo.

Em seu lugar, o movimento comunista vem construindo um caminho que,
embora sinuoso, leva ao mesmo destino: a ditadura do proletariado
exaltada pelo marxismo. Ao contrário dos bolcheviques, que enfrentaram
inimigos de peito aberto, os comunistas atuais são sibilinos e
ardilosos. Aprenderam com o filósofo italiano Antonio Gramsci a combater
o capitalismo pelos flancos mais sensíveis.

Para eles, os valores do regime são protegidos em trincheiras burguesas,
que precisam ser neutralizadas. As mais visadas são Judiciário, Forças
Armadas, partidos ditos conservadores, aparelho policial, Igreja e, por
último mas não menos importante, a família.

Nas últimas semanas assistimos a mais um capítulo dessa revolução tão
dissimulada e subliminar quanto insidiosa. Duas exposições de arte
estiveram no centro das atenções da mídia ao promoverem o contato de
crianças com quadros eróticos e a exibição de um corpo nu, tudo
inadequado para a faixa etária.

(...) Se venho a público, expondo"-me à patrulha ideológica infiltrada
nos meios de comunicação, é para denunciar tais iniciativas como parte
de um plano urdido nas esferas mais sofisticadas do esquerdismo -- ameaça
que, não se enganem é tão mais real quanto elusiva. Exposições são só um
exemplo. Há muitos outros: associações de capitalismo e picaretagem na
dramartugia da tv; glorificação da bandidagem glamurosa; vitimização do
lumpen descamisado das cracolândias; certo discurso politicamente
correto nas escolas.

São todos tópicos da mesma cartilha, que visa a hegemonia cultural como
meio de chegar ao comunismo. Ante tal estratégia, Lênin e companhia
parecem um tanto ingênuos. A imensa maioria dos brasileiros que não
compactua com ditaduras de qualquer cor, resta zelar pelos valores de
nossas sociedade.''\footnote{Do futuro candidato civil da extrema
  direita, e do movimento de mentira em massa na Internet, \versal{MBL}, à
  presidência, Flávio Rocha, ``O comunista está nu'', Folha de S. Paulo,
  20/10/2017, p. 3.}

O macarthismo universal do anticomunista do nada brasileiro estava
enunciado, e bem presente no mundo da vida. Há muito ele se tornara
política consciente.

\chapter{Sobre os textos}

\textbf{\emph{Ordem} e violência no Brasil}, publicado originalmente em
\emph{Bala perdida, a violência policial no Brasil e os desafios para
sua superação}, Boitempo, Carta Maior, 2015

\textbf{Tradição da mentira tradição do ódio}, Revista Serrote, no. 23

\textbf{Crise e alucinose, anticomunismo do nada}, Revista Cult, no. 205

\textbf{A extrema direita de hoje e o Brasil: modos de usar}, Revista
Fevereiro

\textbf{Democracia de extermínio?} Revista Brasileiros, Página B

\textbf{O Estado não está sendo favorável à vida no Brasil}, Revista
Cult, no. 227

\textbf{Um político preso, um preso político}, El País Brasil

\textbf{Fascismo comum, sonho e história}, Revista Peixe"-elétrico, no. 8

Os demais textos são publicados aqui pela primeira vez.
