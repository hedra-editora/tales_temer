\begin{itemize}


\item \textbf{Michel Temer e o fascismo comum} completa a trilogia em que Tales Ab'Sáber disseca
os processos políticos que perpassaram os mandatos dos três últimos presidentes -- Lula, Dilma
Rousseff e Michel Temer, cada qual em um volume --, com enfoque sobretudo nas mentalidades
que davam sustentação a esses mesmos processos -- a ``gestão psíquica do poder'' desses governos.
Nesta terceira publicação, o autor reflete sobre como a onda liberalizante de desmonte das
conquistas sociais está intimamente relacionada à promoção do ódio e da violência, um "fascismo
comum", porque associado ao cotidiano e às práticas comuns, que entorpece a visão
e faz enxergar em exposições, cores e formas de ensino uma ameaça comunista inexistente.
  
\item \textbf{Tales Ab’Sáber}, psicanalista e ensaísta, é professor de Filosofia da Psicanálise da Universidade Federal 
de São Paulo (\textsc{unifesp}), autor de 
\emph{Lulismo, carisma pop e cultura anticrítica},
\textit{O~sonhar
restaurado: formas do sonhar em Bion, Winnicott e Freud} e
\textit{A música do tempo infinito}. 

\end{itemize}

